%
% File: pulsedopo/base/pulsedopo.tex [plain TeX code]
% Last change: January 9, 2025
%
% Derivation of the equations governing pulse propagation in optical
% parametric amplification/oscillation (OPA/OPO) processes in the presence
% of optical activity and dispersion.
%
% Copyright (C) 2025, Fredrik Jonsson, under GPL 3.0. See enclosed LICENSE.
%
\input macros/epsf.tex
\input macros/eplain.tex
\font\ninerm=cmr9
\font\twentyrm=cmr12 at 20 truept
\font\twelvesc=cmcsc10 at 12 truept
\input amssym % to get the {\Bbb E} font (strikethrough E)
\def\captionwide{\advance\leftskip by 60pt
  \advance\rightskip by 60pt}
\def\document #1 {\hsize=150mm\hoffset=4.6mm\vsize=230mm\voffset=7mm
  \topskip=0pt\baselineskip=12pt\parskip=0pt\leftskip=0pt\parindent=15pt
  \headline={\ifnum\pageno>1\ifodd\pageno\rightheadline\else\leftheadline\fi
    \else\hfill\fi}
  \def\rightheadline{\tenrm{\it #1}
    \hfil{\it\date}}
  \def\leftheadline{\tenrm{\it\date}
    \hfil{\it #1}}
  \noindent~\vskip-60pt\hskip-40pt{\epsfbox{macros/UU_logo_color.eps}}
  \vskip-42pt\hfill\vbox{\hbox{{\it\author}}
  \hbox{{\it\date}}}\vskip 36pt
  \centerline{\twelvesc #1}
  \vskip 24pt\noindent}
\def\section #1 {\bigskip\goodbreak\noindent{\bf #1}
  \par\nobreak\smallskip\noindent}
\def\subsection #1 {\bigskip\goodbreak\noindent{\it #1}
  \par\nobreak\smallskip\noindent}
\def\iint{\mathop{\int\kern-8pt\int}}
\def\iiint{\mathop{\int\kern-8pt\int\kern-8pt\int}}
\def\oiint{\mathop{\int\kern-8pt\int\kern-13.2pt{\bigcirc}}}
\def\Re{\mathop{\rm Re}\nolimits} % real part
\def\Im{\mathop{\rm Im}\nolimits} % imaginary part
\def\Tr{\mathop{\rm Tr}\nolimits} % quantum mechanical trace
\def\fourier{\mathop{\frak F}\nolimits}
\def\eqq{\mathop{\vbox{\hbox{\hskip2pt?}\vskip-6pt\hbox{=}}}}

%
% Blackboard bold fonts.
%
\font\mbb=msbm10 
\newfam\bbb
\textfont\bbb=\mbb

%
% Define in which way we want displayed equation numbers to appear in
% the text. In this case, it is preferred to have the equation numbers
% as one single running index, rather than the form otherwise commonly
% found format used in books, (<chapter number>.<equation number>).
%
\def\eqconstruct#1{#1}

%
% Define in which way we want displayed subequation numbers
% to appear in the text.
%
\newcount\subref
\def\eqsubreftext#1#2{%
  \subref = #2           % The space stops a <number>.
  \advance\subref by 96  % `a' is character code 97.
  #1{\rm\char\subref}%
}

%
% Define the 'boxit' macro from D.E. Knuths "The TeXbook, Exercise 21.3.
%
\def\boxit#1{\vbox{\hrule\hbox{\vrule\kern3pt
  \vbox{\kern3pt#1\kern3pt}\kern3pt\vrule}\hrule}}

\def\date{January 9, 2025}
\def\author{Fredrik Jonsson}
\document{Pulse propagation in chiral optical parametric processes}
\vskip24pt

\section{Conventions for the complex-valued fields}
Throughout this analysis, we stick to the convention that the real-valued
electric field ${\bf E}({\bf r},t)$ and polarization density
${\bf P}({\bf r},t)={\bf P}^{({\rm L})}({\bf r},t)+{\bf P}^{({\rm L})}({\bf r},t)$
are defined in terms of the slowly-varying complex-valued electric field
envelope ${\bf E}_{\omega}({\bf r},t)$ and polarization density envelope
${\bf P}_{\omega}({\bf r},t)$ as
$$
  \eqalignno{
    {\bf E}({\bf r},t)
      &=\sum_{\omega_{\sigma}}{\rm Re}[
           {\bf E}_{\omega_{\sigma}}({\bf r},t)\exp(-i\omega_{\sigma}t)],
      \eqdefn{eq:fields10}&\eqsubdef{eq:fields10a}\cr
    {\bf P}({\bf r},t)
      &=\sum_{\omega_{\sigma}}{\rm Re}[
           {\bf P}_{\omega_{\sigma}}({\bf r},t)\exp(-i\omega_{\sigma}t)]\cr
      &=\sum_{\omega_{\sigma}}{\rm Re}[
           ({\bf P}^{({\rm L})}_{\omega_{\sigma}}({\bf r},t)
             +{\bf P}^{({\rm NL})}_{\omega_{\sigma}}({\bf r},t))
           \exp(-i\omega_{\sigma}t)],
      &\eqsubdef{eq:fields10b}\cr
}
$$
where the complex-valued envelope ${\bf P}^{({\rm L})}_{\omega_{\sigma}}({\bf r},t)$
contains a constitutive description of all terms linear in the electric field
strength and ${\bf P}^{({\rm NL})}_{\omega_{\sigma}}({\bf r},t)$ all nonlinear terms.

\section{The linear part of the constitutive relation}
The starting point in the analysis is the wave equation for the propagation of
electromagnetic waves in chiral media, with the linear part of the electric
polarization density given as
$$
  {\bf P}^{(L)}({\bf r},t)=\varepsilon_0{\bf e}_{\mu}
    \int^{\infty}_{-\infty}\Big(
      \chi_{\mu\alpha}(-\omega;\omega)
      +\gamma_{\mu\alpha\beta}(-\omega;\omega){{\partial}\over{\partial x_{\beta}}}
    \Big)E_{\alpha}({\bf r},\omega)\exp(-i\omega t)\,d\omega.
$$
In this standard form, $\chi_{\mu\alpha}$ is the linear susceptibility tensor due
to electric dipolar interactions, while $\gamma_{\mu\alpha\beta}$ is the gyration
tensor due to electric quadrupolar interactions.

\section{The governing wave equation}
With the constitutive relation for the linear polarization density as above,
the nonlinear wave equation for the propagation, including the nonlinear part
of the electric polarization density ${\bf P}^{({\rm NL})}$ in the complex-valued
form
$$
  \eqalign{
    &
    \nabla\times\nabla\times{\bf E}_{\omega_{\sigma}}({\bf r},t)
      -{\bf e}_{\mu}
        \Big(
          k^2_{\mu\alpha}(\omega_{\sigma})
            +id_{\mu\alpha}(\omega_{\sigma}){{\partial}\over{\partial t}}
            -e_{\mu\alpha}(\omega_{\sigma}){{\partial^2}\over{\partial t^2}}
        \Big) E^{\alpha}_{\omega_{\sigma}}({\bf r},t)
      =\mu_0\omega^2_{\sigma}{\bf P}^{({\rm NL})}_{\omega_{\sigma}}({\bf r},t)\cr
  }
  \eqdef{eq:waveeq10}
$$
where the dispersive coefficients are given in their tensor form as
$$
  \eqalign{
    d_{\mu\alpha}&=d_{\mu\alpha}(\omega_{\sigma})
      \equiv 2k_{\mu\alpha}(\omega_{\sigma})
      {{d k_{\mu\alpha}}\over{d\omega}}\Big|_{\omega_{\sigma}},
      \qquad\hbox{(no sum)}\cr
    e_{\mu\alpha}&=e_{\mu\alpha}(\omega_{\sigma})
      \equiv k_{\mu\alpha}(\omega_{\sigma})
      {{d^2 k_{\mu\alpha}}\over{d\omega^2}}\Big|_{\omega_{\sigma}},
      \qquad\hbox{(no sum)}\cr
  }
$$
and where the tensor form of the wave vector $k_{\mu\alpha}(\omega)$ is defined as
$$
  k^2_{\mu\alpha}(\omega) = {{\omega^2}\over{c^2}}
      \Big(
        \delta_{\mu\alpha}
          +\chi_{\mu\alpha}(-\omega;\omega)
          +\gamma_{\mu\alpha\beta}(-\omega;\omega)
             {{\partial}\over{\partial x_{\beta}}}
      \Big).
  \eqdef{eq:waveeq12}
$$
Here the derivative is to be interpreted as the result of derivation of the
electric field, or ``the corresponding wave vector $\beta-$component of the
electric field''.

\section{Infinite plane wave approximation}
For the case of infinite plane waves, we may replace
$$
  \nabla\times\nabla\times\to-\nabla^2\to-{{d^2}\over{dz^2}},
$$
reducing Eq.~\eqref{eq:waveeq10} to
$$
  {\bf e}_{\mu}
  {{\partial^2 E^{\mu}_{\omega_{\sigma}}(z,t)}\over{\partial z^2}}
    +{\bf e}_{\mu}
      \Big(
        k^2_{\mu\alpha}(\omega_{\sigma})
          +id_{\mu\alpha}(\omega_{\sigma}){{\partial}\over{\partial t}}
          -e_{\mu\alpha}(\omega_{\sigma}){{\partial^2}\over{\partial t^2}}
      \Big)
      E^{\alpha}_{\omega_{\sigma}}(z,t)
    = -\mu_0\omega^2_{\sigma}{\bf P}^{({\rm NL})}_{\omega_{\sigma}}(z,t),
  \eqdef{eq:waveeq20}
$$
where we should keep in mind that the linear chirality is included in the
definition of $k^2_{\mu\alpha}$ in Eq.~\eqref{eq:waveeq12}, and that repeated
indices should be interpreted under the Einstein convention of summation,
here over the coordinates $\mu=x,y,z$.

As we will see in the following treatment, in Eq.~\eqref{eq:waveeq20} the term
with an explicit $k^2_{\mu\alpha}(\omega_{\sigma})$ as coefficient governs the
{\it phase velocity} in the medium in the vicinity of the centre angular
frequency $\omega_{\sigma}$, while the term with $d_{\mu\alpha}(\omega_{\sigma})$
as coefficient determines the {\it group velocity} $v_g=(\partial k_{\mu\omega}
/\partial\omega)^{-1}$, and finally the term with $e_{\mu\alpha}(\omega_{\sigma})$
as coefficient determines the {\it group velocity dispersion} in the medium.

\section{The wave vector and chiral dispersion coefficients}
If we assume that the contribution to the dispersion\numberedfootnote{In other
  words the dependence of the material parameter with respect to the angular
  frequency of light around a centre frequency $\omega_{\sigma}$.
  This dispersion, which is the primary cause for pulse broadening,
  should here not be confused by the wider concept of dispersion which
  is the cause for the, say, medium at the idler, signal and pump
  frequencies $(\omega_{\rm i},\omega_{\rm s},\omega_{\rm p})$ of a
  parametric process to possess different refractive indices, forming
  the basis of a potential phase mismatch.}
from the gyration tensor $\gamma_{\mu\alpha\beta}(-\omega;\omega)$ is negligible
compared to the dispersion of the linear electric dipolar susceptibility
$\chi_{\mu\alpha}(-\omega;\omega)$, we may express the coefficients of the
linear term of the wave equation in terms of the relative dielectric
permittivity from the dipolar interaction,
$$
  \varepsilon_{\mu\alpha}(\omega)
    \equiv\delta_{\mu\alpha}+\chi_{\mu\alpha}(-\omega;\omega),
$$
as
$$
  \eqalign{
    k_{\mu\alpha}(\omega)
      &= {{\omega}\over{c}}
        \Big(
          \varepsilon_{\mu\alpha}(\omega)
            +\gamma_{\mu\alpha\beta}(-\omega;\omega)
               {{\partial}\over{\partial x_{\beta}}}
        \Big)^{1/2}\cr
      &= {{\omega}\over{c}}\varepsilon^{1/2}_{\mu\alpha}(\omega)
        \Big(1+{{\gamma_{\mu\alpha\beta}(-\omega;\omega)}
               \over{\varepsilon_{\mu\alpha}(\omega)}}
               {{\partial}\over{\partial x_{\beta}}}
        \Big)^{1/2}\cr
        &=\big\{\hbox{$(1+\epsilon)^{1/2}\approx 1+\epsilon/2$,
                      if $\epsilon\ll 1$}\big\}\cr
      &\approx
    {{\omega}\over{c}}\varepsilon^{1/2}_{\mu\alpha}(\omega)
          +{{1}\over{2}}{{\omega}\over{c}}
            {{\gamma_{\mu\alpha\beta}(-\omega;\omega)}
               \over{\varepsilon^{1/2}_{\mu\alpha}(\omega)}}
            {{\partial}\over{\partial x_{\beta}}}.\cr
  }
  \eqdef{eq:wavevec10}
$$
In other words, in the nomenclature here used, we exclude any nonlocal or
magnetic fipolar interaction from the concept of the electric permittivity
$\varepsilon_{\mu\alpha}$, and keep this entirely as a placeholder for electric
dipolar interactions. By defining the effective gyration coefficient as
$$
  g_{\mu\alpha\beta}(\omega)\equiv
    {{\gamma_{\mu\alpha\beta}(-\omega;\omega)}
      \over{2\varepsilon^{1/2}_{\mu\alpha}(\omega)}},\qquad\hbox{(no sum)}
  \eqdef{eq:wavevec20}
$$
the wavevector $k_{\mu\alpha}$ (phase velocity) and derived dispersive
coefficients $d_{\mu\alpha}$ (group velocity) and $e_{\mu\alpha}$ (group
velocity dispersion) may hence be expressed as
$$
  \eqalignno{
    k_{\mu\alpha}(\omega_{\sigma})
      &\approx{{\omega_{\sigma}}\over{c}}
         \bigg(
           \varepsilon^{1/2}_{\mu\alpha}(\omega_{\sigma})
              +g_{\mu\alpha\beta}(\omega_{\sigma})
                 {{\partial}\over{\partial x_{\beta}}}
         \bigg),
      \eqdefn{eq:wavevec30}&\eqsubdef{eq:wavevec30a}\cr
    d_{\mu\alpha}(\omega_{\sigma})
      &\equiv 2k_{\mu\alpha}(\omega_{\sigma})
      {{d k_{\mu\alpha}(\omega)}\over{d\omega}}|_{\omega_{\sigma}}
      \qquad\hbox{(no sum)}\cr
      &\approx 2
      \underbrace{
        {{\omega}\over{c}}
         \bigg(
           \varepsilon^{1/2}_{\mu\alpha}(\omega_{\sigma})
             +g_{\mu\alpha\beta}(\omega_{\sigma}){{\partial}\over{\partial x_{\beta}}}
         \bigg)}_{k_{\mu\alpha}(\omega_{\sigma})}
      {{d}\over{d\omega}}
      \underbrace{
      \bigg(
        {{\omega}\over{c}}
          \bigg(
            \varepsilon^{1/2}_{\mu\alpha}(\omega)
              +g_{\mu\alpha\beta}(\omega){{\partial}\over{\partial x_{\beta}}}
          \bigg)
      \bigg)}_{k_{\mu\alpha}(\omega)}\bigg|_{\omega_{\sigma}},
      \ \hbox{(no sum)}\qquad
      &\eqsubdef{eq:wavevec30b}\cr
    e_{\mu\alpha}(\omega_{\sigma})
      &\equiv k_{\mu\alpha}(\omega_{\sigma})
      {{d^2 k_{\mu\alpha}(\omega)}\over{d\omega^2}}|_{\omega_{\sigma}}
      \qquad\hbox{(no sum)}\cr
      &\approx
      \underbrace{
        {{\omega}\over{c}}
         \bigg(
           \varepsilon^{1/2}_{\mu\alpha}(\omega_{\sigma})
             +g_{\mu\alpha\beta}(\omega_{\sigma}){{\partial}\over{\partial x_{\beta}}}
         \bigg)}_{k_{\mu\alpha}(\omega_{\sigma})}
      {{d^2}\over{d\omega^2}}
      \underbrace{
      \bigg(
        {{\omega}\over{c}}
          \bigg(
            \varepsilon^{1/2}_{\mu\alpha}(\omega)
              +g_{\mu\alpha\beta}(\omega){{\partial}\over{\partial x_{\beta}}}
          \bigg)
      \bigg)}_{k_{\mu\alpha}(\omega)}\bigg|_{\omega_{\sigma}}.
      \ \hbox{(no sum)}\qquad
      &\eqsubdef{eq:wavevec30c}
  }
$$
Again, it should be emphasized that in the case of an otherwise isotropic
medium, the square root of the relative electric-dipolar part of the
permittivity, $\varepsilon^{1/2}_{\mu\alpha}(\omega)$, should be interpreted
as the regular electric-dipolar part of the refractive index $n(\omega)$.%
\numberedfootnote{
  If we neglect the chiral nature of the dispersion coefficients, that is
  to say, by ignoring the gyration coefficients $g_{\mu\alpha\beta}$ inside
  the derivatives in $\omega$, we obtain the simplified coefficients for
  the group velocity and group velocity dispersion as
  $$
    \eqalign{
      d_{\mu\alpha}(\omega_{\sigma})
        &\approx 2{{\omega_{\sigma}}\over{c}}
        \bigg(
          \varepsilon^{1/2}_{\mu\alpha}(\omega_{\sigma})
            +g_{\mu\alpha\beta}(\omega_{\sigma}){{\partial}\over{\partial x_{\beta}}}
        \bigg)
        {{d}\over{d\omega}}
        \bigg(
          {{\omega\varepsilon^{1/2}_{\mu\alpha}(\omega)}\over{c}}
        \bigg)\bigg|_{\omega_{\sigma}},
        \qquad\hbox{(no sum)}\cr
      e_{\mu\alpha}(\omega_{\sigma})
        &\approx {{\omega_{\sigma}}\over{c}}
        \bigg(
          \varepsilon^{1/2}_{\mu\alpha}(\omega_{\sigma})
            +g_{\mu\alpha\beta}(\omega_{\sigma}){{\partial}\over{\partial x_{\beta}}}
        \bigg)
        {{d^2}\over{d\omega^2}}
        \bigg(
          {{\omega\varepsilon^{1/2}_{\mu\alpha}(\omega)}\over{c}}
        \bigg)\bigg|_{\omega_{\sigma}}.
        \qquad\hbox{(no sum)}\cr
    }
  $$
  However tempting this simplification is, we would in this case unfortunately
  loose any correction to the group velocity and group velocity dispersion
  from the chirality, and the only way the chirality would manifest itself
  would be through the phase velocity, hence ignoring and difference in
  temporal pulse walk-off between, say, LCP and RCP modes. In order to keep
  generality, we should hence try to keep the gyration coefficients in the
  analysis as far as possible.
}

\section{Choice of medium}
By furthermore applying the symmetries of a certain medium to the form
of the susceptibilities, we may put the wave equation in a concrete form,
considering the previously outlined approximations.
By choosing a medium belonging to point-symmetry group 32 (trigonal),
the linear polar (local) susceptibility tensor consists of three non-zero
elements of which two are independent,
$$
  \eqalign{
    \chi_{xx}&=\chi_{yy},\qquad\chi_{zz}\cr
  }
  \eqdef{eq:chi10}
$$
while the second-order, rank-three polar susceptibility tensor (governing the
optical parametric process) consists of ten non-zero elements of which four
are independent,
$$
  \eqalign{
    \chi_{xxx}&=-\chi_{xyy}=-\chi_{yxy}=-\chi_{yyx},\cr
    \chi_{xyz}&=-\chi_{yxz},\qquad
    \chi_{xzy}=-\chi_{yzx},\qquad
    \chi_{zxy}=-\chi_{zyx},\cr
  }
  \eqdef{eq:chi20}
$$
In similar, the chiral properties are described by axial (non-local) tensors,
for which the linear chiral rank-three tensor $\gamma_{\mu\alpha\beta}$ governing
the gyrotropy consists of exactly the same set of nonzero elements as
the all-electric dipolar $\chi_{\mu\alpha\beta}$,
$$
  \eqalign{
    \gamma_{xxx}&=-\gamma_{xyy}=-\gamma_{yxy}=-\gamma_{yyx},\cr
    \gamma_{xyz}&=-\gamma_{yxz},\qquad
    \gamma_{xzy}=-\gamma_{yzx},\qquad
    \gamma_{zxy}=-\gamma_{zyx},\cr
  }
  \eqdef{eq:gamma10}
$$
while the rank-four axial susceptibility tensor (governing the chiral
contribution to the optical parametric process) consists of 37 non-zero
elements of which 14 are independent,
$$
  \eqalign{
    \gamma_{xxxx}&=\gamma_{yyyy}\cr
    \gamma_{xxyy}&=\gamma_{yyxx}\cr
    \gamma_{xyxy}&=\gamma_{yxyx}\cr
                &=-\gamma_{xyyx}-\gamma_{yyxx}+\gamma_{yyyy}\cr
                &=-\gamma_{yxxy}-\gamma_{yyxx}+\gamma_{yyyy}\cr
    \gamma_{xxyz}&=-\gamma_{yyyz}=\gamma_{xyxz}=\gamma_{yxxz}\cr
    \gamma_{xxzx}&=-\gamma_{yyzy}=\gamma_{xyzx}=\gamma_{yxzx}\cr
    \gamma_{xxzz}&=\gamma_{yyzz}\cr
    \gamma_{xzxy}&=-\gamma_{yzyy}=\gamma_{xzyx}=\gamma_{yzxx}\cr
    \gamma_{xzxz}&=\gamma_{yzyz}\cr
    \gamma_{xzzx}&=\gamma_{yzzy}\cr
    \gamma_{zxxy}&=-\gamma_{zyyy}=\gamma_{zxyx}=\gamma_{zyxx}\cr
    \gamma_{zxxz}&=\gamma_{zyyz}\cr
    \gamma_{zxzx}&=\gamma_{zyzy}\cr
    \gamma_{zzxx}&=\gamma_{zzyy}\cr
    \gamma_{zzzz}&=\hbox{indep.}\cr
  }
  \eqdef{eq:gamma20}
$$
Since we here consider the propagation along the $z$-axis of infinite plane
waves in the $(x,y)$-plane, the only tensor elements which here will be of
importance in the sets listed above are the electric dipolar components
$$
  \chi_{xx}=\chi_{yy},\qquad
  \chi_{xxx}=-\chi_{xyy}=-\chi_{yxy}=-\chi_{yyx},
  \eqdef{eq:chi30}
$$
and the electric quadrupolar, or chiral, components
$$
  \gamma_{xyz}=-\gamma_{yxz},\qquad
  \gamma_{xxyz}=-\gamma_{yyyz}=\gamma_{xyxz}=\gamma_{yxxz}.
  \eqdef{eq:gamma30}
$$
In order to sort out things in order, we will in the following make a brief
pre-work before compiling the complete linear and nonlinear polarization
densities originating from this set of elements of the susceptibility tensors.

\subsection{Phase velocity terms in the 32 (trigonal) medium}
When applied to the dispersive terms in Eq.~\eqref{eq:waveeq20}, given their
form provided by Eqs.~\eqref{eq:wavevec10}--\eqref{eq:wavevec30} and an
electric field strength arbitrarily polarized in the $(x,y)$-plane, orthogonal
to the direction of propagation along the rotational symmetry axis of the 32
(trigonal) medium, this set of susceptibility tensor elements results in
$$
  {\bf e}_{\mu}k^2_{\mu\alpha}(\omega_{\sigma})E^{\alpha}_{\omega_{\sigma}}({\bf r},t)
    ={{\omega^2 n^2(\omega)}\over{c^2}}{\bf E}_{\omega_{\sigma}}({\bf r},t)
       -{{\omega^2}\over{c^2}}\gamma_{xyz}(-\omega;\omega)
          \nabla\times{\bf E}_{\omega_{\sigma}}({\bf r},t)
  \eqdef{eq:poldensp10}
$$
where $n(\omega)$ is the regular dipolar part of the refractive index, defined
by
$$
  n(\omega)=\big(1+\chi_{xx}(-\omega;\omega)\big)^{1/2}.
  \eqdef{eq:refind10}
$$
From the shape of the terms in the expression provided by
Eq.~\eqref{eq:poldensp10}, which is governing the phase
velocity in the medium, we immediately recognize ``$\nabla\times$'' as the
term which for a field propagating along the $z$-axis of rotational symmetry
of the 32 (trigonal) medium will contribute an additional, chiral part to the
regular dipolar refractive index with $\pm$ sign to the orthogonal LCP/RCP
modes, just as expected.

\subsection{Group velocity terms in the 32 (trigonal) medium}
In similar, for the first dispersive term with $d_{\mu\alpha}$ as coefficient%
\numberedfootnote{Again under the assumption of wave propagation along
  the $z$-axis of rotational symmetry in a medium belonging to the 32
  (trigonal) point-symmetry group.}
governing the group velocity, we obtain
$$
  \eqalign{
    {\bf e}_{\mu} d_{\mu\alpha}(\omega_{\sigma})
     &{{\partial E^{\alpha}_{\omega_{\sigma}}({\bf r},t)}\over{\partial t}}
       ={\bf e}_{\mu} 2k_{\mu\alpha}(\omega_{\sigma})
            {{d k_{\mu\alpha}(\omega)}\over{d\omega}}\Big|_{\omega_{\sigma}}
            {{\partial E^{\alpha}_{\omega_{\sigma}}({\bf r},t)}\over{\partial t}}\cr
      &=2{{\omega}\over{c}}
            \big(n(\omega_{\sigma})-g_{xyz}(\omega_{\sigma})\nabla\times\big)
          {{d}\over{d\omega}}
          \bigg(
            {{\omega}\over{c}}\big(n(\omega)-g_{xyz}(\omega)\nabla\times\big)
          \bigg)\bigg|_{\omega_{\sigma}}
          {{\partial{\bf E}_{\omega_{\sigma}}({\bf r},t)}\over{\partial t}}\cr
      &=2\big(k(\omega_{\sigma})-g(\omega_{\sigma})\nabla\times\big)
         \big(k'(\omega_{\sigma})-g'(\omega_{\sigma})\nabla\times\big)
          {{\partial{\bf E}_{\omega_{\sigma}}({\bf r},t)}\over{\partial t}}\cr
      &=2\big(k(\omega_{\sigma})-g(\omega_{\sigma})\nabla\times\big)
         \bigg(
           k'(\omega_{\sigma})
             {{\partial{\bf E}_{\omega_{\sigma}}({\bf r},t)}\over{\partial t}}
           -g'(\omega_{\sigma})\nabla\times
             {{\partial{\bf E}_{\omega_{\sigma}}({\bf r},t)}\over{\partial t}}
         \bigg)\cr
      &=2\bigg(
           k(\omega_{\sigma})k'(\omega_{\sigma})
             {{\partial{\bf E}_{\omega_{\sigma}}({\bf r},t)}\over{\partial t}}
           -g(\omega_{\sigma})k'(\omega_{\sigma})\nabla\times
             {{\partial{\bf E}_{\omega_{\sigma}}({\bf r},t)}\over{\partial t}}
      \cr&\qquad\qquad
           -k(\omega_{\sigma})g'(\omega_{\sigma})\nabla\times
             {{\partial{\bf E}_{\omega_{\sigma}}({\bf r},t)}\over{\partial t}}
           -g(\omega_{\sigma})g'(\omega_{\sigma})\nabla\times\nabla\times
             {{\partial{\bf E}_{\omega_{\sigma}}({\bf r},t)}\over{\partial t}}
         \bigg)\cr
      &\approx 2\bigg(
           k(\omega_{\sigma})k'(\omega_{\sigma})
             {{\partial{\bf E}_{\omega_{\sigma}}({\bf r},t)}\over{\partial t}}
           -\big(k'(\omega_{\sigma})g(\omega_{\sigma})
              +k(\omega_{\sigma})g'(\omega_{\sigma})\big)\nabla\times
             {{\partial{\bf E}_{\omega_{\sigma}}({\bf r},t)}\over{\partial t}}
         \bigg)\cr
      &=2\bigg(
           \underbrace{
           k(\omega_{\sigma})k'(\omega_{\sigma})
             {{\partial{\bf E}_{\omega_{\sigma}}({\bf r},t)}\over{\partial t}}
           }_{\vbox{\hbox{``Classic'' group velocity}
\vskip-4pt\hbox{\qquad$v_g(\omega)\sim 1/k'(\omega)$}
}}
           -\underbrace{
             {{d\big(k(\omega)g(\omega)\big)}\over{d\omega}}
           \Big|_{\omega_{\sigma}} {{\partial}\over{\partial t}}
           \nabla\times{\bf E}_{\omega_{\sigma}}({\bf r},t)
             }_{\vbox{\hbox{Chiral modification to}\vskip-4pt\hbox{the group velocity}}}
         \bigg),
  }
  \eqdef{eq:poldensp20}
$$
where we in the approximate step dropped the ``$\nabla\times\nabla\times$''
term, and where we adopted the following short-hand notations, starting with
the dipolar contribution $k(\omega)$ to the magnitude of the wave vector,
$$
  k(\omega_{\sigma})\equiv{{\omega_{\sigma} n(\omega_{\sigma})}\over{c}},\qquad
  k'(\omega_{\sigma})\equiv{{dk(\omega)}\over{d\omega}}\Big|_{\omega_{\sigma}},\qquad
  k''(\omega_{\sigma})\equiv{{d^2k(\omega)}\over{d\omega^2}}\Big|_{\omega_{\sigma}},
  \eqdef{eq:dipolwavevec10}
$$
and for the gyration coefficient $g(\omega)$ and its dispersive components,
$$
  g(\omega_{\sigma})\equiv{{\omega_{\sigma} g_{xyz}(\omega_{\sigma})}\over{c}}
    ={{\omega_{\sigma}\gamma_{xyz}(-\omega_{\sigma};\omega_{\sigma})}
       \over{cn(\omega_{\sigma})}},\quad
  g'(\omega_{\sigma})\equiv{{dg(\omega)}\over{d\omega}}\Big|_{\omega_{\sigma}},\quad
  g''(\omega_{\sigma})\equiv{{d^2g(\omega)}\over{d\omega^2}}\Big|_{\omega_{\sigma}}.
  \eqdef{eq:gyrcoeff10}
$$
To summarize the product $k(\omega)g(\omega)$ appearing as a derivative in the
final line in the derivation of the dispersive terms, this is given as
$$
  k(\omega)g(\omega)={{\omega^2}\over{c^2}}\gamma_{xyz}(-\omega;\omega),
$$
that is to say a purely chiral component without any inclusion of electric
dipolar interactions.

Judging from the shape of the final terms in Eq.~\eqref{eq:poldensp20},
we may immediately anticipate that in terms of a circularly polarized
decomposition, the ``$\nabla\times$'' term will make a differential
contribution to the LCP/RCP components with equal magnitude but with
opposite sign, just in the way the gyration term in Eq.~\eqref{eq:poldensp10}
with a similar ``$\nabla\times$'' term will affect the phase velocity.

\subsection{Group velocity dispersion terms in the 32 (trigonal) medium}
For the second dispersive term with $e_{\mu\alpha}$ as coefficient%
\numberedfootnote{Yet again under the assumption of wave propagation along
  the $z$-axis of rotational symmetry in a medium belonging to the 32
  (trigonal) point-symmetry group.}
governing the group velocity dispersion, we obtain
$$
  \eqalign{
    {\bf e}_{\mu} e_{\mu\alpha}&(\omega_{\sigma})
     {{\partial^2 E^{\alpha}_{\omega_{\sigma}}({\bf r},t)}\over{\partial t^2}}
       ={\bf e}_{\mu} k_{\mu\alpha}(\omega_{\sigma})
          {{d^2 k_{\mu\alpha}(\omega)}\over{d\omega^2}}\Big|_{\omega_{\sigma}}
          {{\partial^2 E^{\alpha}_{\omega_{\sigma}}({\bf r},t)}\over{\partial t^2}}\cr
      &={{\omega}\over{c}}
            \big(n(\omega_{\sigma})-g_{xyz}(\omega_{\sigma})\nabla\times\big)
          {{d^2}\over{d\omega^2}}
          \bigg(
            {{\omega}\over{c}}\big(n(\omega)-g_{xyz}(\omega)\nabla\times\big)
          \bigg)\bigg|_{\omega_{\sigma}}
          {{\partial^2{\bf E}_{\omega_{\sigma}}({\bf r},t)}\over{\partial t^2}}\cr
      &=\big(k(\omega_{\sigma})-g(\omega_{\sigma})\nabla\times\big)
         \big(k''(\omega_{\sigma})-g''(\omega_{\sigma})\nabla\times\big)
          {{\partial^2{\bf E}_{\omega_{\sigma}}({\bf r},t)}\over{\partial t^2}}\cr
      &=\big(k(\omega_{\sigma})-g(\omega_{\sigma})\nabla\times\big)
         \bigg(
           k''(\omega_{\sigma})
             {{\partial^2{\bf E}_{\omega_{\sigma}}({\bf r},t)}\over{\partial t^2}}
           -g''(\omega_{\sigma})\nabla\times
             {{\partial^2{\bf E}_{\omega_{\sigma}}({\bf r},t)}\over{\partial t^2}}
         \bigg)\cr
      &=\bigg(
           k(\omega_{\sigma})k''(\omega_{\sigma})
             {{\partial^2{\bf E}_{\omega_{\sigma}}({\bf r},t)}\over{\partial t^2}}
           -g(\omega_{\sigma})k''(\omega_{\sigma})\nabla\times
             {{\partial^2{\bf E}_{\omega_{\sigma}}({\bf r},t)}\over{\partial t^2}}
      \cr&\qquad\qquad
           -k(\omega_{\sigma})g''(\omega_{\sigma})\nabla\times
             {{\partial^2{\bf E}_{\omega_{\sigma}}({\bf r},t)}\over{\partial t^2}}
           -g(\omega_{\sigma})g''(\omega_{\sigma})\nabla\times\nabla\times
             {{\partial^2{\bf E}_{\omega_{\sigma}}({\bf r},t)}\over{\partial t^2}}
         \bigg)\cr
      &\approx \bigg(
           k(\omega_{\sigma})k''(\omega_{\sigma})
             {{\partial^2{\bf E}_{\omega_{\sigma}}({\bf r},t)}\over{\partial t^2}}
           -\big(k''(\omega_{\sigma})g(\omega_{\sigma})
              +k(\omega_{\sigma})g''(\omega_{\sigma})\big)\nabla\times
             {{\partial^2{\bf E}_{\omega_{\sigma}}({\bf r},t)}\over{\partial t^2}}
         \bigg)\cr
      &=\bigg(
           \underbrace{
           k(\omega_{\sigma})k''(\omega_{\sigma})
             {{\partial^2{\bf E}_{\omega_{\sigma}}({\bf r},t)}\over{\partial t^2}}
             }_{\vbox{\hbox{``Classic'' group}\vskip-4pt\hbox{velocity dispersion}}}
          -\underbrace{
           \bigg(
             {{d^2\big(k(\omega)g(\omega)\big)}\over{d\omega^2}}
             -2k'(\omega)g'(\omega)
           \bigg)
           \Big|_{\omega_{\sigma}} {{\partial^2}\over{\partial t^2}}
           \nabla\times{\bf E}_{\omega_{\sigma}}({\bf r},t)
             }_{\vbox{\hbox{Chiral modification to the}\vskip-4pt\hbox{group velocity dispersion}}}
         \bigg),
  }
  \eqdef{eq:poldensp30}
$$
Again, just as in the case of the previous group velocity terms, we may from
the shape of the final terms in Eq.~\eqref{eq:poldensp30}, immediately
anticipate that in terms of a circularly polarized decomposition, the presence
of the ``$\nabla\times$'' term will make a differential contribution to the
LCP/RCP components with equal magnitude but with opposite sign.

\section{The linear part of the polarization density}
To recapitulate, the linear part of the polarization density is in Cartesian
coordinates expressed in terms of the linear electric dipolar susceptibility
tensor $\chi_{\mu\alpha}$ of Eq.~\eqref{eq:chi30} and linear quadrupolar
(non-local) susceptibility tensor $\gamma_{\mu\alpha\beta}$ of
Eq.~\eqref{eq:gamma30} as
$$
  \eqalign{
    \varepsilon^{-1}_0{\bf P}^{({\rm L})}_{\omega}
      &={\bf e}_x\chi_{xx}(-\omega;\omega)E^x_{\omega}
        +{\bf e}_y\chi_{yy}(-\omega;\omega)E^y_{\omega}
        +{\bf e}_x\gamma_{xyz}(-\omega;\omega)
          {{\partial E^y_{\omega}}\over{\partial z}}
        +{\bf e}_y\gamma_{yxz}(-\omega;\omega)
          {{\partial E^x_{\omega}}\over{\partial z}}\cr
      &=\chi_{xx}(-\omega;\omega)
          ({\bf e}_x E^x_{\omega} + {\bf e}_y E^y_{\omega})
        +\gamma_{xyz}(-\omega;\omega)
          \Big({\bf e}_x
            {{\partial E^y_{\omega}}\over{\partial z}}
              -{\bf e}_y{{\partial E^x_{\omega}}\over{\partial z}}
          \Big)\cr
      &={\bf e}_x \Big(
           \chi_{xx}(-\omega;\omega)E^x_{\omega}
           +\gamma_{xyz}(-\omega;\omega){{\partial E^y_{\omega}}\over{\partial z}}
        \Big)
      +{\bf e}_y \Big(
           \chi_{xx}(-\omega;\omega)E^y_{\omega}
           -\gamma_{xyz}(-\omega;\omega){{\partial E^x_{\omega}}\over{\partial z}}
        \Big).\cr
  }
$$
As we express this in terms of the formulation of the wave equation in the
presence of dispersion from Eq.~\eqref{eq:waveeq10}, we for the phase velocity,
group velocity and group velocity dispersion terms have
$$
  \eqalign{
    {\bf e}_{\mu}&
      \Big(
        \underbrace{
          k^2_{\mu\alpha}(\omega_{\sigma})E^{\alpha}_{\omega_{\sigma}}({\bf r},t)
        }_{\hbox{See Eq.~\eqref{eq:poldensp10}}}
        +i\underbrace{
           d_{\mu\alpha}(\omega_{\sigma})
            {{\partial E^{\alpha}_{\omega_{\sigma}}({\bf r},t)}\over{\partial t}}
        }_{\hbox{See Eq.~\eqref{eq:poldensp20}}}
        -\underbrace{
            e_{\mu\alpha}(\omega_{\sigma})
            {{\partial^2 E^{\alpha}_{\omega_{\sigma}}({\bf r},t)}\over{\partial t^2}}
        }_{\hbox{See Eq.~\eqref{eq:poldensp30}}}
      \Big)
    \cr&\hskip40pt\approx
        \underbrace{
          {{\omega^2 n^2(\omega)}\over{c^2}}{\bf E}_{\omega_{\sigma}}({\bf r},t)
            -{{\omega^2}\over{c^2}}\gamma_{xyz}(-\omega;\omega)
               \nabla\times{\bf E}_{\omega_{\sigma}}({\bf r},t)
        }_{\hbox{From Eq.~\eqref{eq:poldensp10}}}
    \cr&\hskip80pt
      +i\underbrace{
        2\bigg(
           k(\omega_{\sigma})k'(\omega_{\sigma})
             {{\partial{\bf E}_{\omega_{\sigma}}({\bf r},t)}\over{\partial t}}
           -a'(\omega_{\sigma}){{\partial}\over{\partial t}}
           \nabla\times{\bf E}_{\omega_{\sigma}}({\bf r},t)
         \bigg)
        }_{\hbox{From Eq.~\eqref{eq:poldensp20}}}\cr
    \cr&\hskip80pt
      -\underbrace{
        \bigg(
           k(\omega_{\sigma})k''(\omega_{\sigma})
             {{\partial^2{\bf E}_{\omega_{\sigma}}({\bf r},t)}\over{\partial t^2}}
          -b''(\omega_{\sigma}){{\partial^2}\over{\partial t^2}}
           \nabla\times{\bf E}_{\omega_{\sigma}}({\bf r},t)
         \bigg)
       }_{\hbox{From Eq.~\eqref{eq:poldensp30}}},\cr
  }
  \eqdef{eq:poldensp40}
$$
where we defined the coefficient $a'_{\sigma}(\omega)$, for the chiral
differential deviation of the group velocity of pulses as
$$
  a'_{\sigma}\equiv a'(\omega_{\sigma})\equiv
    {{d\big(k(\omega)g(\omega)\big)}\over{d\omega}}\Big|_{\omega_{\sigma}},
  \eqdef{eq:poldensp50}
$$
and the coefficient  $b''_{\sigma}(\omega)$, for the chiral differential
broadening of pulses, or equivalently the differential group velocity
dispersion, as
$$
  b''_{\sigma}\equiv b''(\omega_{\sigma})\equiv
    \bigg(
      {{d^2\big(k(\omega)g(\omega)\big)}\over{d\omega^2}}-2k'(\omega)g'(\omega)
    \bigg)\Big|_{\omega_{\sigma}}.
  \eqdef{eq:poldensp60}
$$
The ``prime'' and ``bis'' on the $a'(\omega_{\sigma})$ and $b''(\omega_{\sigma})$
coefficients are not only cosmetic ornaments to match the corresponding
$k'(\omega_{\sigma})$ and $k''(\omega_{\sigma})$ coefficients for the first and
second order partial derivatives in time, but also since these coefficients
in fact {\it are} first and second order derivatives.

As the terms of Eq.~\eqref{eq:poldensp40} are inserted into the wave
equation~\eqref{eq:waveeq10} we hence obtain the vector wave equation
for propagation along the $z$-axis as\numberedfootnote{In this form, one
  might think that the $a'(\omega_{\sigma})$ and $b''(\omega_{\sigma})$
  coefficients seem to each lack a ``$k(\omega_{\sigma})$'' factor when
  comparing to the other terms of first and second partial derivatives
  in time. However, keep in mind that the $\nabla\times$ operating on
  the electric field will produce exactly this; hence it is perfectly
  well that the $a'(\omega_{\sigma})$ and $b''(\omega_{\sigma})$ coefficients,
  which stem from the chiral nature of the medium, lack this.}
$$
  \eqalign{
    &\nabla\times\nabla\times{\bf E}_{\omega_{\sigma}}({\bf r},t)
      -\Bigg\{
          k^2_{\sigma}
          +2ik_{\sigma}k'_{\sigma}{{\partial}\over{\partial t}}
          -k_{\sigma}k''_{\sigma}{{\partial^2}\over{\partial t^2}}
       \Bigg\}{\bf E}_{\omega_{\sigma}}({\bf r},t)
    \cr&\hskip40pt
      +\Bigg\{
        {{\omega^2}\over{c^2}}\gamma_{xyz}(-\omega;\omega)
          +2ia'_{\sigma}{{\partial}\over{\partial t}}
          -b''_{\sigma}{{\partial^2}\over{\partial t^2}}
       \Bigg\}\nabla\times{\bf E}_{\omega_{\sigma}}({\bf r},t)
      =\mu_0\omega^2_{\sigma}{\bf P}^{({\rm NL})}_{\omega_{\sigma}}({\bf r},t).\cr
  }
  \eqdef{eq:poldensp70}
$$

\section{Compiling the wave equation for the temporal envelopes}
By adopting the short-hand notation for the dipolar part of the magnitude of
the wave vector as
$$
  k_{\sigma}=k(\omega_{\sigma})={{\omega_{\sigma} n(\omega_{\sigma})}\over{c}},\quad
  k'_{\sigma}=k'(\omega_{\sigma})={{d}\over{d\omega}}
           \bigg({{\omega n(\omega)}\over{c}}\bigg)\Big|_{\omega_{\sigma}},\quad
  k''_{\sigma}=k''(\omega_{\sigma})={{d^2}\over{d\omega^2}}
           \bigg({{\omega n(\omega)}\over{c}}\bigg)\Big|_{\omega_{\sigma}},
$$
the wave equation~\eqref{eq:waveeq20} in the vicinity of each centre frequency
$\omega_{\sigma}$ of the parametric process, for the envelopes
$E^{\mu}_{\omega_{\sigma}}(z,t)$ and of the inifinite plane waves, expressed
component-wise from Eq.~\eqref{eq:poldensp70} for Cartesian coordinates
$\mu=x,y$, becomes
\par\boxit{
$$
  \eqalignno{
    &{{\partial^2 E^x_{\omega_{\sigma}}(z,t)}\over{\partial z^2}}
      +\Bigg(
          k^2_{\sigma}
          +2ik_{\sigma}k'_{\sigma}{{\partial}\over{\partial t}}
          -k_{\sigma}k''_{\sigma}{{\partial^2}\over{\partial t^2}}
       \Bigg) E^x_{\omega_{\sigma}}(z,t)
    \cr&\hskip90pt
      +\Bigg(
          {{\omega^2\gamma_{\sigma}}\over{c^2}}
          +2ia'_{\sigma}{{\partial}\over{\partial t}}
          -b''_{\sigma}{{\partial^2}\over{\partial t^2}}
       \Bigg) {{\partial E^y_{\omega_{\sigma}}(z,t)}\over{\partial z}}
      =-\mu_0\omega^2_{\sigma} P^{({\rm NL})x}_{\omega_{\sigma}}(z,t),\qquad
    \eqdefn{eq:waveeq40}&\eqsubdef{eq:waveeq40a}\cr
    &{{\partial^2 E^y_{\omega_{\sigma}}(z,t)}\over{\partial z^2}}
      +\Bigg(
          k^2_{\sigma}
          +2ik_{\sigma}k'_{\sigma}{{\partial}\over{\partial t}}
          -k_{\sigma}k''_{\sigma}{{\partial^2}\over{\partial t^2}}
       \Bigg) E^y_{\omega_{\sigma}}(z,t)
    \cr&\hskip90pt
      -\Bigg(
          {{\omega^2\gamma_{\sigma}}\over{c^2}}
          +2ia'_{\sigma}{{\partial}\over{\partial t}}
          -b''_{\sigma}{{\partial^2}\over{\partial t^2}}
       \Bigg) {{\partial E^x_{\omega_{\sigma}}(z,t)}\over{\partial z}}
      =-\mu_0\omega^2_{\sigma} P^{({\rm NL})y}_{\omega_{\sigma}}(z,t),\qquad
    &\eqsubdef{eq:waveeq40b}\cr
  }
$$
}\noindent
where, just to recapitulate the notation, the coefficients $a'_{\sigma}\equiv
a'(\omega_{\sigma})$ and $b''_{\sigma}\equiv b''(\omega_{\sigma})$ are given by
Eqs.~\eqref{eq:poldensp50} and~\eqref{eq:poldensp60}, respectively, and where
$$
  n_{\sigma}\equiv n(\omega_{\sigma})
    =(1+\chi_{xx}(-\omega_{\sigma};\omega_{\sigma}))^{1/2}
$$
is the dipolar linear part of the regular refractive index, and
$$
  \gamma_{\sigma}\equiv\gamma_{xyz}(-\omega_{\sigma};\omega_{\sigma})
$$
the corresponding chiral correction to the refractive index from the
non-local (chiral) part of the light-matter interaction.

\section{The nonlinear polarization density expressed in a Cartesian
         coordinate system}
In the one-dimensional wave equation $\eqref{eq:waveeq40}$ for the respective
centre frequencies $\omega_{\sigma}=\omega_{\rm i}$, $\omega_{\rm s}$,
$\omega_{\rm p}$, we have yet to explicitly state the nature of the nonlinear
wave interactions which act as the mixer between the involved frequencies.
From the nonzero applicable sets of susceptibility elements of
Eqs.~\eqref{eq:chi30} and~\eqref{eq:gamma30}, we obtain the nonlinear
polarization density expressed in cartesian coordinates as follows.

\subsection{Nonlinear polarization density for the idler}
Given the nonzero tensor elements as listed in Eqs.~\eqref{eq:chi30}
and~\eqref{eq:gamma30}, we for the idler, at angular frequency
$\omega_1=\omega_3-\omega_2$, have the nonlinear polarization density
expressed in the Cartesian coordinate system\numberedfootnote{And, of course,
  with the susceptibilities expressed with their frequency arguments following
  the convention of P.~N.~Butcher and D.~Cotter's {The Elements of Nonlinear
  Optics} (Cambridge, 1990).}
as
$$
  \eqalign{
    \varepsilon^{-1}_0
    {\bf P}^{({\rm NL})}_{\omega_1}
     =&{\bf e}_x\big[
         \chi_{xxx}(-\omega_1;\omega_3,-\omega_2)E^x_{\omega_3}E^{x*}_{\omega_2}
         +\chi_{xyy}(-\omega_1;\omega_3,-\omega_2)E^y_{\omega_3}E^{y*}_{\omega_2}
       \big]\cr
      &+{\bf e}_y\big[
         \chi_{yxy}(-\omega_1;\omega_3,-\omega_2)E^x_{\omega_3}E^{y*}_{\omega_2}
         +\chi_{yyx}(-\omega_1;\omega_3,-\omega_2)E^y_{\omega_3}E^{x*}_{\omega_2}
       \big]\cr
      &+{\bf e}_x\big[
         \gamma_{xxyz}(-\omega_1;\omega_3,-\omega_2)
           {{\partial}\over{\partial z}}\big(E^x_{\omega_3}E^{y*}_{\omega_2}\big)
         +\gamma_{xyxz}(-\omega_1;\omega_3,-\omega_2)
           {{\partial}\over{\partial z}}\big(E^y_{\omega_3}E^{x*}_{\omega_2}\big)
       \big]\cr
      &+{\bf e}_y\big[
         \gamma_{yyyz}(-\omega_1;\omega_3,-\omega_2)
           {{\partial}\over{\partial z}}\big(E^y_{\omega_3}E^{y*}_{\omega_2}\big)
         +\gamma_{yxxz}(-\omega_1;\omega_3,-\omega_2)
           {{\partial}\over{\partial z}}\big(E^x_{\omega_3}E^{x*}_{\omega_2}\big)
       \big].\cr
  }
$$
where we for the sake of simplicity in notation omitted the expolicit arguments
to the temporal field envelopes $E^{\mu}_{\omega_k}\equiv E^{\mu}_{\omega_k}(z,t)$.
By using the relations from Eqs.~\eqref{eq:chi30} and~\eqref{eq:gamma30}, which
are results of spatial symmetry condiderations and always hold, regardless of
frequency arguments of frequency regime, we may reduce this to the single
unique elements $\chi_{xxx}$ and $\gamma_{xxyz}$ to yield
$$
  \eqalign{
    \varepsilon^{-1}_0
    {\bf P}^{({\rm NL})}_{\omega_1}
     =&\chi_{xxx}(-\omega_1;\omega_3,-\omega_2)
       \big[
         {\bf e}_x
           \big(E^x_{\omega_3}E^{x*}_{\omega_2}-E^y_{\omega_3}E^{y*}_{\omega_2}\big)
         -{\bf e}_y
           \big(E^x_{\omega_3}E^{y*}_{\omega_2}+E^y_{\omega_3}E^{x*}_{\omega_2}\big)
       \big]\cr
      &+\gamma_{xxyz}(-\omega_1;\omega_3,-\omega_2){{\partial}\over{\partial z}}
       \Big[
         {\bf e}_x
         \big(E^x_{\omega_3}E^{y*}_{\omega_2}+E^y_{\omega_3}E^{x*}_{\omega_2}\big)
        -{\bf e}_y
         \big(E^y_{\omega_3}E^{y*}_{\omega_2}-E^x_{\omega_3}E^{x*}_{\omega_2}\big)
       \Big].\cr
  }
  \eqdef{eq:nlpoldensidler10}
$$
We should here emphasize the complex conjugation of the fields associated with
the signal at $\omega_2$, connected to the negative frequency argument of
$\omega_2$ in the nonlinear susceptibility tensors
$\chi_{\mu\alpha\beta}(-\omega_1;\omega_3,-\omega_2)$ and
$\gamma_{\mu\alpha\beta\gamma}(-\omega_1;\omega_3,-\omega_2)$.

\subsection{Nonlinear polarization density for the signal}
In the same way as for the idler, we for the signal, at angular frequency
$\omega_2=\omega_3-\omega_1$, obtain the nonlinear polarization density
expressed in the Cartesian coordinate system as
$$
  \eqalign{
    \varepsilon^{-1}_0
    {\bf P}^{({\rm NL})}_{\omega_2}
     =&\chi_{xxx}(-\omega_2;\omega_3,-\omega_1)
       \big[
         {\bf e}_x
           \big(E^x_{\omega_3}E^{x*}_{\omega_1}-E^y_{\omega_3}E^{y*}_{\omega_1}\big)
         -{\bf e}_y
           \big(E^x_{\omega_3}E^{y*}_{\omega_1}+E^y_{\omega_3}E^{x*}_{\omega_1}\big)
       \big]\cr
      &+\gamma_{xxyz}(-\omega_2;\omega_3,-\omega_1){{\partial}\over{\partial z}}
       \Big[
         {\bf e}_x
         \big(E^x_{\omega_3}E^{y*}_{\omega_1}+E^y_{\omega_3}E^{x*}_{\omega_1}\big)
        -{\bf e}_y
         \big(E^y_{\omega_3}E^{y*}_{\omega_1}-E^x_{\omega_3}E^{x*}_{\omega_1}\big)
       \Big].\cr
  }
  \eqdef{eq:nlpoldenssignal10}
$$
where the only difference compared to Eq.~\eqref{eq:nlpoldensidler10} is the
swap of idler and signal angular frequencies
$\omega_1\rightleftharpoons\omega_2$.
Again, we emphasize the complex conjugation of the fields associated with
the idler at $\omega_1$, connected to the negative frequency argument of
$\omega_1$ in the nonlinear susceptibility tensors
$\chi_{\mu\alpha\beta}(-\omega_2;\omega_3,-\omega_1)$ and
$\gamma_{\mu\alpha\beta\gamma}(-\omega_2;\omega_3,-\omega_1)$.

\subsection{Nonlinear polarization density for the pump}
In the same way as for the idler and signal, we for the signal, at angular
frequency $\omega_3=\omega_1+\omega_2$, obtain the nonlinear polarization
density expressed in the Cartesian coordinate system as
$$
  \eqalign{
    \varepsilon^{-1}_0
    {\bf P}^{({\rm NL})}_{\omega_3}
     =&\chi_{xxx}(-\omega_3;\omega_1,\omega_2)
       \big[
         {\bf e}_x
           \big(E^x_{\omega_1}E^x_{\omega_2}-E^y_{\omega_1}E^y_{\omega_2}\big)
         -{\bf e}_y
           \big(E^x_{\omega_1}E^y_{\omega_2}+E^y_{\omega_1}E^x_{\omega_2}\big)
       \big]\cr
      &+\gamma_{xxyz}(-\omega_3;\omega_1,\omega_2){{\partial}\over{\partial z}}
       \Big[
         {\bf e}_x
         \big(E^x_{\omega_1}E^y_{\omega_2}+E^y_{\omega_1}E^x_{\omega_2}\big)
        -{\bf e}_y
         \big(E^y_{\omega_1}E^y_{\omega_2}-E^x_{\omega_1}E^x_{\omega_2}\big)
       \Big].\cr
  }
  \eqdef{eq:nlpoldenspump10}
$$
In contrary to the nonlinear polarization densities for the idler and signal
waves, there is in the corresponding expression for the pump at
$\omega_3=\omega_1+\omega_2$ no complex conjugation of the mixed fields,
connected to the all-positive frequency arguments of $\omega_1$ and
$\omega_2$ in the nonlinear susceptibility tensors
$\chi_{\mu\alpha\beta}(-\omega_3;\omega_1,\omega_2)$ and
$\gamma_{\mu\alpha\beta\gamma}(-\omega_3;\omega_1,\omega_2)$.

\section{Inserting the nonlinear polarization density into the wave equation}
At this stage, we have all the nonlinear polarization densities for the idler,
signal and pump derived, for the electric dipolar part (tensor
$\chi_{\mu\alpha\beta}$) as well as the non-local, quadrupolar part (tensor
$\gamma_{\mu\alpha\beta\gamma}$).
These nonlinear polarization densities will act as sources in the otherwise
homogeneous partial differential equations for the temporal field
envelopes,\numberedfootnote{By {\it temporal field envelope}, we here mean
  a field where the rapid harmonic oscillation in time has been removed
  from the equation, by the convention of notation for complex-valued
  fields as introduced in Eqs.~\eqref{eq:fields10},
  $$
    \eqalign{
      {\bf E}({\bf r},t)&=\sum_{\omega_{\sigma}}{\rm Re}[
             {\bf E}_{\omega_{\sigma}}({\bf r},t)\exp(-i\omega_{\sigma}t)],\cr
      {\bf P}({\bf r},t)&=\sum_{\omega_{\sigma}}{\rm Re}[
             ({\bf P}^{({\rm L})}_{\omega_{\sigma}}({\bf r},t)
               +{\bf P}^{({\rm NL})}_{\omega_{\sigma}}({\bf r},t))
             \exp(-i\omega_{\sigma}t)].\cr
    }
  $$
  }
and by inserting Eqs.~\eqref{eq:nlpoldensidler10}--\eqref{eq:nlpoldenspump10}
consecutively into Eq.~\eqref{eq:waveeq40} for the temporal field envelopes,
we in Cartesian coordinates obtain the respective wave equation for the total
fields of the idler, signal and pump.

It should here be emphasized that we so far have not separated these total
fields into any forward and backward traveling components; this will come
as soon as we make an ansatz of the temporal envlopes having typical
$\exp(i\omega n_k z/c)$ or $\exp(-i\omega n_k z/c)$ dependencies, for the
forward and backward traveling components, respectively.

\subsection{Wave equation for the temporal envelope of the total field of
            the idler}
To start with, by inserting Eqs.~\eqref{eq:nlpoldensidler10} into
Eq.~\eqref{eq:waveeq40}, we for the temporal field envelope of the idler
in Cartesian coordinates obtain
$$
  \eqalignno{
    &{{\partial^2 E^x_{\omega_1}}\over{\partial z^2}}
      +\bigg(
          k^2_1
          +2ik_1k'_1{{\partial}\over{\partial t}}
          -k_1k''_1{{\partial^2}\over{\partial t^2}}
       \bigg) E^x_{\omega_1}
      +\bigg(
          {{\omega^2_1\gamma_1}\over{c^2}}
          +2ia'_1{{\partial}\over{\partial t}}
          -b''_1{{\partial^2}\over{\partial t^2}}
       \bigg) {{\partial E^y_{\omega_1}}\over{\partial z}}
    \cr&\hskip80pt
    = -\Big({{\omega_1}\over{c}}\Big)^2
      \Big[
       \chi_{xxx}(-\omega_1;\omega_3,-\omega_2)
           \big(E^x_{\omega_3}E^{x*}_{\omega_2}-E^y_{\omega_3}E^{y*}_{\omega_2}\big)
      \cr&\hskip150pt
       +\gamma_{xxyz}(-\omega_1;\omega_3,-\omega_2){{\partial}\over{\partial z}}
         \big(E^x_{\omega_3}E^{y*}_{\omega_2}+E^y_{\omega_3}E^{x*}_{\omega_2}\big)
      \Big],
    \eqdefn{eq:waveeq50}&\eqsubdef{eq:waveeq50a}\cr
    &{{\partial^2 E^y_{\omega_1}}\over{\partial z^2}}
      +\bigg(
          k^2_1
          +2ik_1k'_1{{\partial}\over{\partial t}}
          -k_1k''_1{{\partial^2}\over{\partial t^2}}
       \bigg) E^y_{\omega_1}
      -\bigg(
          {{\omega^2_1\gamma_1}\over{c^2}}
          +2ia'_1{{\partial}\over{\partial t}}
          -b''_1{{\partial^2}\over{\partial t^2}}
       \bigg) {{\partial E^x_{\omega_1}}\over{\partial z}}\cr
    \cr&\hskip80pt
      = -\Big({{\omega_1}\over{c}}\Big)^2
      \Big[
       -\chi_{xxx}(-\omega_1;\omega_3,-\omega_2)
           \big(E^x_{\omega_3}E^{y*}_{\omega_2}+E^y_{\omega_3}E^{x*}_{\omega_2}\big)
      \cr&\hskip150pt
       -\gamma_{xxyz}(-\omega_1;\omega_3,-\omega_2){{\partial}\over{\partial z}}
         \big(E^y_{\omega_3}E^{y*}_{\omega_2}-E^x_{\omega_3}E^{x*}_{\omega_2}\big)
      \Big],
    &\eqsubdef{eq:waveeq50b}\cr
  }
$$
where we used $\varepsilon_0\mu_0\equiv c^{-2}$, and where we for the sake of
simplicity dropped the explicit spatial and temporal arguments of the field
envelopes $E^{\mu}_{\omega_k}\equiv E^{\mu}_{\omega_k}(z,t)$, introduced the
short-hand notation
$$
  n_k \equiv n(\omega_k)\equiv(1+\chi_{xx}(-\omega_k;\omega_k))^{1/2},\qquad
  \gamma_k \equiv \gamma(\omega_k) \equiv \gamma_{xyz}(-\omega_k;\omega_k),
$$
and, for the coefficients of dispersion,
$$
  k'_k \equiv k'(\omega_k)
       \equiv {{d}\over{d\omega}}
               \bigg({{\omega n(\omega)}\over{c}}\bigg)\bigg|_{\omega_k},\qquad
  k''_k \equiv k''(\omega_k)
        \equiv {{d^2}\over{d\omega^2}}
               \bigg({{\omega n(\omega)}\over{c}}\bigg)\bigg|_{\omega_k},
$$
while the coefficients $a'_{\sigma}\equiv a'(\omega_{\sigma})$ and
$b''_{\sigma}\equiv b''(\omega_{\sigma})$ as previously are given by
Eqs.~\eqref{eq:poldensp50} and~\eqref{eq:poldensp60}, respectively.

\subsection{Wave equation for the temporal envelope of the total field of
            the signal}
Using the same notation as introduced for the idler in the previous section,
we by inserting Eqs.~\eqref{eq:nlpoldenssignal10} into Eq.~\eqref{eq:waveeq40},
for the temporal field envelope of the signal, obtain
$$
  \eqalignno{
    &{{\partial^2 E^x_{\omega_2}}\over{\partial z^2}}
      +\bigg(
          k^2_2
          +2ik_2k'_2{{\partial}\over{\partial t}}
          -k_2k''_2{{\partial^2}\over{\partial t^2}}
       \bigg) E^x_{\omega_2}
      +\bigg(
          {{\omega^2_2\gamma_2}\over{c^2}}
          +2ia'_2{{\partial}\over{\partial t}}
          -b''_2{{\partial^2}\over{\partial t^2}}
       \bigg) {{\partial E^y_{\omega_2}}\over{\partial z}}
    \cr&\hskip80pt
      = -\Big({{\omega_2}\over{c}}\Big)^2
      \Big[
       \chi_{xxx}(-\omega_2;\omega_3,-\omega_1)
           \big(E^x_{\omega_3}E^{x*}_{\omega_1}-E^y_{\omega_3}E^{y*}_{\omega_1}\big)
      \cr&\hskip150pt
       +\gamma_{xxyz}(-\omega_2;\omega_3,-\omega_1){{\partial}\over{\partial z}}
         \big(E^x_{\omega_3}E^{y*}_{\omega_1}+E^y_{\omega_3}E^{x*}_{\omega_1}\big)
      \Big],
    \eqdefn{eq:waveeq60}&\eqsubdef{eq:waveeq60a}\cr
    &{{\partial^2 E^y_{\omega_2}}\over{\partial z^2}}
      +\bigg(
          k^2_2
          +2ik_2k'_2{{\partial}\over{\partial t}}
          -k_2k''_2{{\partial^2}\over{\partial t^2}}
       \bigg) E^y_{\omega_2}
      -\bigg(
          {{\omega^2_2\gamma_2}\over{c^2}}
          +2ia'_2{{\partial}\over{\partial t}}
          -b''_2{{\partial^2}\over{\partial t^2}}
       \bigg) {{\partial E^x_{\omega_2}}\over{\partial z}}\cr
    \cr&\hskip80pt
      = -\Big({{\omega_2}\over{c}}\Big)^2
      \Big[
       -\chi_{xxx}(-\omega_2;\omega_3,-\omega_1)
           \big(E^x_{\omega_3}E^{y*}_{\omega_1}+E^y_{\omega_3}E^{x*}_{\omega_1}\big)
      \cr&\hskip150pt
       -\gamma_{xxyz}(-\omega_2;\omega_3,-\omega_1){{\partial}\over{\partial z}}
         \big(E^y_{\omega_3}E^{y*}_{\omega_1}-E^x_{\omega_3}E^{x*}_{\omega_1}\big)
      \Big].
    &\eqsubdef{eq:waveeq60b}\cr
  }
$$

\subsection{Wave equation for the temporal envelope of the total field of
            the pump}
Finally, by inserting Eqs.~\eqref{eq:nlpoldenspump10} into
Eq.~\eqref{eq:waveeq40}, for the temporal field envelope of the pump, we obtain
$$
  \eqalignno{
    &{{\partial^2 E^x_{\omega_3}}\over{\partial z^2}}
      +\bigg(
          k^2_3
          +2ik_3k'_3{{\partial}\over{\partial t}}
          -k_3k''_3{{\partial^2}\over{\partial t^2}}
       \bigg) E^x_{\omega_3}
      +\bigg(
          {{\omega^2_3\gamma_3}\over{c^2}}
          +2ia'_3{{\partial}\over{\partial t}}
          -b''_3{{\partial^2}\over{\partial t^2}}
       \bigg) {{\partial E^y_{\omega_3}}\over{\partial z}}
    \cr&\hskip80pt
      = -\Big({{\omega_3}\over{c}}\Big)^2
      \Big[
       \chi_{xxx}(-\omega_3;\omega_1,\omega_2)
           \big(E^x_{\omega_1}E^x_{\omega_2}-E^y_{\omega_1}E^y_{\omega_2}\big)
      \cr&\hskip150pt
       +\gamma_{xxyz}(-\omega_3;\omega_1,\omega_2){{\partial}\over{\partial z}}
         \big(E^x_{\omega_1}E^y_{\omega_2}+E^y_{\omega_1}E^x_{\omega_2}\big)
      \Big],
    \eqdefn{eq:waveeq70}&\eqsubdef{eq:waveeq70a}\cr
    &{{\partial^2 E^y_{\omega_3}}\over{\partial z^2}}
      +\bigg(
          k^2_3
          +2ik_3k'_3{{\partial}\over{\partial t}}
          -k_3k''_3{{\partial^2}\over{\partial t^2}}
       \bigg) E^y_{\omega_3}
      -\bigg(
          {{\omega^2_3\gamma_3}\over{c^2}}
          +2ia'_3{{\partial}\over{\partial t}}
          -b''_3{{\partial^2}\over{\partial t^2}}
       \bigg) {{\partial E^x_{\omega_3}}\over{\partial z}}\cr
    \cr&\hskip80pt
      = -\Big({{\omega_3}\over{c}}\Big)^2
      \Big[
       -\chi_{xxx}(-\omega_3;\omega_1,\omega_2)
           \big(E^x_{\omega_1}E^y_{\omega_2}+E^y_{\omega_1}E^x_{\omega_2}\big)
      \cr&\hskip150pt
       -\gamma_{xxyz}(-\omega_3;\omega_1,\omega_2){{\partial}\over{\partial z}}
         \big(E^y_{\omega_1}E^y_{\omega_2}-E^x_{\omega_1}E^x_{\omega_2}\big)
      \Big].
    &\eqsubdef{eq:waveeq70b}\cr
  }
$$
Again, notice the adsence of complex conjugation of mixed fields in the
right-hand side of these equations for the temmporal field envelope for
the pump.

\section{Expressing the wave equations in circularly polarized basis vectors}
We {\it\'a priori} know that the gyrotropic nature introduced by the nonlocal
interaction will manifest itself as a circular birefringence. Therefore, we
may now carry through the straightforward but nevertheless somewhat cumbersome
exercise of expressing the wave
equations~\eqref{eq:waveeq50}--\eqref{eq:waveeq70} in a circularly polarized
base.

\subsection{The circularly polarized basis vectors}
By using the convention from Eq.~\eqref{eq:fields10},
$$
  {\bf E}({\bf r},t)=\sum_{\omega_{\sigma}}{\rm Re}[
           {\bf E}_{\omega_{\sigma}}({\bf r},t)\exp(-i\omega_{\sigma}t)],
$$
for the notation of complex-valued temporal field envelopes, the appropriate
sign convention for the circularly polarized base vectors ${\bf e}_+$ (left
circular polarization, LCP) and ${\bf e}_-$ (right circular polarization, RCP)
yields
$$
  {\bf e}_+={{1}\over{\sqrt{2}}}({\bf e}_x+i{\bf e}_y),\qquad
  {\bf e}_-={{1}\over{\sqrt{2}}}({\bf e}_x-i{\bf e}_y).
  \eqdef{eq:circbasis10}
$$
The orthogonality properties of these basis vectors yield
$$
  {\bf e}^*_{\pm}\cdot{\bf e}_{\pm}=1,\qquad
  {\bf e}^*_{\pm}\cdot{\bf e}_{\mp}=0,\qquad
  {\bf e}_{\pm}\times{\bf e}_{\mp}=\mp i{\bf e}_z,\qquad
  {\bf e}_{\pm}\times{\bf e}_z = \pm i{\bf e}_{\pm}.
  \eqdef{eq:circbasis20}
$$
We may equally well inversely express the Cartesian basis vectors ${\bf e}_x$
and ${\bf e}_y$ in terms of the circularly polarized basis vectors as
$$
  {\bf e}_x={{1}\over{\sqrt{2}}}({\bf e}_++{\bf e}_-),\qquad
  {\bf e}_y={{1}\over{i\sqrt{2}}}({\bf e}_+-{\bf e}_-).
  \eqdef{eq:circbasis30}
$$

\subsection{The circularly polarized field components}
The previously defined circularly polarized basis vectors provide a very
convenielt tool when it comes to the formulation of the circularly polarized
field representation, since we may just project the circularly polarized field
components $E^+_{\omega}$ (LCP) and $E^-_{\omega}$ (RCP) out of the Cartesian
representation by
$$
  \eqalignno{
    E^+_{\omega}
      &={\bf e}^*_+\cdot(E^x_{\omega}{\bf e}_x+E^y_{\omega}{\bf e}_y)
       ={{1}\over{\sqrt{2}}}({\bf e}_x-i{\bf e}_y)
          \cdot(E^x_{\omega}{\bf e}_x+E^y_{\omega}{\bf e}_y)
       ={{1}\over{\sqrt{2}}}(E^x_{\omega}-iE^y_{\omega}),
       \eqdefn{eq:circbasis40}&\eqsubdef{eq:circbasis40a}\cr
    E^-_{\omega}
      &={\bf e}^*_-\cdot(E^x_{\omega}{\bf e}_x+E^y_{\omega}{\bf e}_y)
       ={{1}\over{\sqrt{2}}}({\bf e}_x+i{\bf e}_y)
          \cdot(E^x_{\omega}{\bf e}_x+E^y_{\omega}{\bf e}_y)
       ={{1}\over{\sqrt{2}}}(E^x_{\omega}+iE^y_{\omega}).
       &\eqsubdef{eq:circbasis40b}\cr
  }
$$
Of course, the projection of a circularly polarized component from a
field which already is formulated in the circularly polarized basis is
trivial, using the orthogonality conditions
${\bf e}^*_{\pm}\cdot{\bf e}_{\pm}=1$ and
${\bf e}^*_{\pm}\cdot{\bf e}_{\mp}=0$,
$$
  E^{\pm}_{\omega}={\bf e}^*_{\pm}
    \cdot(E^+_{\omega}{\bf e}_++E^-_{\omega}{\bf e}_-).
$$
Again, we may just as well inverse these relations and express the fields in
the Cartesian coordinate system in terms of these circularly polarized
components, as
$$
  E^x_{\omega}={{1}\over{\sqrt{2}}}(E^+_{\omega}+E^-_{\omega}),\qquad
  E^y_{\omega}={{i}\over{\sqrt{2}}}(E^+_{\omega}-E^-_{\omega}).
  \eqdef{eq:circbasis50}
$$
Since we in the wave equations~\eqref{eq:waveeq50}--\eqref{eq:waveeq70}
have all fields expressed in a Cartesian system, we will in the
formulation of the equivalent systems in a circularly polarized basis
primarily use Eqs.~\eqref{eq:circbasis30} and~\eqref{eq:circbasis50}.

\subsection{The circularly polarized representation of the wave equation
            for the idler}
By expressing all fields in the wave equation~\eqref{eq:waveeq50} for the
temporal envelope of the idler at angular frequency $\omega_1=\omega_3-\omega_2$
in the circularly polarized basis from Eq.~\eqref{eq:circbasis30}, in terms of
the LCP and RCP field components as provided by Eq.~\eqref{eq:circbasis50}, we
after some cumbersome but straightforward algebra obtain\numberedfootnote{From
  a practical algebraic point, we in the derivation of this expression
  simply express all Cartesian field components in terms of their circularly
  polarized components via Eqs.~\eqref{eq:circbasis50} and solve for
  $E^+_{\omega_1}$ and $E^-_{\omega_1}$.}
\par\boxit{
$$
  \eqalign{
   &\Big[
     {{\partial^2}\over{\partial z^2}}
       +k^2_1
       +2ik_1 k'_1 {{\partial}\over{\partial t}}
       -k_1 k''_1 {{\partial^2}\over{\partial t^2}}
       \pm i\Big(
          {{\omega^2_1\gamma_1}\over{c^2}}
          +2ia'_1 {{\partial}\over{\partial t}}
          -b''_1 {{\partial^2}\over{\partial t^2}}
       \Big) {{\partial}\over{\partial z}}
    \Big] E^{\pm}_{\omega_1}
    \cr&\hskip220pt
      = -\Big({{\omega_1}\over{c}}\Big)^2
      \Big(p_1 \pm iq_1 {{\partial}\over{\partial z}}\Big)
      \big(E^{\mp}_{\omega_3}E^{\pm*}_{\omega_2}\big).\cr
  }
  \eqdef{eq:waveeq80}
$$
}\noindent
with, just to recapitulate, the short-hand notation for the coefficients as
$$
  \eqalign{
  n_1&=n(\omega_1)=[1+\chi_{xx}(-\omega_1;\omega_1)]^{1/2},\qquad
    \gamma_1=\gamma_{xyz}(-\omega_1;\omega_1),\cr
  p_1 &= 2^{1/2}\chi_{xxx}(-\omega_1;\omega_3,-\omega_2),\qquad
  q_1 = 2^{1/2}\gamma_{xxyz}(-\omega_1;\omega_3,-\omega_2),\cr
  k_1 &\equiv k(\omega_1)
       \equiv {{\omega_1 n(\omega_1)}\over{c}},\quad
  k'_1 \equiv k'(\omega_1)
       \equiv {{d}\over{d\omega}}
               \bigg({{\omega n(\omega)}\over{c}}\bigg)\bigg|_{\omega_1},\quad
  k''_1 \equiv k''(\omega_1)
        \equiv {{d^2}\over{d\omega^2}}
               \bigg({{\omega n(\omega)}\over{c}}\bigg)\bigg|_{\omega_1}.\cr
  a'_1 &\equiv a'(\omega_1)
        \equiv {{d\big(k(\omega)g(\omega)\big)}\over{d\omega}}\bigg|_{\omega_1},
        \quad
  b''_1 \equiv b''(\omega_1)
        \equiv
    \bigg(
      {{d^2\big(k(\omega)g(\omega)\big)}\over{d\omega^2}}-2k'(\omega)g'(\omega)
    \bigg)\bigg|_{\omega_1},\cr
  g(\omega)&\equiv{{\omega g_{xyz}(\omega_{\sigma})}\over{c}}
    ={{\omega\gamma_{xyz}(-\omega;\omega)}
       \over{cn(\omega)}},\quad
  g'(\omega)\equiv{{dg(\omega)}\over{d\omega}},\cr
  }
$$
where the coefficients $g(\omega_1)$, $a'_1\equiv a'(\omega_1)$ and
$b''_1\equiv b''(\omega_1)$ as previously are originally defined by
Eqs.~\eqref{eq:gyrcoeff10}, \eqref{eq:poldensp50} and~\eqref{eq:poldensp60},
respectively.

\subsection{The circularly polarized representation of the wave equation
            for the signal}
In similar to the previous expression for the idler, we by expressing all fields
in the wave equation~\eqref{eq:waveeq60} for the temporal envelope of the signal
at angular frequency $\omega_2=\omega_3-\omega_1$ in the circularly polarized
basis from Eq.~\eqref{eq:circbasis30} obtain
\par\boxit{
$$
  \eqalign{
   &\Big[
     {{\partial^2}\over{\partial z^2}}
       +k^2_2
       +2ik_2 k'_2 {{\partial}\over{\partial t}}
       -k_2 k''_2 {{\partial^2}\over{\partial t^2}}
       \pm i\Big(
          {{\omega^2_2\gamma_2}\over{c^2}}
          +2ia'_2 {{\partial}\over{\partial t}}
          -b''_2 {{\partial^2}\over{\partial t^2}}
       \Big) {{\partial}\over{\partial z}}
    \Big] E^{\pm}_{\omega_2}
    \cr&\hskip220pt
      = -\Big({{\omega_2}\over{c}}\Big)^2
      \Big(p_2 \pm iq_2 {{\partial}\over{\partial z}}\Big)
      \big(E^{\mp}_{\omega_3}E^{\pm*}_{\omega_1}\big).\cr
  }
  \eqdef{eq:waveeq90}
$$
}\noindent
with, just to recapitulate, the short-hand notation for the coefficients as
$$
  \eqalign{
  n_2&=n(\omega_2)=[1+\chi_{xx}(-\omega_2;\omega_2)]^{1/2},\qquad
    \gamma_2=\gamma_{xyz}(-\omega_2;\omega_2),\cr
  p_2 &= 2^{1/2}\chi_{xxx}(-\omega_2;\omega_3,-\omega_1),\qquad
  q_2 = 2^{1/2}\gamma_{xxyz}(-\omega_2;\omega_3,-\omega_1),\cr
  k_2 &\equiv k(\omega_2)
       \equiv {{\omega_2 n(\omega_2)}\over{c}},\quad
  k'_2 \equiv k'(\omega_2)
       \equiv {{d}\over{d\omega}}
               \bigg({{\omega n(\omega)}\over{c}}\bigg)\bigg|_{\omega_2},\quad
  k''_2 \equiv k''(\omega_2)
        \equiv {{d^2}\over{d\omega^2}}
               \bigg({{\omega n(\omega)}\over{c}}\bigg)\bigg|_{\omega_2}.\cr
  a'_2 &\equiv a'(\omega_2)
        \equiv {{d\big(k(\omega)g(\omega)\big)}\over{d\omega}}\bigg|_{\omega_2},
        \quad
  b''_2 \equiv b''(\omega_2)
        \equiv
    \bigg(
      {{d^2\big(k(\omega)g(\omega)\big)}\over{d\omega^2}}-2k'(\omega)g'(\omega)
    \bigg)\bigg|_{\omega_2},\cr
  }
$$

\subsection{The circularly polarized representation of the wave equation
            for the pump}
Finally, we by expressing all fields in the wave equation~\eqref{eq:waveeq70}
for the temporal envelope of the pump at angular frequency
$\omega_3=\omega_1+\omega_2$ in the circularly polarized basis from
Eq.~\eqref{eq:circbasis30} obtain\numberedfootnote{In the presentation of
  the field equation for the pump, notice that we have the altered order
  ``$\mp$'' of the circular polarizations, to match the way the pump fields
  enter Eqs.~\eqref{eq:waveeq80} and~\eqref{eq:waveeq90}, for the idler
  and signal, respectively. Also notice the absence of complex conjugation
  of the fields involved in the interaction with the pump beam at
  $\omega_3=\omega_1+\omega_2$.}
\par\boxit{
$$
  \eqalign{
   &\Big[
     {{\partial^2}\over{\partial z^2}}
       +k^2_3
       +2ik_3 k'_3 {{\partial}\over{\partial t}}
       -k_3 k''_3 {{\partial^2}\over{\partial t^2}}
       \mp i\Big(
          {{\omega^2_3\gamma_3}\over{c^2}}
          +2ia'_3 {{\partial}\over{\partial t}}
          -b''_3 {{\partial^2}\over{\partial t^2}}
       \Big) {{\partial}\over{\partial z}}
    \Big] E^{\mp}_{\omega_3}
    \cr&\hskip220pt
      = -\Big({{\omega_3}\over{c}}\Big)^2
      \Big(p_3 \mp iq_3 {{\partial}\over{\partial z}}\Big)
      \big(E^{\pm}_{\omega_1}E^{\pm}_{\omega_2}\big).\cr
  }
  \eqdef{eq:waveeq100}
$$
}\noindent
with, just to recapitulate, the short-hand notation for the coefficients as
$$
  \eqalign{
  n_3&=n(\omega_3)=[1+\chi_{xx}(-\omega_3;\omega_3)]^{1/2},\qquad
    \gamma_3=\gamma_{xyz}(-\omega_3;\omega_3),\cr
  p_3 &= 2^{1/2}\chi_{xxx}(-\omega_3;\omega_1,\omega_2),\qquad
  q_3 = 2^{1/2}\gamma_{xxyz}(-\omega_3;\omega_1,\omega_2),\cr
  k_3 &\equiv k(\omega_3)
       \equiv {{\omega_3 n(\omega_3)}\over{c}},\quad
  k'_3 \equiv k'(\omega_3)
       \equiv {{d}\over{d\omega}}
               \bigg({{\omega n(\omega)}\over{c}}\bigg)\bigg|_{\omega_3},\quad
  k''_3 \equiv k''(\omega_3)
        \equiv {{d^2}\over{d\omega^2}}
               \bigg({{\omega n(\omega)}\over{c}}\bigg)\bigg|_{\omega_3}.\cr
  a'_3 &\equiv a'(\omega_3)
        \equiv {{d\big(k(\omega)g(\omega)\big)}\over{d\omega}}\bigg|_{\omega_3},
        \quad
  b''_3 \equiv b''(\omega_3)
        \equiv
    \bigg(
      {{d^2\big(k(\omega)g(\omega)\big)}\over{d\omega^2}}-2k'(\omega)g'(\omega)
    \bigg)\bigg|_{\omega_3},\cr
  }
$$

\section{Separating the total fields into their forward and backward
         traveling components}
We will now move from the description of the total idler, signal and pump fields
as described by Eqs.~\eqref{eq:waveeq80}--\eqref{eq:waveeq100} and will from now
separate them into their forward and backward traveling components.
Already now, we know that the forward traveling LCP/RCP waves will follow as
$\exp(i(\omega/c)(n_k\pm\gamma_k)z)$ while the backward traveling waves will
follow $\exp(-i(\omega/c)(n_k\pm\gamma_k)z)$; however, we will for the sake of
simplicity, and to make the derivation step-wise, we will make the first
separation as
$$
  E^{\pm}_{\omega_k}(z,t)=
    E^{f\pm}_{\omega_k}(z,t)\exp(i(\omega n_k/c)z)
    +E^{b\mp}_{\omega_k}(z,t)\exp(-i(\omega n_k/c)z),
  \eqdef{eq:waveeq110}
$$
for $k=1,2,3$ for the idler, signal and pump, respectively. In this separation,
we will in the envelopes $E^{f\pm}_{\omega_k}(z,t)$ and $E^{b\pm}_{\omega_k}(z,t)$
hence still have the phase dependence from the gyrotropy ($\gamma_k$
coefficients) present, something which we will sort out in the next step of
reduction of the complexity of the wave equations for the parametric process.

By inserting the separation described by Eq.~\eqref{eq:waveeq110} into the
wave equation~\eqref{eq:waveeq80} for the idler at angular frequency~$\omega_1$,
we obtain
$$
  \eqalign{
    &\bigg\{
       {{\partial E^{f\pm}_{\omega_1}}\over{\partial z}}
       \pm {{\omega_1 \gamma_1}\over{2cn_1}}
       \Big(
         {{\partial E^{f\pm}_{\omega_1}}\over{\partial z}}+ik_1 E^{f\pm}_{\omega_1}
       \Big)
       +\Big(
         (k'_1\mp a'_1) {{\partial}\over{\partial t}}
         +{{i}\over{2}}(k''_1\mp b''_1) {{\partial^2}\over{\partial t^2}}
       \Big) E^{f\pm}_{\omega_1}
       \cr&\hskip180pt
       \pm\Big(
         i{{a'_1}\over{k_1}} {{\partial}\over{\partial t}}
         -{{b''_1}\over{2k_1}} {{\partial^2}\over{\partial t^2}}
       \Big) {{\partial E^{f\pm}_{\omega_1}}\over{\partial z}}
    \bigg\}
    \exp\Big(i{{\omega_1 n_1}\over{c}}z\Big)\cr
    &\quad+\bigg\{
       -{{\partial E^{b\mp}_{\omega_1}}\over{\partial z}}
       \pm {{\omega_1 \gamma_1}\over{2cn_1}}
       \Big(
         {{\partial E^{b\mp}_{\omega_1}}\over{\partial z}}-ik_1 E^{b\mp}_{\omega_1}
       \Big)
       +\Big(
         (k'_1\pm a'_1) {{\partial}\over{\partial t}}
         +{{i}\over{2}}(k''_1\pm b''_1) {{\partial^2}\over{\partial t^2}}
       \Big) E^{b\mp}_{\omega_1}
       \cr&\hskip180pt
       \pm\Big(
         i{{a'_1}\over{k_1}} {{\partial}\over{\partial t}}
         -{{b''_1}\over{2k_1}} {{\partial^2}\over{\partial t^2}}
       \Big) {{\partial E^{b\mp}_{\omega_1}}\over{\partial z}}
    \bigg\}
    \exp\Big(-i{{\omega_1 n_1}\over{c}}z\Big)
    \cr&\hskip80pt
      = i{{\omega_1}\over{2cn_1}}
      \Big(p_1 \pm iq_1 {{\partial}\over{\partial z}}\Big)
      \Big[
        E^{f\mp}_{\omega_3} E^{f\pm*}_{\omega_2}
          \exp\Big(
            i\Big({{\omega_3 n_3}\over{c}}-{{\omega_2 n_2}\over{c}}\Big)z
          \Big)
    \cr&\hskip200pt
       +E^{f\mp}_{\omega_3} E^{b\mp*}_{\omega_2}
          \exp\Big(
            i\Big({{\omega_3 n_3}\over{c}}+{{\omega_2 n_2}\over{c}}\Big)z
          \Big)
    \cr&\hskip200pt
       +E^{b\pm}_{\omega_3} E^{f\pm*}_{\omega_2}
          \exp\Big(
            -i\Big({{\omega_3 n_3}\over{c}}+{{\omega_2 n_2}\over{c}}\Big)z
          \Big)
    \cr&\hskip200pt
       +E^{b\pm}_{\omega_3} E^{b\mp*}_{\omega_2}
          \exp\Big(
            -i\Big({{\omega_3 n_3}\over{c}}-{{\omega_2 n_2}\over{c}}\Big)z
          \Big)
      \Big],\cr
  }
  \eqdef{eq:waveeq120}
$$
while by inserting Eq.~\eqref{eq:waveeq110} into the wave
equation~\eqref{eq:waveeq90} for the signal at angular frequency~$\omega_2$,
$$
  \eqalign{
    &\bigg\{
       {{\partial E^{f\pm}_{\omega_2}}\over{\partial z}}
       \pm {{\omega_2 \gamma_2}\over{2cn_2}}
       \Big(
         {{\partial E^{f\pm}_{\omega_2}}\over{\partial z}}+ik_2 E^{f\pm}_{\omega_2}
       \Big)
       +\Big(
         (k'_2\mp a'_2) {{\partial}\over{\partial t}}
         +{{i}\over{2}}(k''_2\mp b''_2) {{\partial^2}\over{\partial t^2}}
       \Big) E^{f\pm}_{\omega_2}
       \cr&\hskip180pt
       \pm\Big(
         i{{a'_2}\over{k_2}} {{\partial}\over{\partial t}}
         -{{b''_2}\over{2k_2}} {{\partial^2}\over{\partial t^2}}
       \Big) {{\partial E^{f\pm}_{\omega_2}}\over{\partial z}}
    \bigg\}
    \exp\Big(i{{\omega_2 n_2}\over{c}}z\Big)\cr
    &\quad+\bigg\{
       -{{\partial E^{b\mp}_{\omega_2}}\over{\partial z}}
       \pm {{\omega_2 \gamma_2}\over{2cn_2}}
       \Big(
         {{\partial E^{b\mp}_{\omega_2}}\over{\partial z}}-ik_2 E^{b\mp}_{\omega_2}
       \Big)
       +\Big(
         (k'_2\pm a'_2) {{\partial}\over{\partial t}}
         +{{i}\over{2}}(k''_2\pm b''_2) {{\partial^2}\over{\partial t^2}}
       \Big) E^{b\mp}_{\omega_2}
       \cr&\hskip180pt
       \pm\Big(
         i{{a'_2}\over{k_2}} {{\partial}\over{\partial t}}
         -{{b''_2}\over{2k_2}} {{\partial^2}\over{\partial t^2}}
       \Big) {{\partial E^{b\mp}_{\omega_2}}\over{\partial z}}
    \bigg\}
    \exp\Big(-i{{\omega_2 n_2}\over{c}}z\Big)
    \cr&\hskip80pt
      = i{{\omega_2}\over{2cn_2}}
      \Big(p_2 \pm iq_2 {{\partial}\over{\partial z}}\Big)
      \Big[
        E^{f\mp}_{\omega_3} E^{f\pm*}_{\omega_1}
          \exp\Big(
            i\Big({{\omega_3 n_3}\over{c}}-{{\omega_1 n_1}\over{c}}\Big)z
          \Big)
    \cr&\hskip200pt
       +E^{f\mp}_{\omega_3} E^{b\mp*}_{\omega_1}
          \exp\Big(
            i\Big({{\omega_3 n_3}\over{c}}+{{\omega_1 n_1}\over{c}}\Big)z
          \Big)
    \cr&\hskip200pt
       +E^{b\pm}_{\omega_3} E^{f\pm*}_{\omega_1}
          \exp\Big(
            -i\Big({{\omega_3 n_3}\over{c}}+{{\omega_1 n_1}\over{c}}\Big)z
          \Big)
    \cr&\hskip200pt
       +E^{b\pm}_{\omega_3} E^{b\mp*}_{\omega_1}
          \exp\Big(
            -i\Big({{\omega_3 n_3}\over{c}}-{{\omega_1 n_1}\over{c}}\Big)z
          \Big)
      \Big],\cr
  }
  \eqdef{eq:waveeq122}
$$
and finally, by inserting Eq.~\eqref{eq:waveeq110} into the wave
equation~\eqref{eq:waveeq100} for the pump at angular frequency~$\omega_3$,
$$
  \eqalign{
    &\bigg\{
       {{\partial E^{f\mp}_{\omega_3}}\over{\partial z}}
       \mp {{\omega_3 \gamma_3}\over{2cn_3}}
       \Big(
         {{\partial E^{f\mp}_{\omega_3}}\over{\partial z}}+ik_3 E^{f\mp}_{\omega_3}
       \Big)
       +\Big(
         (k'_3\pm a'_3) {{\partial}\over{\partial t}}
         +{{i}\over{2}}(k''_3\pm b''_3) {{\partial^2}\over{\partial t^2}}
       \Big) E^{f\mp}_{\omega_3}
       \cr&\hskip180pt
       \pm\Big(
         i{{a'_3}\over{k_3}} {{\partial}\over{\partial t}}
         -{{b''_3}\over{2k_3}} {{\partial^2}\over{\partial t^2}}
       \Big) {{\partial E^{f\mp}_{\omega_3}}\over{\partial z}}
    \bigg\}
    \exp\Big(i{{\omega_3 n_3}\over{c}}z\Big)\cr
    &\quad+\bigg\{
       -{{\partial E^{b\pm}_{\omega_3}}\over{\partial z}}
       \mp {{\omega_3 \gamma_3}\over{2cn_3}}
       \Big(
         {{\partial E^{b\pm}_{\omega_3}}\over{\partial z}}-ik_3 E^{b\pm}_{\omega_3}
       \Big)
       +\Big(
         (k'_3\mp a'_3) {{\partial}\over{\partial t}}
         +{{i}\over{2}}(k''_3\mp b'_3) {{\partial^2}\over{\partial t^2}}
       \Big) E^{b\pm}_{\omega_3}
       \cr&\hskip180pt
       \mp\Big(
         i{{a''_3}\over{k_3}} {{\partial}\over{\partial t}}
         -{{b''_3}\over{2k_3}} {{\partial^2}\over{\partial t^2}}
       \Big) {{\partial E^{b\pm}_{\omega_3}}\over{\partial z}}
    \bigg\}
    \exp\Big(-i{{\omega_3 n_3}\over{c}}z\Big)
    \cr&\hskip80pt
      = i{{\omega_3}\over{2cn_3}}
      \Big(p_3 \mp iq_3 {{\partial}\over{\partial z}}\Big)
      \Big[
        E^{f\pm}_{\omega_1} E^{f\pm}_{\omega_2}
          \exp\Big(
            i\Big({{\omega_1 n_1}\over{c}}+{{\omega_2 n_2}\over{c}}\Big)z
          \Big)
    \cr&\hskip200pt
       +E^{f\mp}_{\omega_1} E^{b\mp}_{\omega_2}
          \exp\Big(
            i\Big({{\omega_1 n_1}\over{c}}-{{\omega_2 n_2}\over{c}}\Big)z
          \Big)
    \cr&\hskip200pt
       +E^{b\mp}_{\omega_1} E^{f\pm}_{\omega_2}
          \exp\Big(
            -i\Big({{\omega_1 n_1}\over{c}}-{{\omega_2 n_2}\over{c}}\Big)z
          \Big)
    \cr&\hskip200pt
       +E^{b\mp}_{\omega_1} E^{b\mp}_{\omega_2}
          \exp\Big(
            -i\Big({{\omega_1 n_1}\over{c}}+{{\omega_2 n_2}\over{c}}\Big)z
          \Big)
      \Big],\cr
  }
  \eqdef{eq:waveeq124}
$$
where we already from the beginning dropped the second-order spatial derivative
of the fields, by applying the slowly-varying envelope approximation to the
forward and backward travelling components,
$$
  \Big|{{\partial^2 E^{f\pm}_{\omega_1}}\over{\partial z^2}}\Big|
    \ll\Big|2\Big({{\omega_1 n_1}\over{c}}\Big)
           {{\partial E^{f\pm}_{\omega_1}}\over{\partial z}}\Big|,\qquad
  \Big|{{\partial^2 E^{b\mp}_{\omega_1}}\over{\partial z^2}}\Big|
    \ll\Big|2\Big({{\omega_1 n_1}\over{c}}\Big)
           {{\partial E^{b\mp}_{\omega_1}}\over{\partial z}}\Big|,\qquad
  \hbox{etc.}
  \eqdef{eq:waveeq130}
$$
In the rather complex coupled wave equation~\eqref{eq:waveeq120}, we could in
principle already at this stage start to project out terms which are closest
to phase matching, in order to simplify the algebra. However, we will keep the
mixed form in the right-hand side for the sake of keeping generality and allow
the very same equations to be used in co-propagating as well as
counter-propagating optical parametric amplification.

In the former, {\it co-propagating} case, we have the classical setup where the
idler~($\omega_1$), signal~($\omega_2$) and pump~($\omega_3$) fields all have
their interaction taking place while propagating in the same direction. This
direction may be either in the forward or backward direction, as in the case
with multiply-resonant OPO. In other words, in this classic case, the involved
triplet of fields in Eq.~\eqref{eq:waveeq120} for the parametric process is
typically $(E^{f\pm}_{\omega_1},E^{f\pm*}_{\omega_2},E^{f\mp}_{\omega_3})$.

On the other hand in the latter, {\it counter-propagating} case, the
idler~($\omega_1$) and pump~($\omega_3$) propagate in, say, the positive
direction, while the signal~($\omega_2$) propagates in the opposite, negative
direction. Thus, by keeping these options open for a while, we may accept the
added complexity in algebraic handling for the sake of getting both cases
covered by the same analysis and reduction of terms.
In other words, in this counter-propagating case, the involved triplet of
fields in Eq.~\eqref{eq:waveeq120} for the parametric process is instead
typically $(E^{f\pm}_{\omega_1},E^{b\mp*}_{\omega_2},E^{f\mp}_{\omega_3})$.

In addition to the standard slowly-varying envelope approximation, we may
also notice that the remaining phase and amplitude changes of the remaining
envelopes in any practical applications change over a length scale which is
radically longer than the wavelength of light; hence we may safely assume that
$$
  \Big|{{\partial E^{f\pm}_{\omega_j}}\over{\partial z}}\Big|
    \ll \Big|\Big({{\omega_j n_j}\over{c}}\Big)E^{f\pm}_{\omega_j}\Big|,\qquad
  \Big|{{\partial E^{b\mp}_{\omega_j}}\over{\partial z}}\Big|
    \ll \Big|\Big({{\omega_j n_j}\over{c}}\Big)E^{b\mp}_{\omega_j}\Big|,
  \eqdef{eq:waveeq150}
$$
for $j=1,2,3$. Also, we take the opportunity to drop all cross-differential
dispersive terms such as
$$
  \Big(
    i{{a'_1}\over{k_1}} {{\partial}\over{\partial t}}
      -{{b''_1}\over{2k_1}} {{\partial^2}\over{\partial t^2}}
  \Big) {{\partial E^{f\pm}_{\omega_1}}\over{\partial z}},\qquad
  \Big(
    i{{a'_2}\over{k_2}} {{\partial}\over{\partial t}}
      -{{b''_2}\over{2k_2}} {{\partial^2}\over{\partial t^2}}
  \Big) {{\partial E^{f\pm}_{\omega_2}}\over{\partial z}},\qquad\hbox{etc.}
$$
Thus, just to take this somewhat cumbersome algebra step-by-step,
Eq.~\eqref{eq:waveeq120} is for the idler at angular frequency~$\omega_1$
reduced to
$$
  \eqalign{
    &\bigg\{
       {{\partial E^{f\pm}_{\omega_1}}\over{\partial z}}
       \pm i{{\omega^2_1 \gamma_1}\over{2c^2}} E^{f\pm}_{\omega_1}
       +\Big(
         (k'_1\mp a'_1) {{\partial}\over{\partial t}}
         +{{i}\over{2}}(k''_1\mp b''_1) {{\partial^2}\over{\partial t^2}}
       \Big) E^{f\pm}_{\omega_1}
    \bigg\}
    \exp\Big(i{{\omega_1 n_1}\over{c}}z\Big)\cr
    &\quad+\bigg\{
       -{{\partial E^{b\mp}_{\omega_1}}\over{\partial z}}
       \mp i{{\omega^2_1 \gamma_1}\over{2c^2}} E^{b\mp}_{\omega_1}
       +\Big(
         (k'_1\pm a'_1) {{\partial}\over{\partial t}}
         +{{i}\over{2}}(k''_1\pm b''_1) {{\partial^2}\over{\partial t^2}}
       \Big) E^{b\mp}_{\omega_1}
    \bigg\}
    \exp\Big(-i{{\omega_1 n_1}\over{c}}z\Big)
    \cr&\hskip80pt
      = i{{\omega_1}\over{2cn_1}}
      \Big(p_1 \pm iq_1 {{\partial}\over{\partial z}}\Big)
      \Big[
        E^{f\mp}_{\omega_3} E^{f\pm*}_{\omega_2}
          \exp\Big(
            i\Big({{\omega_3 n_3}\over{c}}-{{\omega_2 n_2}\over{c}}\Big)z
          \Big)
    \cr&\hskip200pt
       +E^{f\mp}_{\omega_3} E^{b\mp*}_{\omega_2}
          \exp\Big(
            i\Big({{\omega_3 n_3}\over{c}}+{{\omega_2 n_2}\over{c}}\Big)z
          \Big)
    \cr&\hskip200pt
       +E^{b\pm}_{\omega_3} E^{f\pm*}_{\omega_2}
          \exp\Big(
            -i\Big({{\omega_3 n_3}\over{c}}+{{\omega_2 n_2}\over{c}}\Big)z
          \Big)
    \cr&\hskip200pt
       +E^{b\pm}_{\omega_3} E^{b\mp*}_{\omega_2}
          \exp\Big(
            -i\Big({{\omega_3 n_3}\over{c}}-{{\omega_2 n_2}\over{c}}\Big)z
          \Big)
      \Big],\cr
  }
  \eqdef{eq:waveeq160}
$$
while Eq.~\eqref{eq:waveeq122} for the signal at angular frequency~$\omega_2$
is reduced to
$$
  \eqalign{
    &\bigg\{
       {{\partial E^{f\pm}_{\omega_2}}\over{\partial z}}
       \pm i{{\omega^2_2 \gamma_2}\over{2c^2}} E^{f\pm}_{\omega_2}
       +\Big(
         (k'_2\mp a'_2) {{\partial}\over{\partial t}}
         +{{i}\over{2}}(k''_2\mp b''_2) {{\partial^2}\over{\partial t^2}}
       \Big) E^{f\pm}_{\omega_2}
    \bigg\}
    \exp\Big(i{{\omega_2 n_2}\over{c}}z\Big)\cr
    &\quad+\bigg\{
       -{{\partial E^{b\mp}_{\omega_2}}\over{\partial z}}
       \mp i{{\omega^2_2 \gamma_2}\over{2c^2}} E^{b\mp}_{\omega_2}
       +\Big(
         (k'_2\pm a'_2) {{\partial}\over{\partial t}}
         +{{i}\over{2}}(k''_2\pm b''_2) {{\partial^2}\over{\partial t^2}}
       \Big) E^{b\mp}_{\omega_2}
    \bigg\}
    \exp\Big(-i{{\omega_2 n_2}\over{c}}z\Big)
    \cr&\hskip80pt
      = i{{\omega_2}\over{2cn_2}}
      \Big(p_2 \pm iq_2 {{\partial}\over{\partial z}}\Big)
      \Big[
        E^{f\mp}_{\omega_3} E^{f\pm*}_{\omega_1}
          \exp\Big(
            i\Big({{\omega_3 n_3}\over{c}}-{{\omega_1 n_1}\over{c}}\Big)z
          \Big)
    \cr&\hskip200pt
       +E^{f\mp}_{\omega_3} E^{b\mp*}_{\omega_1}
          \exp\Big(
            i\Big({{\omega_3 n_3}\over{c}}+{{\omega_1 n_1}\over{c}}\Big)z
          \Big)
    \cr&\hskip200pt
       +E^{b\pm}_{\omega_3} E^{f\pm*}_{\omega_1}
          \exp\Big(
            -i\Big({{\omega_3 n_3}\over{c}}+{{\omega_1 n_1}\over{c}}\Big)z
          \Big)
    \cr&\hskip200pt
       +E^{b\pm}_{\omega_3} E^{b\mp*}_{\omega_1}
          \exp\Big(
            -i\Big({{\omega_3 n_3}\over{c}}-{{\omega_1 n_1}\over{c}}\Big)z
          \Big)
      \Big],\cr
  }
  \eqdef{eq:waveeq162}
$$
and, finally, that Eq.~\eqref{eq:waveeq124} for the pump at angular
frequency~$\omega_3$ is reduced to
$$
  \eqalign{
    &\bigg\{
       {{\partial E^{f\mp}_{\omega_3}}\over{\partial z}}
       \mp i{{\omega^2_3 \gamma_3}\over{2c^2}} E^{f\mp}_{\omega_3}
       +\Big(
         (k'_3\pm a'_3) {{\partial}\over{\partial t}}
         +{{i}\over{2}}(k''_3\pm b''_3) {{\partial^2}\over{\partial t^2}}
       \Big) E^{f\mp}_{\omega_3}
    \bigg\}
    \exp\Big(i{{\omega_3 n_3}\over{c}}z\Big)\cr
    &\quad+\bigg\{
       -{{\partial E^{b\pm}_{\omega_3}}\over{\partial z}}
       \pm i{{\omega^2_3 \gamma_3}\over{2c^2}} E^{b\pm}_{\omega_3}
       +\Big(
         (k'_3\mp a'_3) {{\partial}\over{\partial t}}
         +{{i}\over{2}}(k''_3\mp b''_3) {{\partial^2}\over{\partial t^2}}
       \Big) E^{b\pm}_{\omega_3}
    \bigg\}
    \exp\Big(-i{{\omega_3 n_3}\over{c}}z\Big)
    \cr&\hskip80pt
      = i{{\omega_3}\over{2cn_3}}
      \Big(p_3 \mp iq_3 {{\partial}\over{\partial z}}\Big)
      \Big[
        E^{f\pm}_{\omega_1} E^{f\pm}_{\omega_2}
          \exp\Big(
            i\Big({{\omega_1 n_1}\over{c}}+{{\omega_2 n_2}\over{c}}\Big)z
          \Big)
    \cr&\hskip200pt
       +E^{f\pm}_{\omega_1} E^{b\mp}_{\omega_2}
          \exp\Big(
            i\Big({{\omega_1 n_1}\over{c}}-{{\omega_2 n_2}\over{c}}\Big)z
          \Big)
    \cr&\hskip200pt
       +E^{b\mp}_{\omega_1} E^{f\pm}_{\omega_2}
          \exp\Big(
            -i\Big({{\omega_1 n_1}\over{c}}-{{\omega_2 n_2}\over{c}}\Big)z
          \Big)
    \cr&\hskip200pt
       +E^{b\mp}_{\omega_1} E^{b\mp}_{\omega_2}
          \exp\Big(
            -i\Big({{\omega_1 n_1}\over{c}}+{{\omega_2 n_2}\over{c}}\Big)z
          \Big)
      \Big].\cr
  }
  \eqdef{eq:waveeq164}
$$
In Eqs.~\eqref{eq:waveeq160}--\eqref{eq:waveeq164}, we may find straight away,
from the appearance of the terms with coefficients $\gamma_1$, $\gamma_2$ and
$\gamma_3$, that the gyrotropic nature of the forward and backward traveling
wave envelopes can be separated by the ansatz
$$
  \eqalignno{
    E^{f\pm}_{\omega_k}(z,t)&=A^{f\pm}_{\omega_k}(z,t)
      \exp\Big(\mp i{{\omega^2_k\gamma_k}\over{2c^2}}z\Big),
        \eqdefn{eq:waveeq170}&\eqsubdef{eq:waveeq170a}\cr
    E^{b\mp}_{\omega_k}(z,t)&=A^{b\mp}_{\omega_k}(z,t)
      \exp\Big(\mp i{{\omega^2_k\gamma_k}\over{2c^2}}z\Big),
        &\eqsubdef{eq:waveeq170b}\cr
  }
$$
with $k=1,2,3$ for the idler, signal and pump, respectively. This way, carrying
out the second step in the reduction of the coupled set of nonlinear PDE:s, the
terms containing the gyration coefficients $\gamma_k$ will be cancelled.
Notice that in dividing the field into forward and backward traveling
components, the backward traveling LCP/RCP components will be connected
to the {\it conjugated} basis vectors ${\bf e}^*_{\pm}$, hence the altered
order of ``$\mp$'' in ``$E^{b\mp}_{\omega_k}$''.

Thus, by inserting the ansatz of Eq.~\eqref{eq:waveeq170} into
Eq.~\eqref{eq:waveeq160}, we for the idler at angular frequency~$\omega_1$
obtain
$$
  \eqalign{
    &\bigg\{
       {{\partial A^{f\pm}_{\omega_1}}\over{\partial z}}
       +(k'_1\mp a'_1) {{\partial A^{f\pm}_{\omega_1}}\over{\partial t}}
       +{{i}\over{2}}(k''_1\mp b''_1)
            {{\partial^2 A^{f\pm}_{\omega_1}}\over{\partial t^2}}
    \bigg\}
    \exp\Big(i\Big({{\omega_1 n_1}\over{c}}
             \mp {{\omega^2_1\gamma_1}\over{2c^2}}\Big)z\Big)
    \cr&\qquad
    +\bigg\{
       -{{\partial A^{b\mp}_{\omega_1}}\over{\partial z}}
       +(k'_1\pm a'_1) {{\partial A^{b\mp}_{\omega_1}}\over{\partial t}}
       +{{i}\over{2}}(k''_1\pm b''_1)
            {{\partial^2 A^{b\mp}_{\omega_1}}\over{\partial t^2}}
    \bigg\}
    \exp\Big(-i\Big({{\omega_1 n_1}\over{c}}z
             \pm {{\omega^2_1\gamma_1}\over{2c^2}}\Big)z\Big)
    \cr&\hskip40pt
      = i{{\omega_1}\over{2cn_1}}
      \Big(p_1 \pm iq_1 {{\partial}\over{\partial z}}\Big)
      \Big[
        A^{f\mp}_{\omega_3} A^{f\pm*}_{\omega_2}
          \exp\Big(
            i\Big({{\omega_3 n_3}\over{c}}-{{\omega_2 n_2}\over{c}}\Big)z
            \pm i\Big({{\omega^2_3\gamma_3}\over{2c^2}}
                       +{{\omega^2_2\gamma_2}\over{2c^2}}\Big)z
          \Big)
    \cr&\hskip140pt
       +A^{f\mp}_{\omega_3} A^{b\mp*}_{\omega_2}
          \exp\Big(
            i\Big({{\omega_3 n_3}\over{c}}+{{\omega_2 n_2}\over{c}}\Big)z
            \pm i\Big({{\omega^2_3\gamma_3}\over{2c^2}}
                       +{{\omega^2_2\gamma_2}\over{2c^2}}\Big)z
          \Big)
    \cr&\hskip140pt
       +A^{b\pm}_{\omega_3} A^{f\pm*}_{\omega_2}
          \exp\Big(
            -i\Big({{\omega_3 n_3}\over{c}}+{{\omega_2 n_2}\over{c}}\Big)z
            \pm i\Big({{\omega^2_3\gamma_3}\over{2c^2}}
                       +{{\omega^2_2\gamma_2}\over{2c^2}}\Big)z
          \Big)
    \cr&\hskip140pt
       +A^{b\pm}_{\omega_3} A^{b\mp*}_{\omega_2}
          \exp\Big(
            -i\Big({{\omega_3 n_3}\over{c}}-{{\omega_2 n_2}\over{c}}\Big)z
            \pm i\Big({{\omega^2_3\gamma_3}\over{2c^2}}
                       +{{\omega^2_2\gamma_2}\over{2c^2}}\Big)z
          \Big)
      \Big],\cr
  }
  \eqdef{eq:waveeq180}
$$
while we for the signal at angular frequency~$\omega_2$ obtain
$$
  \eqalign{
    &\bigg\{
       {{\partial A^{f\pm}_{\omega_2}}\over{\partial z}}
       +(k'_2\mp a'_2) {{\partial A^{f\pm}_{\omega_2}}\over{\partial t}}
       +{{i}\over{2}}(k''_2\mp b''_2)
            {{\partial^2 A^{f\pm}_{\omega_2}}\over{\partial t^2}}
    \bigg\}
    \exp\Big(i\Big({{\omega_2 n_2}\over{c}}
             \mp {{\omega^2_2\gamma_2}\over{2c^2}}\Big)z\Big)
    \cr&\qquad
    +\bigg\{
       -{{\partial A^{b\mp}_{\omega_2}}\over{\partial z}}
       +(k'_2\pm a'_2) {{\partial A^{b\mp}_{\omega_2}}\over{\partial t}}
       +{{i}\over{2}}(k''_2\pm b''_2)
            {{\partial^2 A^{b\mp}_{\omega_2}}\over{\partial t^2}}
    \bigg\}
    \exp\Big(-i\Big({{\omega_2 n_2}\over{c}}z
             \pm {{\omega^2_2\gamma_2}\over{2c^2}}\Big)z\Big)
    \cr&\hskip40pt
      = i{{\omega_2}\over{2cn_2}}
      \Big(p_2 \pm iq_2 {{\partial}\over{\partial z}}\Big)
      \Big[
        A^{f\mp}_{\omega_3} A^{f\pm*}_{\omega_1}
          \exp\Big(
            i\Big({{\omega_3 n_3}\over{c}}-{{\omega_1 n_1}\over{c}}\Big)z
            \pm i\Big({{\omega^2_3\gamma_3}\over{2c^2}}
                       +{{\omega^2_1\gamma_1}\over{2c^2}}\Big)z
          \Big)
    \cr&\hskip140pt
       +A^{f\mp}_{\omega_3} A^{b\mp*}_{\omega_1}
          \exp\Big(
            i\Big({{\omega_3 n_3}\over{c}}+{{\omega_1 n_1}\over{c}}\Big)z
            \pm i\Big({{\omega^2_3\gamma_3}\over{2c^2}}
                       +{{\omega^2_1\gamma_1}\over{2c^2}}\Big)z
          \Big)
    \cr&\hskip140pt
       +A^{b\pm}_{\omega_3} A^{f\pm*}_{\omega_1}
          \exp\Big(
            -i\Big({{\omega_3 n_3}\over{c}}+{{\omega_1 n_1}\over{c}}\Big)z
            \pm i\Big({{\omega^2_3\gamma_3}\over{2c^2}}
                       +{{\omega^2_1\gamma_1}\over{2c^2}}\Big)z
          \Big)
    \cr&\hskip140pt
       +A^{b\pm}_{\omega_3} A^{b\mp*}_{\omega_1}
          \exp\Big(
            -i\Big({{\omega_3 n_3}\over{c}}-{{\omega_1 n_1}\over{c}}\Big)z
            \pm i\Big({{\omega^2_3\gamma_3}\over{2c^2}}
                       +{{\omega^2_1\gamma_1}\over{2c^2}}\Big)z
          \Big)
      \Big],\cr
  }
  \eqdef{eq:waveeq182}
$$
and finally for the pump at angular frequency~$\omega_3$,
$$
  \eqalign{
    &\bigg\{
       {{\partial A^{f\mp}_{\omega_3}}\over{\partial z}}
       +(k'_3\pm a'_3) {{\partial A^{f\mp}_{\omega_3}}\over{\partial t}}
       +{{i}\over{2}}(k''_3\pm b''_3)
            {{\partial^2 A^{f\mp}_{\omega_3}}\over{\partial t^2}}
    \bigg\}
    \exp\Big(i\Big({{\omega_3 n_3}\over{c}}
             \pm {{\omega^2_3\gamma_3}\over{2c^2}}\Big)z\Big)
    \cr&\qquad
    +\bigg\{
       -{{\partial A^{b\pm}_{\omega_3}}\over{\partial z}}
       +(k'_3\mp a'_3) {{\partial A^{b\pm}_{\omega_3}}\over{\partial t}}
       +{{i}\over{2}}(k''_3\mp b''_3)
            {{\partial^2 A^{b\pm}_{\omega_3}}\over{\partial t^2}}
    \bigg\}
    \exp\Big(-i\Big({{\omega_3 n_3}\over{c}}z
             \mp {{\omega^2_3\gamma_3}\over{2c^2}}\Big)z\Big)
    \cr&\hskip40pt
      = i{{\omega_3}\over{2cn_3}}
      \Big(p_3 \mp iq_3 {{\partial}\over{\partial z}}\Big)
      \Big[
        A^{f\pm}_{\omega_1} A^{f\pm}_{\omega_2}
          \exp\Big(
            i\Big({{\omega_1 n_1}\over{c}}+{{\omega_2 n_2}\over{c}}\Big)z
            \mp i\Big({{\omega^2_1\gamma_1}\over{2c^2}}
                       +{{\omega^2_2\gamma_2}\over{2c^2}}\Big)z
          \Big)
    \cr&\hskip140pt
       +A^{f\pm}_{\omega_1} A^{b\mp}_{\omega_2}
          \exp\Big(
            i\Big({{\omega_1 n_1}\over{c}}-{{\omega_2 n_2}\over{c}}\Big)z
            \mp i\Big({{\omega^2_1\gamma_1}\over{2c^2}}
                       +{{\omega^2_2\gamma_2}\over{2c^2}}\Big)z
          \Big)
    \cr&\hskip140pt
       +A^{b\mp}_{\omega_1} A^{f\pm}_{\omega_2}
          \exp\Big(
            -i\Big({{\omega_1 n_1}\over{c}}-{{\omega_2 n_2}\over{c}}\Big)z
            \mp i\Big({{\omega^2_1\gamma_1}\over{2c^2}}
                       +{{\omega^2_2\gamma_2}\over{2c^2}}\Big)z
          \Big)
    \cr&\hskip140pt
       +A^{b\mp}_{\omega_1} A^{b\mp}_{\omega_2}
          \exp\Big(
            -i\Big({{\omega_1 n_1}\over{c}}+{{\omega_2 n_2}\over{c}}\Big)z
            \mp i\Big({{\omega^2_1\gamma_1}\over{2c^2}}
                       +{{\omega^2_2\gamma_2}\over{2c^2}}\Big)z
          \Big)
      \Big].\cr
  }
  \eqdef{eq:waveeq184}
$$
In these expressions, notice how the exponents of the right-hand sides of the
idler, signal and pump equations all have the same sign alteration and same
coefficients for the gyrotropic contribution to the phase evolution, for the
forward as well as backward traveling components.
It should here be particularly emphasized that in the right-hand sides of
Eqs.~\eqref{eq:waveeq180}, the spatial derivatives {\it operate on the entire
nonlinear products acting as source terms}, including the complex-valued
exponential involving the phase mismatch.

We may now assume that the idler is not self-interacting, and that we may
without compromising generality project out the forward and backward traveling
waves by multiplying by respective complex-valued exponents in the left-hand
side and average over a few spatial periods of oscillation of the light, in
the order of a wavelength. This way, we for the envelopes
$A^{f\pm}_{\omega_k}=A^{f\pm}_{\omega_k}(z,t)$ and
$A^{b\mp}_{\omega_k}=A^{b\mp}_{\omega_k}(z,t)$ of the counter-propagating waves
obtain the following system, starting with the {\it forward traveling
components} of the idler~($\omega_1$), signal~($\omega_2$) and
pump~($\omega_3$) as
$$
  \eqalignno{
    &{{\partial A^{f\pm}_{\omega_1}}\over{\partial z}}
       +(k'_1\mp a'_1) {{\partial A^{f\pm}_{\omega_1}}\over{\partial t}}
       +{{i}\over{2}}(k''_1\mp b''_1)
            {{\partial^2 A^{f\pm}_{\omega_1}}\over{\partial t^2}}
    \eqdefn{eq:waveeq190}&\eqsubdef{eq:waveeq190a}\cr
    &\hskip40pt
    =i{{\omega_1}\over{2cn_1}}
    \Big[
      p_1 A^{f\mp}_{\omega_3} A^{f\pm*}_{\omega_2}
      \pm i q_1\Big(
          {{\partial A^{f\mp}_{\omega_3}}\over{\partial z}} A^{f\pm*}_{\omega_2}
          +A^{f\mp}_{\omega_3} {{\partial A^{f\pm*}_{\omega_2}}\over{\partial z}}
          +i(k_3-k_2\pm (\beta_3+\beta_2)) A^{f\mp}_{\omega_3} A^{f\pm*}_{\omega_2}
        \Big)
    \Big]
    &\cr
    &\hskip200pt
        \times\exp(i(k_3-k_2-k_1)z\pm i(\beta_3+\beta_2+\beta_1)z)
    &\cr
    &\hskip40pt
    +i{{\omega_1}\over{2cn_1}}
    \Big[
      p_1 A^{f\mp}_{\omega_3} A^{b\mp*}_{\omega_2}
      \pm i q_1\Big(
          {{\partial A^{f\mp}_{\omega_3}}\over{\partial z}} A^{b\mp*}_{\omega_2}
          +A^{f\mp}_{\omega_3} {{\partial A^{b\mp*}_{\omega_2}}\over{\partial z}}
          +i(k_3+k_2\pm (\beta_3+\beta_2)) A^{f\mp}_{\omega_3} A^{b\mp*}_{\omega_2}
        \Big)
    \Big]
    &\cr
    &\hskip200pt
        \times\exp(i(k_3+k_2-k_1)z\pm i(\beta_3+\beta_2+\beta_1)z)
    &\cr
    &\hskip40pt
    +i{{\omega_1}\over{2cn_1}}
    \Big[
      p_1 A^{b\pm}_{\omega_3} A^{f\pm*}_{\omega_2}
      \pm i q_1\Big(
          {{\partial A^{b\pm}_{\omega_3}}\over{\partial z}} A^{f\pm*}_{\omega_2}
          +A^{b\pm}_{\omega_3} {{\partial A^{f\pm*}_{\omega_2}}\over{\partial z}}
          +i(-k_3-k_2\pm (\beta_3+\beta_2)) A^{b\pm}_{\omega_3} A^{f\pm*}_{\omega_2}
        \Big)
    \Big]
    &\cr
    &\hskip200pt
        \times\exp(-i(k_3+k_2+k_1)z\pm i(\beta_3+\beta_2+\beta_1)z)
    &\cr
    &\hskip40pt
    +i{{\omega_1}\over{2cn_1}}
    \Big[
      p_1 A^{b\pm}_{\omega_3} A^{b\mp*}_{\omega_2}
      \pm i q_1\Big(
          {{\partial A^{b\pm}_{\omega_3}}\over{\partial z}} A^{b\mp*}_{\omega_2}
          +A^{b\pm}_{\omega_3} {{\partial A^{b\mp*}_{\omega_2}}\over{\partial z}}
          +i(-k_3+k_2\pm (\beta_3+\beta_2)) A^{b\pm}_{\omega_3} A^{b\mp*}_{\omega_2}
        \Big)
    \Big]
    &\cr
    &\hskip200pt
        \times\exp(-i(k_3-k_2+k_1)z\pm i(\beta_3+\beta_2+\beta_1)z),
    &\cr
%
% --------------------------------------------------------------
%
    &{{\partial A^{f\pm}_{\omega_2}}\over{\partial z}}
       +(k'_2\mp a'_2) {{\partial A^{f\pm}_{\omega_2}}\over{\partial t}}
       +{{i}\over{2}}(k''_2\mp b''_2)
            {{\partial^2 A^{f\pm}_{\omega_2}}\over{\partial t^2}}
    &\eqsubdef{eq:waveeq190b}\cr
    &\hskip40pt
    =i{{\omega_2}\over{2cn_2}}
    \Big[
      p_2 A^{f\mp}_{\omega_3} A^{f\pm*}_{\omega_1}
      \pm i q_2\Big(
          {{\partial A^{f\mp}_{\omega_3}}\over{\partial z}} A^{f\pm*}_{\omega_1}
          +A^{f\mp}_{\omega_3} {{\partial A^{f\pm*}_{\omega_1}}\over{\partial z}}
          +i(k_3-k_1\pm (\beta_3+\beta_1)) A^{f\mp}_{\omega_3} A^{f\pm*}_{\omega_1}
        \Big)
    \Big]
    &\cr
    &\hskip200pt
        \times\exp(i(k_3-k_1-k_2)z\pm i(\beta_3+\beta_2+\beta_1)z)
    &\cr
    &\hskip40pt
    +i{{\omega_2}\over{2cn_2}}
    \Big[
      p_2 A^{f\mp}_{\omega_3} A^{b\mp*}_{\omega_1}
      \pm i q_2\Big(
          {{\partial A^{f\mp}_{\omega_3}}\over{\partial z}} A^{b\mp*}_{\omega_1}
          +A^{f\mp}_{\omega_3} {{\partial A^{b\mp*}_{\omega_1}}\over{\partial z}}
          +i(k_3+k_1\pm (\beta_3+\beta_1)) A^{f\mp}_{\omega_3} A^{b\mp*}_{\omega_1}
        \Big)
    \Big]
    &\cr
    &\hskip200pt
        \times\exp(i(k_3+k_1-k_2)z\pm i(\beta_3+\beta_2+\beta_1)z)
    &\cr
    &\hskip40pt
    +i{{\omega_2}\over{2cn_2}}
    \Big[
      p_2 A^{b\pm}_{\omega_3} A^{f\pm*}_{\omega_1}
      \pm i q_2\Big(
          {{\partial A^{b\pm}_{\omega_3}}\over{\partial z}} A^{f\pm*}_{\omega_1}
          +A^{b\pm}_{\omega_3} {{\partial A^{f\pm*}_{\omega_1}}\over{\partial z}}
          +i(-k_3-k_1\pm (\beta_3+\beta_1)) A^{b\pm}_{\omega_3} A^{f\pm*}_{\omega_1}
        \Big)
    \Big]
    &\cr
    &\hskip200pt
        \times\exp(-i(k_3+k_1+k_2)z\pm i(\beta_3+\beta_2+\beta_1)z)
    &\cr
    &\hskip40pt
    +i{{\omega_2}\over{2cn_2}}
    \Big[
      p_2 A^{b\pm}_{\omega_3} A^{b\mp*}_{\omega_1}
      \pm i q_2\Big(
          {{\partial A^{b\pm}_{\omega_3}}\over{\partial z}} A^{b\mp*}_{\omega_1}
          +A^{b\pm}_{\omega_3} {{\partial A^{b\mp*}_{\omega_1}}\over{\partial z}}
          +i(-k_3+k_1\pm (\beta_3+\beta_1)) A^{b\pm}_{\omega_3} A^{b\mp*}_{\omega_1}
        \Big)
    \Big]
    &\cr
    &\hskip200pt
        \times\exp(-i(k_3-k_1+k_2)z\pm i(\beta_3+\beta_2+\beta_1)z),
    &\cr
%
% --------------------------------------------------------------
%
    &{{\partial A^{f\mp}_{\omega_3}}\over{\partial z}}
       +(k'_3\pm a'_3) {{\partial A^{f\mp}_{\omega_3}}\over{\partial t}}
       +{{i}\over{2}}(k''_3\pm b''_3)
            {{\partial^2 A^{f\mp}_{\omega_3}}\over{\partial t^2}}
    &\eqsubdef{eq:waveeq190c}\cr
    &\hskip40pt
    =i{{\omega_3}\over{2cn_3}}
    \Big[
      p_3 A^{f\pm}_{\omega_1} A^{f\pm}_{\omega_2}
      \mp i q_3\Big(
          {{\partial A^{f\pm}_{\omega_1}}\over{\partial z}} A^{f\pm}_{\omega_2}
          +A^{f\pm}_{\omega_1} {{\partial A^{f\pm}_{\omega_2}}\over{\partial z}}
          +i(k_1+k_2\mp (\beta_1+\beta_2)) A^{f\pm}_{\omega_1} A^{f\pm}_{\omega_2}
        \Big)
    \Big]
    &\cr
    &\hskip200pt
        \times\exp(i(k_1+k_2-k_3)z\mp i(\beta_3+\beta_2+\beta_1)z)
    &\cr
    &\hskip40pt
    +i{{\omega_3}\over{2cn_3}}
    \Big[
      p_3 A^{f\pm}_{\omega_1} A^{b\mp}_{\omega_2}
      \mp i q_3\Big(
          {{\partial A^{f\pm}_{\omega_1}}\over{\partial z}} A^{b\mp}_{\omega_2}
          +A^{f\pm}_{\omega_1} {{\partial A^{b\mp}_{\omega_2}}\over{\partial z}}
          +i(k_1-k_2\mp (\beta_1+\beta_2)) A^{f\pm}_{\omega_1} A^{b\mp}_{\omega_2}
        \Big)
    \Big]
    &\cr
    &\hskip200pt
        \times\exp(i(k_1-k_2-k_3)z\mp i(\beta_3+\beta_2+\beta_1)z)
    &\cr
    &\hskip40pt
    +i{{\omega_3}\over{2cn_3}}
    \Big[
      p_3 A^{b\mp}_{\omega_1} A^{f\pm}_{\omega_2}
      \mp i q_3\Big(
          {{\partial A^{b\mp}_{\omega_1}}\over{\partial z}} A^{f\pm}_{\omega_2}
          +A^{b\mp}_{\omega_1} {{\partial A^{f\pm}_{\omega_2}}\over{\partial z}}
          +i(-k_1+k_2\mp (\beta_1+\beta_2)) A^{b\mp}_{\omega_1} A^{f\pm}_{\omega_2}
        \Big)
    \Big]
    &\cr
    &\hskip200pt
        \times\exp(-i(k_1-k_2+k_3)z\mp i(\beta_3+\beta_2+\beta_1)z)
    &\cr
    &\hskip40pt
    +i{{\omega_3}\over{2cn_3}}
    \Big[
      p_3 A^{b\mp}_{\omega_1} A^{b\mp}_{\omega_2}
      \mp i q_3\Big(
          {{\partial A^{b\mp}_{\omega_1}}\over{\partial z}} A^{b\mp}_{\omega_2}
          +A^{b\mp}_{\omega_1} {{\partial A^{b\mp}_{\omega_2}}\over{\partial z}}
          +i(-k_1-k_2\mp (\beta_1+\beta_2)) A^{b\mp}_{\omega_1} A^{b\mp}_{\omega_2}
        \Big)
    \Big]
    &\cr
    &\hskip200pt
        \times\exp(-i(k_1+k_2+k_3)z\mp i(\beta_3+\beta_2+\beta_1)z),
    &\cr
  }
$$
where we defined
$$
  k_j\equiv {{\omega_j n_j}\over{c}},\qquad
  \beta_j\equiv {{\omega^2_j\gamma_j}\over{2c^2}},
  \eqdef{eq:waveeq200}
$$
with, as previously, $j=1,2,3$ denoting the idler, signal and pump,
respectively.
Notice the way the phase matching in Eqs.~\eqref{eq:waveeq190} appear,
with the phase mismatch from the electric-dipolar parts occurring with
various combinations of the wave vector magnitudes $k_j$, while the gyrotropic,
non-local contributions all appear as the sum $\beta_1+\beta_2+\beta_3$.

Also notice how the coefficients governing the group velocity ($k'_j\pm a'_j$)
and group velocity dispersion ($k''_j\pm b''_j$) now always appear pair-wise
with ``$\pm$'' discriminating between the LCP/RCP components, just as for the
differential phase velocity for LCP/RCP.

In similar, the {\it backward traveling components} are obtained from
Eqs.~\eqref{eq:waveeq180}--\eqref{eq:waveeq184} by projection, starting
with the idler at angular frequency~$\omega_1$,
$$
  \eqalignno{
   &-{{\partial A^{b\mp}_{\omega_1}}\over{\partial z}}
       +(k'_1\pm a'_1) {{\partial A^{b\mp}_{\omega_1}}\over{\partial t}}
       +{{i}\over{2}}(k''_1\pm b''_1)
            {{\partial^2 A^{b\mp}_{\omega_1}}\over{\partial t^2}}
    \eqdefn{eq:waveeq210}&\eqsubdef{eq:waveeq210a}\cr
    &\hskip40pt
    =i{{\omega_1}\over{2cn_1}}
    \Big[
      p_1 A^{f\mp}_{\omega_3} A^{f\pm*}_{\omega_2}
      \pm i q_1\Big(
          {{\partial A^{f\mp}_{\omega_3}}\over{\partial z}} A^{f\pm*}_{\omega_2}
          +A^{f\mp}_{\omega_3} {{\partial A^{f\pm*}_{\omega_2}}\over{\partial z}}
          +i(k_3-k_2\pm (\beta_3+\beta_2)) A^{f\mp}_{\omega_3} A^{f\pm*}_{\omega_2}
        \Big)
    \Big]
    &\cr
    &\hskip200pt
        \times\exp(i(k_3-k_2+k_1)z\pm i(\beta_3+\beta_2+\beta_1)z)
    &\cr
    &\hskip40pt
    +i{{\omega_1}\over{2cn_1}}
    \Big[
      p_1 A^{f\mp}_{\omega_3} A^{b\mp*}_{\omega_2}
      \pm i q_1\Big(
          {{\partial A^{f\mp}_{\omega_3}}\over{\partial z}} A^{b\mp*}_{\omega_2}
          +A^{f\mp}_{\omega_3} {{\partial A^{b\mp*}_{\omega_2}}\over{\partial z}}
          +i(k_3+k_2\pm (\beta_3+\beta_2)) A^{f\mp}_{\omega_3} A^{b\mp*}_{\omega_2}
        \Big)
    \Big]
    &\cr
    &\hskip200pt
        \times\exp(i(k_3+k_2+k_1)z\pm i(\beta_3+\beta_2+\beta_1)z)
    &\cr
    &\hskip40pt
    +i{{\omega_1}\over{2cn_1}}
    \Big[
      p_1 A^{b\pm}_{\omega_3} A^{f\pm*}_{\omega_2}
      \pm i q_1\Big(
          {{\partial A^{b\pm}_{\omega_3}}\over{\partial z}} A^{f\pm*}_{\omega_2}
          +A^{b\pm}_{\omega_3} {{\partial A^{f\pm*}_{\omega_2}}\over{\partial z}}
          +i(-k_3-k_2\pm (\beta_3+\beta_2)) A^{b\pm}_{\omega_3} A^{f\pm*}_{\omega_2}
        \Big)
    \Big]
    &\cr
    &\hskip200pt
        \times\exp(-i(k_3+k_2-k_1)z\pm i(\beta_3+\beta_2+\beta_1)z)
    &\cr
    &\hskip40pt
    +i{{\omega_1}\over{2cn_1}}
    \Big[
      p_1 A^{b\pm}_{\omega_3} A^{b\mp*}_{\omega_2}
      \pm i q_1\Big(
          {{\partial A^{b\pm}_{\omega_3}}\over{\partial z}} A^{b\mp*}_{\omega_2}
          +A^{b\pm}_{\omega_3} {{\partial A^{b\mp*}_{\omega_2}}\over{\partial z}}
          +i(-k_3+k_2\pm (\beta_3+\beta_2)) A^{b\pm}_{\omega_3} A^{b\mp*}_{\omega_2}
        \Big)
    \Big]
    &\cr
    &\hskip200pt
        \times\exp(-i(k_3-k_2-k_1)z\pm i(\beta_3+\beta_2+\beta_1)z),
    &\cr
%
% --------------------------------------------------------------
%
   &-{{\partial A^{b\mp}_{\omega_2}}\over{\partial z}}
       +(k'_2\pm a'_2) {{\partial A^{b\mp}_{\omega_2}}\over{\partial t}}
       +{{i}\over{2}}(k''_2\pm b''_2)
            {{\partial^2 A^{b\mp}_{\omega_2}}\over{\partial t^2}}
    &\eqsubdef{eq:waveeq210b}\cr
    &\hskip40pt
    =i{{\omega_2}\over{2cn_2}}
    \Big[
      p_2 A^{f\mp}_{\omega_3} A^{f\pm*}_{\omega_1}
      \pm i q_2\Big(
          {{\partial A^{f\mp}_{\omega_3}}\over{\partial z}} A^{f\pm*}_{\omega_1}
          +A^{f\mp}_{\omega_3} {{\partial A^{f\pm*}_{\omega_1}}\over{\partial z}}
          +i(k_3-k_1\pm (\beta_3+\beta_1)) A^{f\mp}_{\omega_3} A^{f\pm*}_{\omega_1}
        \Big)
    \Big]
    &\cr
    &\hskip200pt
        \times\exp(i(k_3-k_1+k_2)z\pm i(\beta_3+\beta_2+\beta_1)z)
    &\cr
    &\hskip40pt
    +i{{\omega_2}\over{2cn_2}}
    \Big[
      p_2 A^{f\mp}_{\omega_3} A^{b\mp*}_{\omega_1}
      \pm i q_2\Big(
          {{\partial A^{f\mp}_{\omega_3}}\over{\partial z}} A^{b\mp*}_{\omega_1}
          +A^{f\mp}_{\omega_3} {{\partial A^{b\mp*}_{\omega_1}}\over{\partial z}}
          +i(k_3+k_1\pm (\beta_3+\beta_1)) A^{f\mp}_{\omega_3} A^{b\mp*}_{\omega_1}
        \Big)
    \Big]
    &\cr
    &\hskip200pt
        \times\exp(i(k_3+k_1+k_2)z\pm i(\beta_3+\beta_2+\beta_1)z)
    &\cr
    &\hskip40pt
    +i{{\omega_2}\over{2cn_2}}
    \Big[
      p_2 A^{b\pm}_{\omega_3} A^{f\pm*}_{\omega_1}
      \pm i q_2\Big(
          {{\partial A^{b\pm}_{\omega_3}}\over{\partial z}} A^{f\pm*}_{\omega_1}
          +A^{b\pm}_{\omega_3} {{\partial A^{f\pm*}_{\omega_1}}\over{\partial z}}
          +i(-k_3-k_1\pm (\beta_3+\beta_1)) A^{b\pm}_{\omega_3} A^{f\pm*}_{\omega_1}
        \Big)
    \Big]
    &\cr
    &\hskip200pt
        \times\exp(-i(k_3+k_1-k_2)z\pm i(\beta_3+\beta_2+\beta_1)z)
    &\cr
    &\hskip40pt
    +i{{\omega_2}\over{2cn_2}}
    \Big[
      p_2 A^{b\pm}_{\omega_3} A^{b\mp*}_{\omega_1}
      \pm i q_2\Big(
          {{\partial A^{b\pm}_{\omega_3}}\over{\partial z}} A^{b\mp*}_{\omega_1}
          +A^{b\pm}_{\omega_3} {{\partial A^{b\mp*}_{\omega_1}}\over{\partial z}}
          +i(-k_3+k_1\pm (\beta_3+\beta_1)) A^{b\pm}_{\omega_3} A^{b\mp*}_{\omega_1}
        \Big)
    \Big]
    &\cr
    &\hskip200pt
        \times\exp(-i(k_3-k_1-k_2)z\pm i(\beta_3+\beta_2+\beta_1)z),
    &\cr
%
% --------------------------------------------------------------
%
   &-{{\partial A^{b\pm}_{\omega_3}}\over{\partial z}}
       +(k'_3\mp a'_3) {{\partial A^{b\pm}_{\omega_3}}\over{\partial t}}
       +{{i}\over{2}}(k''_3\mp b''_3)
            {{\partial^2 A^{b\pm}_{\omega_3}}\over{\partial t^2}}
    &\eqsubdef{eq:waveeq210c}\cr
    &\hskip40pt
    =i{{\omega_3}\over{2cn_3}}
    \Big[
      p_3 A^{f\pm}_{\omega_1} A^{f\pm}_{\omega_2}
      \mp i q_3\Big(
          {{\partial A^{f\pm}_{\omega_1}}\over{\partial z}} A^{f\pm}_{\omega_2}
          +A^{f\pm}_{\omega_1} {{\partial A^{f\pm}_{\omega_2}}\over{\partial z}}
          +i(k_1+k_2\mp (\beta_1+\beta_2)) A^{f\pm}_{\omega_1} A^{f\pm}_{\omega_2}
        \Big)
    \Big]
    &\cr
    &\hskip200pt
        \times\exp(i(k_1+k_2+k_3)z\mp i(\beta_3+\beta_2+\beta_1)z)
    &\cr
    &\hskip40pt
    +i{{\omega_3}\over{2cn_3}}
    \Big[
      p_3 A^{f\pm}_{\omega_1} A^{b\mp}_{\omega_2}
      \mp i q_3\Big(
          {{\partial A^{f\pm}_{\omega_1}}\over{\partial z}} A^{b\mp}_{\omega_2}
          +A^{f\pm}_{\omega_1} {{\partial A^{b\mp}_{\omega_2}}\over{\partial z}}
          +i(k_1-k_2\mp (\beta_1+\beta_2)) A^{f\pm}_{\omega_1} A^{b\mp}_{\omega_2}
        \Big)
    \Big]
    &\cr
    &\hskip200pt
        \times\exp(i(k_1-k_2+k_3)z\mp i(\beta_3+\beta_2+\beta_1)z)
    &\cr
    &\hskip40pt
    +i{{\omega_3}\over{2cn_3}}
    \Big[
      p_3 A^{b\mp}_{\omega_1} A^{f\pm}_{\omega_2}
      \mp i q_3\Big(
          {{\partial A^{b\mp}_{\omega_1}}\over{\partial z}} A^{f\pm}_{\omega_2}
          +A^{b\mp}_{\omega_1} {{\partial A^{f\pm}_{\omega_2}}\over{\partial z}}
          +i(-k_1+k_2\mp (\beta_1+\beta_2)) A^{b\mp}_{\omega_1} A^{f\pm}_{\omega_2}
        \Big)
    \Big]
    &\cr
    &\hskip200pt
        \times\exp(-i(k_1-k_2-k_3)z\mp i(\beta_3+\beta_2+\beta_1)z)
    &\cr
    &\hskip40pt
    +i{{\omega_3}\over{2cn_3}}
    \Big[
      p_3 A^{b\mp}_{\omega_1} A^{b\mp}_{\omega_2}
      \mp i q_3\Big(
          {{\partial A^{b\mp}_{\omega_1}}\over{\partial z}} A^{b\mp}_{\omega_2}
          +A^{b\mp}_{\omega_1} {{\partial A^{b\mp}_{\omega_2}}\over{\partial z}}
          +i(-k_1-k_2\mp (\beta_1+\beta_2)) A^{b\mp}_{\omega_1} A^{b\mp}_{\omega_2}
        \Big)
    \Big]
    &\cr
    &\hskip200pt
        \times\exp(-i(k_1+k_2-k_3)z\mp i(\beta_3+\beta_2+\beta_1)z).
    &\cr
  }
$$
Just as previously for the forward traveling components, the phase mismatch from
the electric-dipolar parts occurring with various combinations of the wave
vector magnitudes $k_j$, while the gyrotropic, non-local contributions all
appear as the sum $\beta_1+\beta_2+\beta_3$.

\section{Co-directional optical parametric amplification}
In the classical, co-directional configuration of optical parametric
amplification, the phase matching will occur for the combination
$k_3-k_2-k_1\sim0$ of electric dipolar wave vector magnitudes.
In this particular case, for the moment just focusing on the forward traveling
components, the system described by Eqs.~\eqref{eq:waveeq190} will after
averaging over a few spatial periods of the light reduce to
$$
  \eqalignno{
    &{{\partial A^{f\pm}_{\omega_1}}\over{\partial z}}
       +(k'_1\mp a'_1) {{\partial A^{f\pm}_{\omega_1}}\over{\partial t}}
       +{{i}\over{2}}(k''_1\mp b''_1)
            {{\partial^2 A^{f\pm}_{\omega_1}}\over{\partial t^2}}
    \eqdefn{eq:waveeq220}&\eqsubdef{eq:waveeq220a}\cr
    &\hskip40pt
    =i{{\omega_1}\over{2cn_1}}
    \Big[
      p_1 A^{f\mp}_{\omega_3} A^{f\pm*}_{\omega_2}
      \pm i q_1\Big(
          {{\partial A^{f\mp}_{\omega_3}}\over{\partial z}} A^{f\pm*}_{\omega_2}
          +A^{f\mp}_{\omega_3} {{\partial A^{f\pm*}_{\omega_2}}\over{\partial z}}
          +i(k_3-k_2\pm (\beta_3+\beta_2)) A^{f\mp}_{\omega_3} A^{f\pm*}_{\omega_2}
        \Big)
    \Big]
    &\cr
    &\hskip200pt
        \times\exp(i(k_3-k_2-k_1)z\pm i(\beta_3+\beta_2+\beta_1)z),
    &\cr
%
% --------------------------------------------------------------
%
    &{{\partial A^{f\pm}_{\omega_2}}\over{\partial z}}
       +(k'_2\mp a'_2) {{\partial A^{f\pm}_{\omega_2}}\over{\partial t}}
       +{{i}\over{2}}(k''_2\mp b''_2)
            {{\partial^2 A^{f\pm}_{\omega_2}}\over{\partial t^2}}
    &\eqsubdef{eq:waveeq220b}\cr
    &\hskip40pt
    =i{{\omega_2}\over{2cn_2}}
    \Big[
      p_2 A^{f\mp}_{\omega_3} A^{f\pm*}_{\omega_1}
      \pm i q_2\Big(
          {{\partial A^{f\mp}_{\omega_3}}\over{\partial z}} A^{f\pm*}_{\omega_1}
          +A^{f\mp}_{\omega_3} {{\partial A^{f\pm*}_{\omega_1}}\over{\partial z}}
          +i(k_3-k_1\pm (\beta_3+\beta_1)) A^{f\mp}_{\omega_3} A^{f\pm*}_{\omega_1}
        \Big)
    \Big]
    &\cr
    &\hskip200pt
        \times\exp(i(k_3-k_1-k_2)z\pm i(\beta_3+\beta_2+\beta_1)z),
    &\cr
%
% --------------------------------------------------------------
%
    &{{\partial A^{f\mp}_{\omega_3}}\over{\partial z}}
       +(k'_3\pm a'_3) {{\partial A^{f\mp}_{\omega_3}}\over{\partial t}}
       +{{i}\over{2}}(k''_3\pm b''_3)
            {{\partial^2 A^{f\mp}_{\omega_3}}\over{\partial t^2}}
    &\eqsubdef{eq:waveeq220c}\cr
    &\hskip40pt
    =i{{\omega_3}\over{2cn_3}}
    \Big[
      p_3 A^{f\pm}_{\omega_1} A^{f\pm}_{\omega_2}
      \mp i q_3\Big(
          {{\partial A^{f\pm}_{\omega_1}}\over{\partial z}} A^{f\pm}_{\omega_2}
          +A^{f\pm}_{\omega_1} {{\partial A^{f\pm}_{\omega_2}}\over{\partial z}}
          +i(k_1+k_2\mp (\beta_1+\beta_2)) A^{f\pm}_{\omega_1} A^{f\pm}_{\omega_2}
        \Big)
    \Big]
    &\cr
    &\hskip200pt
        \times\exp(i(k_1+k_2-k_3)z\mp i(\beta_3+\beta_2+\beta_1)z),
    &\cr
  }
$$
where the phase mismatch is described by the terms
$$
  \eqalignno{
    \Delta\alpha&\equiv k_3-k_2-k_1
      ={{\omega_3 n_3}\over{c}}
        -{{\omega_2 n_2}\over{c}}
        -{{\omega_1 n_1}\over{c}},
    \eqdefn{eq:waveeq230}&\eqsubdef{eq:waveeq230a}\cr
    \Delta\beta&\equiv \beta_3+\beta_2+\beta_1
      ={{\omega^2_1\gamma_1}\over{2c^2}}
      +{{\omega^2_2\gamma_2}\over{2c^2}}
      +{{\omega^2_3\gamma_3}\over{2c^2}}.
    &\eqsubdef{eq:waveeq230b}\cr
  }
$$
Again, we may in Eqs.~\eqref{eq:waveeq220} safely consider the rate of change of
the field envelopes in the right-hand sides to be considerably slower than than
the order of the involved wavelengths,
$$
  \Big|{{\partial}\over{\partial z}}\Big|\ll\big|k_3-k_2\big|,\qquad
  \Big|{{\partial}\over{\partial z}}\Big|\ll\big|k_3-k_1\big|,\qquad
  \Big|{{\partial}\over{\partial z}}\Big|\ll\big|k_1+k_2\big|.
  \eqdef{eq:waveeq240}
$$
Also, we notice that the gyration coefficients $\beta_k$ all are small in
comparison to the magnitude of the wave vectors, and that we hence may
assume that
$$
  \big|\beta_3+\beta_2\big|\ll\big|k_3-k_2\big|,\qquad
  \big|\beta_3+\beta_1\big|\ll\big|k_3-k_1\big|,\qquad
  \big|\beta_1+\beta_2\big|\ll\big|k_1+k_2\big|.
  \eqdef{eq:waveeq250}
$$
Including these simplifications, we arrive at the simplified form for the
forward traveling components as
\par\boxit{
$$
  \eqalignno{
    &{{\partial A^{f\pm}_{\omega_1}}\over{\partial z}}
       +(k'_1\mp a'_1) {{\partial A^{f\pm}_{\omega_1}}\over{\partial t}}
       +{{i}\over{2}} (k''_1\mp b''_1)
            {{\partial^2 A^{f\pm}_{\omega_1}}\over{\partial t^2}}
    \cr&\hskip100pt
    =i{{\omega_1}\over{2cn_1}}
      \big(p_1 \mp q_1(k_3-k_2)\big) A^{f\mp}_{\omega_3} A^{f\pm*}_{\omega_2}
      \exp(i(\Delta\alpha\pm\Delta\beta)z),
      \eqdefn{eq:waveeq260}&\eqsubdef{eq:waveeq260a}\cr
%
% --------------------------------------------------------------
%
    &{{\partial A^{f\pm}_{\omega_2}}\over{\partial z}}
       +(k'_2\mp a'_2) {{\partial A^{f\pm}_{\omega_2}}\over{\partial t}}
       +{{i}\over{2}} (k''_2\mp b''_2)
            {{\partial^2 A^{f\pm}_{\omega_2}}\over{\partial t^2}}
    \cr&\hskip100pt
    =i{{\omega_2}\over{2cn_2}}
      \big(p_2 \mp q_2(k_3-k_1)\big) A^{f\mp}_{\omega_3} A^{f\pm*}_{\omega_1}
      \exp(i(\Delta\alpha\pm\Delta\beta)z),
    &\eqsubdef{eq:waveeq260b}\cr
%
% --------------------------------------------------------------
%
    &{{\partial A^{f\mp}_{\omega_3}}\over{\partial z}}
       +(k'_3\pm a'_3) {{\partial A^{f\mp}_{\omega_3}}\over{\partial t}}
       +{{i}\over{2}} (k''_3\pm b''_3)
            {{\partial^2 A^{f\mp}_{\omega_3}}\over{\partial t^2}}
    \cr&\hskip100pt
    =i{{\omega_3}\over{2cn_3}}
      \big(p_3 \pm q_3(k_1+k_2)\big) A^{f\pm}_{\omega_1} A^{f\pm}_{\omega_2}
      \exp(-i(\Delta\alpha\pm\Delta\beta)z).
    &\eqsubdef{eq:waveeq260c}\cr
  }
$$
}\noindent
%where we for the sake of simplicity in notation from now on will introduce the
%coefficients
%$$
%  k'^{\pm}_j \equiv k'_j\mp a_j,\qquad
%  k''^{\pm}_j \equiv k''_j\mp b_j,\qquad j=1,2,3,
%  \eqdef{eq:waveeq262}
%$$
%for the LCP/RCP-discriminating group velovity and group velovity dispersion.

\section{Pulsed co-directional optical parametric amplification in a moving
         reference frame}
With a starting point in Eqs.~\eqref{eq:waveeq260}, we will now concentrate on
the model
$$
  \eqalignno{
    &\Big({{\partial}\over{\partial z}}
       +(k'_1\mp a'_1) {{\partial}\over{\partial t}}
       +{{i}\over{2}}(k''_1\mp b''_1)
          {{\partial^2}\over{\partial t^2}}\Big) A^{f\pm}_{\omega_1}
       =i\kappa^{\pm}_1 A^{f\mp}_{\omega_3} A^{f\pm*}_{\omega_2} \exp(i\Delta k_{\pm} z),
     \eqdefn{eq:waveeq270}&\eqsubdef{eq:waveeq270a}\cr
    &\Big({{\partial}\over{\partial z}}
       +(k'_2\mp a'_2) {{\partial}\over{\partial t}}
       +{{i}\over{2}}(k''_2\mp b''_2)
          {{\partial^2}\over{\partial t^2}}\Big) A^{f\pm}_{\omega_2}
       =i\kappa^{\pm}_2 A^{f\mp}_{\omega_3} A^{f\pm*}_{\omega_1} \exp(i\Delta k_{\pm} z),
    &\eqsubdef{eq:waveeq270b}\cr
    &\Big({{\partial}\over{\partial z}}
       +(k'_3\pm a'_3) {{\partial}\over{\partial t}}
       +{{i}\over{2}}(k''_3\pm b''_3)
          {{\partial^2}\over{\partial t^2}}\Big) A^{f\mp}_{\omega_3}
       =i\kappa^{\mp}_3 A^{f\pm}_{\omega_1} A^{f\pm}_{\omega_2} \exp(-i\Delta k_{\pm} z),
    &\eqsubdef{eq:waveeq270c}\cr
  }
$$
where we for the notation of the coupling coefficients\numberedfootnote{Please
  observe that we here use $\kappa_j$ (``kappa'') for the coupling coefficients,
  and $k_j$, $k'_j$, $k''_j$, (lower-case ``k'') for the wave vector magnitude
  and its derivatives.}
adopted
$$
  \kappa^{\pm}_1
    \equiv {{\omega_1}\over{2cn_1}}\big(p_1 \mp q_1(k_3-k_2)\big),\quad
  \kappa^{\pm}_2
    \equiv {{\omega_2}\over{2cn_2}}\big(p_2 \mp q_2(k_3-k_1)\big),\quad
  \kappa^{\mp}_3
    \equiv {{\omega_3}\over{2cn_3}}\big(p_3 \pm q_3(k_1+k_2)\big),
  \eqdef{eq:waveeq280}
$$
and for the phase mismatch,
$$
  \eqalign{
  \Delta k_{\pm}&\equiv (\Delta\alpha\pm\Delta\beta)\cr
    &={{\omega_3 n_3}\over{c}}
        -{{\omega_2 n_2}\over{c}}
        -{{\omega_1 n_1}\over{c}}
      \pm\Big(
        {{\omega^2_1\gamma_1}\over{2c^2}}
          +{{\omega^2_2\gamma_2}\over{2c^2}}
          +{{\omega^2_3\gamma_3}\over{2c^2}}
      \Big).
  }
  \eqdef{eq:waveeq290}
$$
We here notice that in Eqs.~\eqref{eq:waveeq270}, the group velocities $v_i$,
$v_s$ and $v_p$ of the propagating pulses for the idler~$\omega_1$,
signal~$\omega_2$ and pump~$\omega_3$ are simply given as the reciprocals
of the respective coefficients $k'^{\pm}_j$, which from the definitions
in Eqs.~\eqref{eq:dipolwavevec10}, \eqref{eq:gyrcoeff10},
and~\eqref{eq:poldensp50} are explicitly obtained as
$$
  \eqalignno{
    v^{-1}_{i\pm}&=k'_1\mp a'_1
      =\bigg({{d k(\omega)}\over{d\omega}}
       \mp{{d\big(k(\omega)g(\omega)\big)}\over{d\omega}}\bigg)\bigg|_{\omega_1},
    \eqdefn{eq:waveeq300}&\eqsubdef{eq:waveeq300a}\cr
    v^{-1}_{s\pm}&=k'_2\mp a'_2
      =\bigg({{d k(\omega)}\over{d\omega}}
       \mp{{d\big(k(\omega)g(\omega)\big)}\over{d\omega}}\bigg)\bigg|_{\omega_2},
    &\eqsubdef{eq:waveeq300b}\cr
    v^{-1}_{p\pm}&=k'_3\mp a'_3
      =\bigg({{d k(\omega)}\over{d\omega}}
       \mp{{d\big(k(\omega)g(\omega)\big)}\over{d\omega}}\bigg)\bigg|_{\omega_3}.
    &\eqsubdef{eq:waveeq300c}\cr
  }
$$
In order to reduce the coupled system described by Eqs.~\eqref{eq:waveeq270},
we first of all transform it into a frame moving along with the group velocity
$v_3$ of the pump and normalize by the characteristic time of duration $\tau$
of the pulse, by substituting the spatial coordinate $z$ and time $t$ for the
normalized and dimensionless quantities\numberedfootnote{We here follow the
  coordinate transformation as presented in Y.~R.~Shen, {\it The Principles
  of Nonlinear Optics} (Wiley, 1984), ISBN 0-471-88998-9, page 518.}
$$
  \zeta = |k''_3|z/\tau^2,\qquad
  s=(t-k'_3 z)/\tau.
  \eqdef{eq:waveeq310}
$$
In this reference frame, keeping in mind that $k'_3=1/v_3$ is the inverse of
the group velocity of the pump, $s$ has the role of a {\it retarded time} with
reference to the motion of the pump pulse.

In terms of the moving reference system $(\zeta,s)$, the spatial and temporal
derivatives are expressed as
$$
  \eqalignno{
    {{\partial}\over{\partial z}}
      &=\underbrace{{{\partial\zeta}\over{\partial z}}}_{|k''_3|/\tau^2}
         {{\partial}\over{\partial\zeta}}
        +\underbrace{{{\partial s}\over{\partial z}}}_{-k'_3/\tau}
           {{\partial}\over{\partial s}}
      ={{|k''_3|}\over{\tau^2}}{{\partial}\over{\partial\zeta}}
        -{{k'_3}\over{\tau}}{{\partial}\over{\partial s}},
    \eqdefn{eq:waveeq320}&\eqsubdef{eq:waveeq320a}\cr
    {{\partial}\over{\partial t}}
      &=\underbrace{{{\partial\zeta}\over{\partial t}}}_{0}
         {{\partial}\over{\partial s}}
        +\underbrace{{{\partial s}\over{\partial t}}}_{1/\tau}
           {{\partial}\over{\partial s}}
      ={{1}\over{\tau}}{{\partial}\over{\partial s}}
       \quad\Rightarrow\quad{{\partial^2}\over{\partial t^2}}
                ={{1}\over{\tau^2}}{{\partial^2}\over{\partial s^2}},
    &\eqsubdef{eq:waveeq320b}\cr
  }
$$
%$$
%  {{\partial}\over{\partial z}}
%    =\underbrace{{{\partial\zeta}\over{\partial z}}}_{1}
%       {{\partial}\over{\partial\zeta}}
%      +\underbrace{{{\partial t}\over{\partial z}}}_{-v^{-1}_p}
%         {{\partial}\over{\partial t}}
%    ={{\partial}\over{\partial\zeta}}
%      -{{1}\over{v_p}}{{\partial}\over{\partial t}},
%$$
and the operators of the system described by Eqs.~\eqref{eq:waveeq270} are
hence transformed into
$$
  \eqalignno{
    &{{\partial}\over{\partial z}}
         +(k'_1\mp a'_1){{\partial}\over{\partial t}}
         +i{{(k''_1\mp b''_1)}\over{2}}{{\partial^2}\over{\partial t^2}}
    =\underbrace{
       {{|k''_3|}\over{\tau^2}}{{\partial}\over{\partial\zeta}}
        -{{k'_3}\over{\tau}}{{\partial}\over{\partial s}}
        }_{{{\partial}\over{\partial z}}}
     +(k'_1\mp a'_1)
     \underbrace{
           {{1}\over{\tau}}{{\partial}\over{\partial s}}
     }_{{{\partial}\over{\partial t}}}
     +i{{(k''_1\mp b''_1)}\over{2}}
     \underbrace{
           {{1}\over{\tau^2}}{{\partial^2}\over{\partial s^2}}
     }_{{{\partial^2}\over{\partial t^2}}},\qquad
  \eqdefn{eq:waveeq330}&\eqsubdef{eq:waveeq330a}\cr
    &{{\partial}\over{\partial z}}
         +(k'_2\mp a'_2){{\partial}\over{\partial t}}
         +i{{(k''_2\mp b''_2)}\over{2}}{{\partial^2}\over{\partial t^2}}
    =\underbrace{
       {{|k''_3|}\over{\tau^2}}{{\partial}\over{\partial\zeta}}
        -{{k'_3}\over{\tau}}{{\partial}\over{\partial s}}
        }_{{{\partial}\over{\partial z}}}
     +(k'_2\mp a'_2)
     \underbrace{
           {{1}\over{\tau}}{{\partial}\over{\partial s}}
     }_{{{\partial}\over{\partial t}}}
     +i{{(k''_2\mp b''_2)}\over{2}}
     \underbrace{
           {{1}\over{\tau^2}}{{\partial^2}\over{\partial s^2}}
     }_{{{\partial^2}\over{\partial t^2}}},\qquad
  &\eqsubdef{eq:waveeq330b}\cr
    &{{\partial}\over{\partial z}}
         +(k'_3\pm a'_3){{\partial}\over{\partial t}}
         +i{{(k''_3\pm b''_3)}\over{2}}{{\partial^2}\over{\partial t^2}}
    =\underbrace{
       {{|k''_3|}\over{\tau^2}}{{\partial}\over{\partial\zeta}}
        -{{k'_3}\over{\tau}}{{\partial}\over{\partial s}}
        }_{{{\partial}\over{\partial z}}}
     +(k'_3\pm a'_3)
     \underbrace{
           {{1}\over{\tau}}{{\partial}\over{\partial s}}
     }_{{{\partial}\over{\partial t}}}
     +i{{(k''_3\pm b''_3)}\over{2}}
     \underbrace{
           {{1}\over{\tau^2}}{{\partial^2}\over{\partial s^2}}
     }_{{{\partial^2}\over{\partial t^2}}},\qquad
  &\eqsubdef{eq:waveeq330c}\cr
  }
$$
that is to say, by collecting the normalized and dimensional temporal ($s$) and
spatial $(\zeta)$ derivatives, we reduce Eqs.~\eqref{eq:waveeq270} into their
dimensionless form\numberedfootnote{The transformation into a
  moving reference my means of a different time might feel like an odd
  and arbitrary choice. In principle, we could just as well apply a
  transformation where we make use of a moving spatial coordinate
  $$
    \zeta=z+v_p t.
  $$
  In this case, the spatial derivative will transform according to
  $$
    {{\partial}\over{\partial z}}
      =\underbrace{{{\partial\zeta}\over{\partial z}}}_{1}
         {{\partial}\over{\partial\zeta}}
        +\underbrace{{{\partial t}\over{\partial z}}}_{-v^{-1}_p}
           {{\partial}\over{\partial t}}
      ={{\partial}\over{\partial\zeta}}
        -{{1}\over{v_p}}{{\partial}\over{\partial t}},
  $$
  which would leave the left-hand side of the system exactly in the same
  form as in Eqs.~\eqref{eq:waveeq330}; however, in this case we would have
  the time entering the phase matching exponent in the right-hand side,
  something which clearly would mess upp things somewhat. Therefore, we
  stick to the transformation of time rather than spatial coordinate.}
$$
  \eqalignno{
     {{|k''_3|}\over{\tau^2}}\Big(
       {{\partial}\over{\partial\zeta}}
       +{{(k'_1\mp a'_1-k'_3)\tau}\over{|k''_3|}}{{\partial}\over{\partial s}}
       +{{i}\over{2}}
        {{(k''_1\mp b''_1)}\over{|k''_3|}}{{\partial^2}\over{\partial s^2}}
     \Big) A^{f\pm}_{\omega_1}
       &=i\kappa^{\pm}_1 A^{f\mp}_{\omega_3} A^{f\pm*}_{\omega_2}
          \exp\Big(i{{\Delta k_{\pm} \tau^2}\over{|k''_3|}}\zeta\Big),\qquad
     \eqdefn{eq:waveeq330}&\eqsubdef{eq:waveeq330a}\cr
     {{|k''_3|}\over{\tau^2}}\Big(
       {{\partial}\over{\partial\zeta}}
       +{{(k'_2\mp a'_2-k'_3)\tau}\over{|k''_3|}}{{\partial}\over{\partial s}}
       +{{i}\over{2}}
        {{(k''_2\mp b''_2)}\over{|k''_3|}}{{\partial^2}\over{\partial s^2}}
     \Big) A^{f\pm}_{\omega_2}
       &=i\kappa^{\pm}_2 A^{f\mp}_{\omega_3} A^{f\pm*}_{\omega_1}
          \exp\Big(i{{\Delta k_{\pm} \tau^2}\over{|k''_3|}}\zeta\Big),\qquad
     &\eqsubdef{eq:waveeq330b}\cr
     {{|k''_3|}\over{\tau^2}}\Big(
       {{\partial}\over{\partial\zeta}}
       \pm{{a'_3\tau}\over{|k''_3|}}{{\partial}\over{\partial s}}
       +{{i}\over{2}}
        {{(k''_3\pm b''_3)}\over{|k''_3|}}{{\partial^2}\over{\partial s^2}}
     \Big) A^{f\mp}_{\omega_3}
       &=i\kappa^{\mp}_3 A^{f\pm}_{\omega_1} A^{f\pm}_{\omega_2}
          \exp\Big(-i{{\Delta k_{\pm} \tau^2}\over{|k''_3|}}\zeta\Big).\qquad
     &\eqsubdef{eq:waveeq330c}\cr
  }
$$
The shape of the system decribed by Eqs.~\eqref{eq:waveeq330} can now be used
to normalize also the field envelopes (which we recall still have the physical
dimension of volts per meter).
Dividing both sides by the common leading coefficient $|k''_3|/\tau^2$ of the
left-hand sides, we obtain
$$
  \eqalignno{
     \Big(
       {{\partial}\over{\partial\zeta}}
       +{{(k'_1\mp a'_1-k'_3)\tau}\over{|k''_3|}}{{\partial}\over{\partial s}}
       +{{i}\over{2}}
        {{(k''_1\mp b''_1)}\over{|k''_3|}}{{\partial^2}\over{\partial s^2}}
     \Big) A^{f\pm}_{\omega_1}
       &=i{{\kappa^{\pm}_1 \tau^2}\over{|k''_3|}}
          A^{f\mp}_{\omega_3} A^{f\pm*}_{\omega_2}
          \exp\Big(i{{\Delta k_{\pm} \tau^2}\over{|k''_3|}}\zeta\Big),\qquad
     \eqdefn{eq:waveeq332}&\eqsubdef{eq:waveeq332a}\cr
     \Big(
       {{\partial}\over{\partial\zeta}}
       +{{(k'_2\mp a'_2-k'_3)\tau}\over{|k''_3|}}{{\partial}\over{\partial s}}
       +{{i}\over{2}}
        {{(k''_2\mp b''_2)}\over{|k''_3|}}{{\partial^2}\over{\partial s^2}}
     \Big) A^{f\pm}_{\omega_2}
       &=i{{\kappa^{\pm}_2 \tau^2}\over{|k''_3|}}
          A^{f\mp}_{\omega_3} A^{f\pm*}_{\omega_1}
          \exp\Big(i{{\Delta k_{\pm} \tau^2}\over{|k''_3|}}\zeta\Big),\qquad
     &\eqsubdef{eq:waveeq332b}\cr
     \Big(
       {{\partial}\over{\partial\zeta}}
       \pm{{a'_3\tau}\over{|k''_3|}}{{\partial}\over{\partial s}}
       +{{i}\over{2}}
        {{(k''_3\pm b''_3)}\over{|k''_3|}}{{\partial^2}\over{\partial s^2}}
     \Big) A^{f\mp}_{\omega_3}
       &=i{{\kappa^{\mp}_3 \tau^2}\over{|k''_3|}}
          A^{f\pm}_{\omega_1} A^{f\pm}_{\omega_2}
          \exp\Big(-i{{\Delta k_{\pm} \tau^2}\over{|k''_3|}}\zeta\Big).\qquad
     &\eqsubdef{eq:waveeq332c}\cr
  }
$$
Assuming positive and real-valued coupling coefficients $\kappa^{\pm}_k$, we
define the normalized and dimensionless field envelopes $a^{f\pm}_{\omega_j}$
for the idler, signal and pump as
$$
  A^{f\pm}_{\omega_1}\equiv
    \Big({{\kappa^{\pm}_1 \tau^2}\over{|k''_3|}}\Big)^{1/2} a^{f\pm}_{\omega_1},\qquad
  A^{f\pm}_{\omega_2}\equiv
    \Big({{\kappa^{\pm}_2 \tau^2}\over{|k''_3|}}\Big)^{1/2} a^{f\pm}_{\omega_2},\qquad
  A^{f\mp}_{\omega_3}\equiv
    \Big({{\kappa^{\mp}_3 \tau^2}\over{|k''_3|}}\Big)^{1/2} a^{f\mp}_{\omega_3},
  \eqdef{eq:waveeq340}
$$
and further define the normalized phase mismatch related to the dimensionless
and normalized spatial coordinate $\zeta$ as
$$
  \eqalign{
    \Delta\phi_{\pm}&\equiv\Delta k_{\pm} \tau^2/|k''_3|\cr
    &={{\tau^2}\over{|k''_3|}}
      \bigg(
        {{\omega_3 n_3}\over{c}}
          -{{\omega_2 n_2}\over{c}}
          -{{\omega_1 n_1}\over{c}}
        \pm
        \Big(
          {{\omega^2_1\gamma_1}\over{2c^2}}
            +{{\omega^2_2\gamma_2}\over{2c^2}}
            +{{\omega^2_3\gamma_3}\over{2c^2}}
        \Big)
      \bigg).
  }
  \eqdef{eq:waveeq350}
$$
With these definitions, the normalized and dimensionless form of the system of
coupled wave equations~\eqref{eq:waveeq332} takes the form
$$
  \eqalignno{
     \Big(
       {{\partial}\over{\partial\zeta}}
       +{{(k'_1\mp a'_1-k'_3)\tau}\over{|k''_3|}}{{\partial}\over{\partial s}}
       +{{i}\over{2}}
        {{(k''_1\mp b''_1)}\over{|k''_3|}}{{\partial^2}\over{\partial s^2}}
     \Big) a^{f\pm}_{\omega_1}
       &=i\kappa_{\pm} a^{f\mp}_{\omega_3} a^{f\pm*}_{\omega_2}
          \exp(i\Delta\phi_{\pm}\zeta),
     \eqdefn{eq:waveeq360}&\eqsubdef{eq:waveeq360a}\cr
     \Big(
       {{\partial}\over{\partial\zeta}}
       +{{(k'_2\mp a'_2-k'_3)\tau}\over{|k''_3|}}{{\partial}\over{\partial s}}
       +{{i}\over{2}}
        {{(k''_2\mp b''_2)}\over{|k''_3|}}{{\partial^2}\over{\partial s^2}}
     \Big) a^{f\pm}_{\omega_2}
       &=i\kappa_{\pm} a^{f\mp}_{\omega_3} a^{f\pm*}_{\omega_1}
          \exp(i\Delta\phi_{\pm}\zeta),
     &\eqsubdef{eq:waveeq360b}\cr
     \Big(
       {{\partial}\over{\partial\zeta}}
       \pm{{a'_3\tau}\over{|k''_3|}}{{\partial}\over{\partial s}}
       +{{i}\over{2}}
        {{(k''_3\pm b''_3)}\over{|k''_3|}}{{\partial^2}\over{\partial s^2}}
     \Big) a^{f\mp}_{\omega_3}
       &=i\kappa_{\pm} a^{f\pm}_{\omega_1} a^{f\pm}_{\omega_2}
          \exp(-i\Delta\phi_{\pm}\zeta),
     &\eqsubdef{eq:waveeq360c}\cr
  }
$$
where the common coupling coefficient is defined in terms of the coupling
coefficients of Eqs.~\eqref{eq:waveeq280} as
$$
  \kappa_{\pm}\equiv
     \Big({{\kappa^{\pm}_1\kappa^{\pm}_2\kappa^{\mp}_3 \tau^6}
       \over{|k''_3|^3}}\Big)^{1/2}.
  \eqdef{eq:waveeq370}
$$
\vfill\eject

\section{Manley--Rowe relations for the pulsed co-directional system}
In Eqs.~\eqref{eq:waveeq360}, we may directly identify that the right-hand side
has the same form as for the case of analysis of a continuous-wave system for
the three-wave mixing. This is of course a natural consequence of that we in
the nonlinear interaction consider all interaction between light and matter to
be instantaneous, without any delay or spectral broadening due to the nonlinear
components of the medium. All such broadening of dispersion is instead handled
in the left-hand side, where the linear properties of the medium such as phase
velocity, group velocity and group velocity dispersion is described in a linear
part of the optical model. In this respect, we see that we may employ a similar
method as in the continuous-wave case when analysing the energy balance between
the idler, signal and pump, commonly denoted as the {\it Manley-Rowe
relations}.\numberedfootnote{J. M. Manley and H. E. Rowe, {\it Some General
  Properties of Nonlinear Elements -- Part I: General Energy Relations},
  Proceedings of the IRE, July 1956, p. 904 -- 913;
  P.~N.~Butcher and D.~Cotter, {\it The Elements of Nonlinear Optics}
  (Cambridge University Press, 1991), ISBN 0-521-34183-3.}
The extraction of the energy balance between these modes goes as follows.

\subsection{Manley--Rowe relations in the absence of group velocity dispersion}
At a first model, we skip the second derivatives in normalized time, hence
dropping the group velocity dispersion from the problem. For this first case,
by multiplying the left- and righ-hand sides of Eqs.~\eqref{eq:waveeq360} by
the respective complex conjugates of the field envelopes and adding the
resulting complex conjugates of the left- and right-hand expressiond, we
obtain\numberedfootnote{Taking into account that
  $${{\partial}\over{\partial s}}|f(\zeta,s)|^2 =
       {{\partial f(\zeta,s)}\over{\partial s}} f^*(\zeta,s)
        +{{\partial f^*(\zeta,s)}\over{\partial s}} f(\zeta,s),\quad\hbox{etc.}
  $$
  and that $\Re[if]=-\Im[f]$.}
$$
  \eqalignno{
     \Big(
       {{\partial}\over{\partial\zeta}}
       +{{(k'_1\mp a'_1-k'_3)\tau}\over{|k''_3|}}{{\partial}\over{\partial s}}
     \Big) |a^{f\pm}_{\omega_1}|^2
       &=-2\kappa_{\pm}\Im\big[a^{f\mp}_{\omega_3} a^{f\pm*}_{\omega_2} a^{f\pm*}_{\omega_1}
          \exp(i\Delta\phi_{\pm}\zeta)\big],
     \eqdefn{eq:manrow10}&\eqsubdef{eq:manrow10a}\cr
     \Big(
       {{\partial}\over{\partial\zeta}}
       +{{(k'_2\mp a'_2-k'_3)\tau}\over{|k''_3|}}{{\partial}\over{\partial s}}
     \Big) |a^{f\pm}_{\omega_2}|^2
       &=-2\kappa_{\pm}\Im\big[a^{f\mp}_{\omega_3} a^{f\pm*}_{\omega_2} a^{f\pm*}_{\omega_1}
          \exp(i\Delta\phi_{\pm}\zeta)\big],
     &\eqsubdef{eq:manrow10b}\cr
     \Big(
       {{\partial}\over{\partial\zeta}}
       \pm{{a'_3\tau}\over{|k''_3|}}{{\partial}\over{\partial s}}
     \Big) |a^{f\mp}_{\omega_3}|^2
       &=-2\kappa_{\pm}\Im\big[a^{f\pm}_{\omega_1} a^{f\pm}_{\omega_2} a^{f\mp*}_{\omega_3}
          \exp(-i\Delta\phi_{\pm}\zeta)\big],
     &\cr
       &=2\kappa_{\pm}\Im\big[ a^{f\mp}_{\omega_3} a^{f\pm*}_{\omega_2} a^{f\pm*}_{\omega_1}
          \exp(i\Delta\phi_{\pm}\zeta)\big].
     &\eqsubdef{eq:manrow10c}\cr
  }
$$
By direct inspection, the idler, signal and pump intensities from the system
of Eqs.~\eqref{eq:manrow10} hence obey the energy conservation rules, or
equivalently Manley--Rowe relations,
\par\boxit{
$$
  \eqalign{
  \Big(
    {{\partial}\over{\partial\zeta}}
      +{{(k'_1\mp a'_1-k'_3)\tau}\over{|k''_3|}}{{\partial}\over{\partial s}}
  \Big) |a^{f\pm}_{\omega_1}|^2
 =\Big(
    {{\partial}\over{\partial\zeta}}
      +{{(k'_2\mp a'_2-k'_3)\tau}\over{|k''_3|}}{{\partial}\over{\partial s}}
  \Big) |a^{f\pm}_{\omega_2}|^2
 =-\Big(
    {{\partial}\over{\partial\zeta}}
      \pm{{a'_3\tau}\over{|k''_3|}}{{\partial}\over{\partial s}}
   \Big) |a^{f\mp}_{\omega_3}|^2
   }
  \eqdef{eq:manrow20}
$$
}\noindent
which due to the inclusion of the relative group velocities in the coefficients
of $\partial/\partial s$ are interpreted as the rate of energy transfer between
the modes during propagation along the $\zeta$-axis, from the presence of
$\partial/\partial\zeta$ terms, but also as the change in energy transfer rate
due to any mismatch in group velocity between the pulses, from the presence of
coefficients to the $\partial/\partial s$ terms.

Notice that the Manley--Rowe relations as per above include any mismatch in
group velocities between the idler-relative-pump and signal-relative-pump, as
well as the differential contributions to the group velocities for LCP/RCP via
the $\mp a'_j$ terms. This is overall a signature of the purely reactive form of
the medium, creating a virtual flow from a field at one frequency via the medium
over to the other interacting frequencies in the current three-way mixing.

\subsection{Manley--Rowe relations in the presence of group velocity dispersion}
As for the Manley--Rowe relations in the absence of group velocity dispersion,
this was a pretty straightforward and easy derivation. Let us now therefore
turn our attention to the more general case in which we also include group
velocity dispersion, that is to say including non-zero coefficients $k''_j$
and $b''_j$ from Eqs.~\eqref{eq:waveeq360c}.
In this case, the same multiplication by complex-conjugated field envelopes
followed by an identical addition of the resulting complex conjugates leads
to the form
$$
  \eqalignno{
     \Big(
       {{\partial}\over{\partial\zeta}}
       +{{(k'_1\mp a'_1-k'_3)\tau}\over{|k''_3|}}{{\partial}\over{\partial s}}
     \Big) |a^{f\pm}_{\omega_1}|^2
       +{{i}\over{2}}&
        {{(k''_1\mp b''_1)}\over{|k''_3|}}
     \Big(
          {{\partial^2 a^{f\pm}_{\omega_1}}\over{\partial s^2}} a^{f\pm*}_{\omega_1}
          -a^{f\pm}_{\omega_1} {{\partial^2 a^{f\pm*}_{\omega_1}}\over{\partial s^2}}
     \Big)\cr
       &=-2\kappa_{\pm}\Im\big[
            a^{f\mp}_{\omega_3} a^{f\pm*}_{\omega_2} a^{f\pm*}_{\omega_1}
          \exp(i\Delta\phi_{\pm}\zeta)\big],
     \eqdefn{eq:manrow30}&\eqsubdef{eq:manrow30a}\cr
     \Big(
       {{\partial}\over{\partial\zeta}}
       +{{(k'_2\mp a'_2-k'_3)\tau}\over{|k''_3|}}{{\partial}\over{\partial s}}
     \Big) |a^{f\pm}_{\omega_2}|^2
       +{{i}\over{2}}&
        {{(k''_2\mp b''_2)}\over{|k''_3|}}
     \Big(
          {{\partial^2 a^{f\pm}_{\omega_2}}\over{\partial s^2}} a^{f\pm*}_{\omega_2}
          -a^{f\pm}_{\omega_2} {{\partial^2 a^{f\pm*}_{\omega_2}}\over{\partial s^2}}
     \Big)\cr
       &=-2\kappa_{\pm}\Im\big[
            a^{f\mp}_{\omega_3} a^{f\pm*}_{\omega_2} a^{f\pm*}_{\omega_1}
          \exp(i\Delta\phi_{\pm}\zeta)\big],
     &\eqsubdef{eq:manrow30b}\cr
     \Big(
       {{\partial}\over{\partial\zeta}}
       \pm{{a'_3\tau}\over{|k''_3|}}{{\partial}\over{\partial s}}
     \Big) a^{f\mp}_{\omega_3}
       +{{i}\over{2}}&
        {{(k''_3\pm b''_3)}\over{|k''_3|}}
     \Big(
       {{\partial^2 a^{f\mp}_{\omega_3}}\over{\partial s^2}} a^{f\mp*}_{\omega_3}
       -a^{f\mp}_{\omega_3} {{\partial^2 a^{f\mp*}_{\omega_3}}\over{\partial s^2}}
     \Big)\cr
       &=2\kappa_{\pm}\Im\big[ a^{f\mp}_{\omega_3} a^{f\pm*}_{\omega_2} a^{f\pm*}_{\omega_1}
          \exp(i\Delta\phi_{\pm}\zeta)\big],
     &\eqsubdef{eq:manrow30c}\cr
  }
$$
which clearly is not as easy to interprete as an energy balance as the system of
Eqs.~\eqref{eq:manrow10}. Here, the mixed second derivatives $\partial^2/
\partial s^2$ with an intermediate minus alters the situation compared to the
one in absence of group velocity dispersion. As they currently stand in
Eqs.~\eqref{eq:manrow30}, these terms do not allow for a direct interpretation
in terms of derivatives of the ``intensities'' $|a^{f\pm}_{\omega_j}|^2$, $j=1,2,3$.
However, as we via the dispersion still have not altered any premises for the
medium, which still is assumed to be purely reactive in its interaction with
the light, we may still expect some kind of symmetry or energy conservation
rule to apply.
\vskip20pt
\centerline{[ - - - TO BE CONTINUED - - - ]}
\vskip20pt

\section{Final normalization of the system for pulsed co-directional
         optical parametric amplification}
It should here be emphasized that the form of Eqs.~\eqref{eq:waveeq360} with a
{\it common} coupling coefficient opens up for a final normalization which
enables us to get rid of this coupling coefficient altogether, for an even
cleaner algebraic shape. We can see how this is done by multiplying both sides
of the equations in the system by the common coupling coefficient $\kappa_{\pm}$,
followed by defining the normalized field envelopes, which now are scaled be
the common coupling coefficient, as
$$
  \tilde{a}^{f\pm}_{\omega_1}(\zeta,s)=\kappa_{\pm} a^{f\pm}_{\omega_1}(\zeta,s),\qquad
  \tilde{a}^{f\pm}_{\omega_2}(\zeta,s)=\kappa_{\pm} a^{f\pm}_{\omega_2}(\zeta,s),\qquad
  \tilde{a}^{f\mp}_{\omega_3}(\zeta,s)=\kappa_{\pm} a^{f\mp}_{\omega_3}(\zeta,s),
  \eqdef{eq:waveeq370}
$$
leading to Eqs.~\eqref{eq:waveeq360} taking the final normalized and
dimensionless form\numberedfootnote{Notice that the linear part of the
  normalized form for the pump exactly follows the resulting Eq.~(26.20)
  in Y.~R.~Shen, {\it The Principles of Nonlinear Optics} (Wiley, 1984),
  ISBN 0-471-88998-9, page 518:
  $$
    -i{{\partial a}\over{\partial\xi}} = {{1}\over{2}}
      \Big({{\partial v_g}\over{\partial\omega}}\Big/
          \Big|{{\partial v_g}\over{\partial\omega}}\Big|\Big)
      {{\partial^2 a}\over{\partial s^2}} + |a|^2 a.
  $$
  The only difference is that here, we are
  concerned with a system of equations for OPA, while in Shen, the single
  equation concerns propagation in an optical Kerr medium, hence the
  ``$|a|^2 a$'' source term in the right-hand side. Still, the approach
  of normalization follows analogously.}
\par\boxit{
$$
  \eqalignno{
     \Big(
       {{\partial}\over{\partial\zeta}}
       +{{(k'_1\mp a'_1-k'_3)\tau}\over{|k''_3|}}{{\partial}\over{\partial s}}
       +{{i}\over{2}}
        {{(k''_1\mp b''_1)}\over{|k''_3|}}{{\partial^2}\over{\partial s^2}}
     \Big) \tilde{a}^{f\pm}_{\omega_1}
       &=i\tilde{a}^{f\mp}_{\omega_3} \tilde{a}^{f\pm*}_{\omega_2}
          \exp(i\Delta\phi_{\pm}\zeta),
     \eqdefn{eq:waveeq370}&\eqsubdef{eq:waveeq370a}\cr
     \Big(
       {{\partial}\over{\partial\zeta}}
       +{{(k'_2\mp a'_2-k'_3)\tau}\over{|k''_3|}}{{\partial}\over{\partial s}}
       +{{i}\over{2}}
        {{(k''_2\mp b''_2)}\over{|k''_3|}}{{\partial^2}\over{\partial s^2}}
     \Big) \tilde{a}^{f\pm}_{\omega_2}
       &=i\tilde{a}^{f\mp}_{\omega_3} \tilde{a}^{f\pm*}_{\omega_1}
          \exp(i\Delta\phi_{\pm}\zeta),
     &\eqsubdef{eq:waveeq370b}\cr
     \Big(
       {{\partial}\over{\partial\zeta}}
       \pm{{a'_3\tau}\over{|k''_3|}}{{\partial}\over{\partial s}}
       +{{i}\over{2}}
        {{(k''_3\pm b''_3)}\over{|k''_3|}}{{\partial^2}\over{\partial s^2}}
     \Big) \tilde{a}^{f\mp}_{\omega_3}
       &=i\tilde{a}^{f\pm}_{\omega_1} \tilde{a}^{f\pm}_{\omega_2}
          \exp(-i\Delta\phi_{\pm}\zeta),
     &\eqsubdef{eq:waveeq370c}\cr
  }
$$
}\noindent
where, to recapitulate the normalization of spatial and temporal coordinates
from Eq.~\eqref{eq:waveeq310}, $\zeta$ is the normalized and dimensionless
spatial coordinate following the system along with the group velocity
$v_p=1/k'_3$ of the pump and scaled against the characteristic pulse width
$\tau/v_p$ and magnitude of the group velocity dispersion $|k''_3|$ of the
pump, while $s$ is the corresponding normalized and dimensionless time, also
scaled against the characteristic pulse duration $\tau$ of the pump.

\subsection{The dispersion parameters}
From Eq.~\eqref{eq:waveeq300}, and the definition of the coefficients $k'_k$
and $k''_k$,
$$
  k'_k \equiv k'(\omega_k)
       \equiv {{d}\over{d\omega}}
               \bigg({{\omega n(\omega)}\over{c}}\bigg)\bigg|_{\omega_k},\qquad
  k''_k \equiv k''(\omega_k)
        \equiv {{d^2}\over{d\omega^2}}
               \bigg({{\omega n(\omega)}\over{c}}\bigg)\bigg|_{\omega_k},
$$
for $k=1,2,3$ for the idler, signal and pump, respectively, we from the wave
equation~\eqref{eq:waveeq360} have that the coefficient
$(k'_1-k'_3)\tau/|k''_3|$ can be interpreted as the difference in reciprocal
group velocities between the idler ($v_i=1/k'_1$) and pump ($v_p=1/k'_3$)
pulses, normalized against the group velocity dispersion $|k''_3|$ of the pump,
$$
  {{(k'_1-k'_3)\tau}\over{|k''_3|}}=
  {{\tau}\over{|k''_3|}}\Big({{1}\over{v_i}}-{{1}\over{v_p}}\Big).
$$
Notice that the sign of the group-velocity dispersion $k''_k$ can be either
positive or negative. In the case of positive group-velocity dispersion,
$d^2k/d\omega^2>0$, a long-wavelength, or equivalently low-frequency, pulse
travels faster than a short-wavelength, or high-frequency, pulse.

\subsection{Resemblance to the nonlinear Schr\"odinger equation}
The system~\eqref{eq:waveeq360}, in particular the final equation for the
pump wave, slightly rearranged as
$$
  i{{\partial\tilde{a}^{f\mp}_{\omega_3}}\over{\partial\zeta}}
    ={{1}\over{2}}{{k''_3}\over{|k''_3|}}
     {{\partial^2\tilde{a}^{f\mp}_{\omega_3}}\over{\partial s^2}}
    -\tilde{a}^{f\pm}_{\omega_1} \tilde{a}^{f\pm}_{\omega_2}
       \exp(-i\Delta\phi_{\pm}\zeta),
$$
clearly resembles the one-dimensional nonlinear Schr\"odinger equation of the
classic form
$$
  i\hbar{{\partial\psi(z,t)}\over{\partial t}}
    =\Big[-{{\hbar^2}\over{2m}}{{\partial^2}\over{\partial z^2}}
       +V(z,t)\Big]\psi(z,t)
$$
for the wave function $\psi(z,t)$; however, notice that in the system of PDEs
for the OPA process in Eqs.~\eqref{eq:waveeq360}, the single first-derivative
in the left-hand side is for the normalized {\it spatial coordinate} $\zeta$
(rather than for the time $t$ in the Schr\"odinger equation), while the higher
derivatives are for $s$, being the normalized ${\it time}$ (rather than for
the spatial coordinate $z$ in the Schr\"odinger equation).

In this respect, we may say that we have switched order of the variables
between dispersion and spatial interaction; also notice that in this analogy,
swapping space and time, our present potential would correspond to a potential
which only varies in {\it time}, while it for the OPA process, as evident from
Eqs.~\eqref{eq:waveeq360}, instead only has a spatial dependence in terms of
the nonlinear interaction and the phase mismatch factors
$\exp(i\Delta\phi_{\pm}\zeta)$ and $\exp(-i\Delta\phi_{\pm}\zeta)$.

\section{Numerical simulation of the pulsed chiral OPA process}
We will now proceed with solving the normalized system of nonlinear partial
differential equations~\eqref{eq:waveeq360} by means of the {\it Split-Step
Fourier method},\numberedfootnote{See, for example, Govind P.~Agrawal,
  {\it Nonlinear Fiber Optics} (Academic Press, 1995), ISBN 0-12-045142-5,
  Section 2.4.1 - {\it Split-Step Fourier Method}, pages 50--54.}%
\numberedfootnote{Ralf Deiterding, Roland Glowinski, Hilde Oliver
  and Stephen Poole, {\it A Reliable Split-Step Fourier Method for
  the Propagation Equation of Ultra-Fast Pulses in Single-Mode Optical
  Fibers}, J. Lightwave Technology {\bf 31}, 2008--2017 (2013).}

\subsection{Formulation of the system of PDEs by operator formalism}
In order to solve the normalized system of nonlinear partial differential
equations described by the system~\eqref{eq:waveeq360} by means of the
{\it Split-Step Fourier method}, we rewrite the system as
$$
  {{\partial{\bf a}(\zeta,s)}\over{\partial\zeta}}
    = (\hat{\fam\bbb D}+\hat{\fam\bbb N}){\bf a}(\zeta,s),
  \eqdef{eq:numsol10}
$$
where the $3\times1$ column vector ${\bf a}(\zeta,s)$ is defined as
$$
  {\bf a}(\zeta,s)\equiv
  \pmatrix{
    \tilde{a}^{f\pm}_{\omega_1}(\zeta,s)\cr
    \tilde{a}^{f\pm}_{\omega_2}(\zeta,s)\cr 
    \tilde{a}^{f\mp}_{\omega_3}(\zeta,s)\cr 
  }
  \eqdef{eq:numsol20}
$$
and where the operators $\hat{\fam\bbb D}$ and $\hat{\fam\bbb N}$ are
defined to handle the dispersion and nonlinear interactions, respectively.
Here, the arguments $(\zeta,s)$ act as the normalized version of the regular
spatial coordinate and time $(z,s)$ via Eqs.~\eqref{eq:waveeq310}, with the
principal difference that the normlized ``time'' $s$ is expressed in a retarded
time frame, effectively following the mean group velocity of the pump pulse.

Starting with the diagonal operator $\hat{\fam\bbb D}$ for the dispersion,
this is from Eqs.~\eqref{eq:waveeq370} defined in terms of the normalized
time derivatives $\partial/\partial s$ and $\partial^2/\partial s^2$ as
$$
  \hat{\fam\bbb D}\equiv
  -\pmatrix{
    \displaystyle
     D'^{\pm}_{11}{{\partial}\over{\partial s}}
       +D''^{\pm}_{11}{{\partial^2}\over{\partial s^2}}
      & 0 & 0 \cr
    0 &
    \displaystyle
     D'^{\pm}_{22}{{\partial}\over{\partial s}}
       +D''^{\pm}_{22}{{\partial^2}\over{\partial s^2}}
      & 0 \cr
    0 & 0 & 
    \displaystyle
     D'^{\pm}_{33}{{\partial}\over{\partial s}}
       +D''^{\pm}_{33}{{\partial^2}\over{\partial s^2}}\cr
  },
  \eqdef{eq:numsol30}
$$
where the coefficients of the first and second derivatives $\partial/\partial s$
and $\partial^2/\partial s^2$ in the normalized ``time'' $s$ along the diagonal
elements are
$$
  \eqalignno{
    D'^{\pm}_{11}&={{(k'_1\mp a'_1-k'_3)\tau}\over{|k''_3|}},\qquad
    D''^{\pm}_{11}=i{{(k''_1\mp b''_1)}\over{2|k''_3|}},
    \eqdefn{eq:numsol32}&\eqsubdef{eq:numsol32a}\cr
    D'^{\pm}_{22}&={{(k'_2\mp a'_2-k'_3)\tau}\over{|k''_3|}},\qquad
    D''^{\pm}_{22}=i{{(k''_2\mp b''_2)}\over{2|k''_3|}},
    &\eqsubdef{eq:numsol32b}\cr
    D'^{\pm}_{33}&=\pm{{a'_3\tau}\over{|k''_3|}},\hskip64pt
    D''^{\pm}_{33}=i{{(k''_3\pm b''_3)}\over{2|k''_3|}},
    &\eqsubdef{eq:numsol32c}\cr
  }
$$
where we may recall that the simplified form of the third element, for the
pump, stems from the fact that we chose the group velocity of the pump as
reference in the normalized time $s$ from Eq.~\eqref{eq:waveeq310}.
Notice that the linear $\hat{\fam\bbb D}$ operator only involves the
normalized {\it time}; in just a moment, we will make use of that the time
derivatives which occur in $\hat{\fam\bbb D}$ can be expressed as a
polynomial in the angular frequency in Fourier domain.
It should here be stressed that the $\hat{\fam\bbb D}$ operator has no
explicit spatial dependence, and of course (due to its linearity) no
dependence of the involved fields as such.

Meanwhile, the nonlinear operator $\hat{\fam\bbb N}$ is from
Eqs.~\eqref{eq:waveeq370} defined as
$$
  \hat{\fam\bbb N}\equiv
  \pmatrix{
    \displaystyle
      0 & 0 & i  \tilde{a}^{f\pm*}_{\omega_2}\exp(i\Delta\phi_{\pm}\zeta)\cr
    \displaystyle
      0 & 0 & i\tilde{a}^{f\pm*}_{\omega_1}\exp(i\Delta\phi_{\pm}\zeta)\cr
    \displaystyle
      i\tilde{a}^{f\pm}_{\omega_2}\exp(-i\Delta\phi_{\pm}\zeta) & 0 & 0 \cr
  },
  \eqdef{eq:numsol40}
$$
where we for the sake of simplifying the notation omitted the explicit
arguments of the normalized fields
$\tilde{a}^{f\pm}_{\omega_k}=\tilde{a}^{f\pm}_{\omega_k}(\zeta,s)$.
The operator $\hat{\fam\bbb N}$ handles the nonlinear mixing between the
fields, as well as the explicit spatial dependence on the phase mismatch
via the factors $\exp(i\Delta\phi_{\pm}\zeta)$ (for the idler and signal
fields) and $\exp(-i\Delta\phi_{\pm}\zeta)$ (for the pump field).

Notice that due to the mixed-term nature of the OPA process, the definition of
$\hat{\fam\bbb N}$ is not unique; for example, we could equally well have
defined the last row as instead having
$i\tilde{a}^{f\pm}_{\omega_1}\exp(-i\Delta\phi_{\pm}\zeta)$ in the second column,
operating on the signal field $\tilde{a}^{f\pm}_{\omega_2}$, that is to say instead
using
$$
  \hat{\fam\bbb N}\equiv
  \pmatrix{
    \displaystyle
      0 & 0 & i  \tilde{a}^{f\pm*}_{\omega_2}\exp(i\Delta\phi_{\pm}\zeta)\cr
    \displaystyle
      0 & 0 & i\tilde{a}^{f\pm*}_{\omega_1}\exp(i\Delta\phi_{\pm}\zeta)\cr
      0 &
    \displaystyle
      i\tilde{a}^{f\pm}_{\omega_1}\exp(-i\Delta\phi_{\pm}\zeta) & 0 \cr
  }.
$$
This alternative would when multiplied by the field vector ${\bf a}$ of
Eq.~\eqref{eq:numsol20} produce {\it exactly the same mixing term} involving
the product $\tilde{a}^{f\pm}_{\omega_1}\tilde{a}^{f\pm}_{\omega_2}$ acting as a
source in the equation for the pump. However, just for the sake of sticking
to one definition, we in the present analysis choose $\hat{\fam\bbb N}$ as
given by Eq.~\eqref{eq:numsol40}.

\subsection{A scalar analogy to the operator formulation in spatial domain}
As a preable to the Split-Step Fourier Method, the form of
Eqs.~\eqref{eq:numsol10} suggests that we could start off with a pedagogic
check of the scalar behaviour. Thus, assume that $D$ and $N$ are (possibly
complex-valued) constants and that $a(z)$ is a scalar function obeying the
equation with the close-to-trivial solution
$$
  {{da(z)}\over{dz}}=(D+N)a(z)
  \qquad\Leftrightarrow\qquad
  a(z)=C\exp\big((D+N)z\big),
  \eqdef{eq:analogy10}
$$
where $C$ is a constant of integration. In other words, having obtained
a solution at any spatial coordinate $z$, the solution of
Eq.~\eqref{eq:analogy10} at $z+\Delta z$ is then given as
$$
  \eqalign{
    a(z+\Delta z)&=C\exp\big((D+N)(z+\Delta z)\big)\cr
      &=\underbrace{C\exp\big((D+N)z\big)}_{\displaystyle a(z)}
        \exp\big((D+N)\Delta z\big)\cr
      &=\exp(D\Delta z)\exp(N\Delta z)a(z).\cr
  }
  \eqdef{eq:analogy20}
$$
Thus, the evolution of the solution given by Eq.~\eqref{eq:analogy20} from
$z$ to $z+\Delta z$ can be described by a simple multiplication by
$\exp\big(D\Delta z\big)\exp\big(N\Delta z\big)$, and it is the purpose of
the Split-Step Fourier Method to perform such small steps of simulation,
though considering that $\hat{\fam\bbb D}$ and $\hat{\fam\bbb N}$ in
Eq.~\eqref{eq:numsol10} are matrix operators, and that $\hat{\fam\bbb N}$
in particular also has an explicit spatial and nonlinear dependence.

\subsection{A scalar analogy to the operator formulation in spectral domain}
Let us now analyze a variant of Eq.~\eqref{eq:analogy10} in which the envelope
$a=a(z,t)$ is dependent on the spatial coordinate $z$ and time $t$, and in
which the operator managing the dispersion is a polynomial of time derivatives,
$$
  {{\partial a(z,t)}\over{\partial z}}
    =\Big[D\Big({{\partial}\over{\partial t}}\Big)+N\Big]a(z,t),
  \eqdef{eq:analogy30}
$$
where $D$ is the polynomial
$$
  D(X) = \sum^{\infty}_{k=0}\beta_k X^2.
  \eqdef{eq:analogy40}
$$
If we as a first step, just as in the previous section, assume that we may treat
the right-hand side as constant for a sufficiently small step $\Delta z$, we
could then, again in analogy to Eq.~\eqref{eq:analogy20} in the previous
section, express the solution in terms of the (perhaps somewhat na\"ive)
propagator
$$
  a(z+\Delta z,t)=C\exp\big((D+N)(z+\Delta z)\big)
     =\exp\Big[\Delta z D\Big({{\partial}\over{\partial t}}\Big)\Big]
      \exp(N\Delta z)a(z,t).
  \eqdef{eq:analogy50}
$$
The question is then how we should evaluate the exponential involving the time
derivatives. Clearly, from a numerical perspective we should avoid splitting
the envelopes in small time steps and assume that we may go ahead straight away
with just computing differences and divide by the time steps. Such an approach
is extremely sensitive to noise and numerical cancellation, and will lead
straight to numerical inconsistencies and instabilities.

Another approach would be to observe that the exponential of a polynomial in
derivatives in time, operating on an arbitrary time-dependent function $f(t)$,
may be handled in angular frequency domain (Fourier domain) as
$$
  \eqalign{
    \exp\Big[\Delta z D\Big({{\partial}\over{\partial t}}\Big)\Big]f(t)
      &=\fourier^{-1}\Big\{
          \fourier\Big[
            \exp\Big[\Delta z D\Big({{\partial}\over{\partial t}}\Big)\Big]f(t)
          \Big]
        \Big\}\cr
      &=\fourier^{-1}\Big\{
          \exp[\Delta z D(-i\omega)]\fourier[f(t)]
        \Big\}\cr
      &=\fourier^{-1}\Big\{
          \exp[\Delta z D(-i\omega)]f(\omega)
        \Big\}.\cr
  }
  \eqdef{eq:analogy60}
$$
Thus, we may from the ``half-solution'' expressed by Eq.~\eqref{eq:analogy50}
conclude that we may proceed in a split-step fashion as
$$
  a(z+\Delta z,t)
    =\underbrace{\fourier^{-1}\Big\{\exp[\Delta z D(-i\omega)]
        \fourier[\underbrace{\exp(N\Delta z)a(z,t)}_{\hbox{Step 1}}]
     \Big\}}_{\hbox{Step 2}},
  \eqdef{eq:analogy70}
$$
that is to say, in the first step applying the operator $N$ under the assumption
of this being constant (in the case of a nonlinear $N$ implying that the
nonlinear contribution is treated as piece-wise constant over the small step
$\Delta z$), followed by the second step in which we apply the time-dependent
dispersive operator $D$ and interpret the involved time derivatives in Fourier
domain.

In fact, these two steps form the very basis of the {\it Split-Step Fourier
Method} as will now be formalised.

\subsection{Formulation of step-wise propagation by means of the Split-Step
            Fourier Method}
Having formulated the system of coupled and nonlinear PDEs in terms of the
operator formalism described by Eqs.~\eqref{eq:numsol10}--\eqref{eq:numsol40},
we will now proceed with formulating the method of solution by means of the
Split-Step Fourier Method. As a starter, we will describe the original
formulation of the Split-Step Fourier Method, however directly afterwards we
will describe a more refined version with an only marginally more complex
algorithm.

{\bf [Step 0 -- Initialization]} Before entering the first step in the
Split-Step Fourier Method, we initialize the involved field envelopes
contained by the vector ${\bf a}={\bf a}(\zeta,s)$ at $\zeta=0$ as vectors
in $N$ samples of discretized normalized time $s_k$, $k=0,1,\ldots,N-1$.
As for the notation of a system of $M$ fields contained by the ${\bf a}$
column vector, we use each row to denote the $m$:th field envelope,
$$
  {\bf a}(\zeta,s)
    =\pmatrix{
       a_0(\zeta,s)\cr
       a_1(\zeta,s)\cr
       \vdots\cr
       a_{M-1}(\zeta,s)\cr
    },
  \eqdef{eq:numsol0-10}
$$
that is to say, ${\bf a}\in{\fam\bbb R}^{[M\times1]}$. However, when discretizing
this along the normalized time axis, we from a practical point allocate the
field vector at normalized and dimensionless spatial coordinate $\zeta$ as a
two dimensional array with elements $a_{m,k}$, with as previously the $M$ field
envelopes along the row and the discrete samples in normalized and dimensionless
time $s_k$ along the columns,
$$
  \eqalign{
  {\bf a}(\zeta)
    &=\pmatrix{
        a_0(\zeta,s_0)&a_0(\zeta,s_1)&a_0(\zeta,s_2)&\cdots&a_0(\zeta,s_{N-1})\cr
        a_1(\zeta,s_0)&a_1(\zeta,s_1)&a_1(\zeta,s_2)&\cdots&a_1(\zeta,s_{N-1})\cr
        a_2(\zeta,s_0)&a_2(\zeta,s_1)&a_2(\zeta,s_2)&\cdots&a_2(\zeta,s_{N-1})\cr
        \vdots & \vdots & \vdots & \ddots & \vdots\cr
        a_{M-1}(\zeta,s_0)&a_{M-1}(\zeta,s_1)&a_{M-1}(\zeta,s_2)&\cdots&a_{M-1}(\zeta,s_{N-1})\cr
      }\cr
    &=\pmatrix{
        a_{0,0}&a_{0,1}&a_{0,2}&\cdots&a_{0,N-1}\cr
        a_{1,0}&a_{1,1}&a_{1,2}&\cdots&a_{1,N-1}\cr
        a_{2,0}&a_{2,1}&a_{2,2}&\cdots&a_{2,N-1}\cr
        \vdots & \vdots & \vdots & \ddots & \vdots\cr
        a_{M-1,0}&a_{M-1,1}&a_{M-1,2}&\cdots&a_{M-1,N-1}\cr
      },\cr
    }
  \eqdef{eq:numsol0-20}
$$
that is to say, ${\bf a}\in{\fam\bbb R}^{[M\times N]}$. With reference to the
Fourier transform appearing in the interpretation of dispersion in the scalar
example of Eq.~\eqref{eq:analogy70}, we will hence in the following apply the
Fourier transform row-wise of the array ${\bf a}$, separately for each field
envelope $m=0,1,\ldots,M-1$.

Typically, for a pulse having its peak occurring at normalized time $s=0$, we
would need a set of time samples properly covering the pulse before (negative
time) and after (positive time) this peak. Hence, if we assume a symmetric
coverage around the normalized time $s=0$ of its launch, we may for an even
number of samples, say $N=2^p$ for some positive integer $p$ (just to ensure
an efficient fast Fourier transform in what is to come), and with a uniform
time step $\Delta s$
choose\numberedfootnote{Sanity check: $s_0=-(N-1)\Delta s/2$ and
  $s_{N-1}=(N-1)\Delta s/2$, symmetrically placed around the centre
  normalized and dimensionless time $s=0$ as expected.}
$$
  s_k=\bigg(k-{{(N-1)}\over{2}}\bigg)\Delta s,\quad
  k=0,1,\ldots,N-1,
  \eqdef{eq:numsol0-30}
$$
spanning the interval $s_{N-1}-s_0=(N-1)\Delta s$ from start to stop.
At these discrete samples of normalized time, we initialize the normalized
field envelopes as described by Eq.~\eqref{eq:numsol0-20} and take this as
our starting point for $\zeta=0$.

For a set of $N=2^p$ samples of normalized and dimensionless time, we will have
an associated set of $N$ samples of the corresponding normalized angular
frequency, which we for simplicity will denote by $\omega_k$. (Despite that we
usually reserve $\omega$ as symbol for the ``real'' angular frequency with
physical unit of rad/s.)
Used in context of a Fast Fourier Transform (FFT) over $N$ samples $s_k$
in normalized and dimensionless time, the associated angular frequencies
are\numberedfootnote{The ordering of these frequencies depends on the
  programming language of choice. An array of frequencies from a given
  $\Delta s$ and number of samples $N$ can with Python's scientific
  library {\tt scipy.fftpack} be obtained by {\tt fftshift(fftfreq(N,d=ds))},
  with ${\tt N}\equiv N$ and ${\tt ds}\equiv \Delta s$.}
$$
  \omega_k = 2\pi k f_{\rm s}/N,\qquad k=0,1,\ldots,N,
$$
where $f_{\rm s}=1/\Delta s$ is the (normalized and dimensionless) temporal
sample rate of the signal. As we in the dispersive step of propagation will
make use of the Fourier-transformed properties of derivatives, replacing the
derivatives $\partial/\partial s$ in normalized time by multiplication by
$-i\omega$, we will in the initialization step pre-compute these angular
frequencies, as well as all powers $w^n_k$ needed to cover the highest
time-derivative present in the dispersive operator.

{\bf [Step 1 -- nonlinear propagation]} As the first step, we ignore the
dispersive operator and set it to zero,
$$
  \hat{\fam\bbb D}={\bf 0}
  \qquad\Rightarrow\qquad
  {{\partial{\bf a}}\over{\partial\zeta}} = \hat{\fam\bbb N}{\bf a}.
  \eqdef{eq:numsol80}
$$
In other words, sort of pretend that we are dealing with an ordinary
differential equation (ODE) in one single variable $\zeta$, albeit being
a nonlinear equation in the field envelopes.
For a sufficiently small step $\Delta\zeta$ in the normalized spatial
coordinate $\zeta$, we may approximate $\hat{\fam\bbb N}$ by being
constant over the small interval. In this case, an exact solution to
Eq.~\eqref{eq:numsol80} becomes
$$
  {\bf a}(\zeta+\Delta\zeta,s)
    =\exp(\hat{\fam\bbb N}\Delta\zeta){\bf a}(\zeta,s),
  \eqdef{eq:numsol90}
$$
in which $\exp(\hat{\fam\bbb N}\Delta\zeta)$ should be interpreted as the
{\it matrix exponential};\numberedfootnote{In analogy with the classic,
  scalar exponential function, the {\it matrix exponential} of a matrix
  $\fam\bbb X$ can be defined as
  $$
    \exp({\fam\bbb X})=\sum^{\infty}_{n=0}{{1}\over{n!}}{\fam\bbb X}^n,
  $$
  where the first term ${\fam\bbb X}^0={\fam\bbb I}$ as usually is to be
  interpreted as the identity matrix.
  See, for example, {\tt https://mathworld.wolfram.com/MatrixExponential.html}
  or the excellent episode {\it A Number to the Power of a Matrix} by
  Numberphile on Youtube at {\tt https://www.youtube.com/watch?v=CHozRTwHInE}}
that is to say, {\it not} as the common element-wise exponential. Notice that
this exponential involves the envelopes of the fields present at the current
position $\zeta$, and will change over the course of simulation; hence the
exponential accordingly will need to be re-computed for each step, and cannot
in general be pre-computed.

\numberedfootnote{Notice that we here define the temporal Fourier
  transform $f(\omega)$ and its inverse $f(t)$ by
  $$
    \eqalignno{
      f(\omega)&=\fourier[f(t)]
        ={{1}\over{2\pi}}
        \int^{\infty}_{-\infty}f(\tau)\exp(i\omega\tau)\,d\tau,
        \eqdefn{eq:fourier10}&\eqsubdef{eq:fourier10a}\cr
      f(t)&=\fourier^{-1}[f(\omega)]
        =\int^{\infty}_{-\infty}f(\omega)\exp(-i\omega t)\,d\omega,
        &\eqsubdef{eq:fourier10b}\cr
    }
  $$
  conforming to the standard notation in optics, say Born and
  Wolf.\numberedfootnote{Max Born and Emil Wolf, {\it Principles of Optics},
  7th Edn.~(Cambridge University Press, 1999). ISBN 978-0521642224.}
  Notice that by using this convention, derivatives in time will appear
  with a {\it negative} ``$-i\omega$'' factor in the Fourier domain as
  $$
    \fourier\Big[{{\partial f(t)}\over{\partial t}}\Big]
    =-i\omega\fourier[f(t)].
  $$
}
\vfill\eject

\section{Example gallery of pulsed colinear OPA}
\subsection{Parameters used in the simulation}
The definition of the normalized length used in the simulation is from
Eq.~\eqref{eq:waveeq310},
$$
  \zeta = |k''_3|z/\tau^2,\qquad
  s=(t-k'_3 z)/\tau,
$$
interlinking the physical propagation length $z_{\rm max}$, pulse duration $\tau$
and group velocity dispersion parameter $|k''_3|$ for the pump.
For a choice of a physical propagation length of $z_{\rm max}=10\ {\rm mm}$ and a
pulse duration of $\tau=10\ {\rm ps}=10\times10^{-12}\ {\rm s}$ for the pump,
which also is used as a reference for the idler and signal, a range of
$\zeta_{\rm max}=1$ for the normalized length corresponds to a group dispersion
parameter of $|k''_3|=10.0\times10^{-21}\ {\rm s}^2/{\rm m}$.
In this respect, the group dispersion parameter is highly unknown, but we choose
this value as reference. In any case, we may just as well choose a longer pulse
duration $\tau$ in case that the chosen group velocity dispersion parameter
$|k''_3|$ is found to be too low for a real medium.

For the initial pulses, we choose these all as linearly polarized, and we further choose normalized amplitudes for the fields of Eq.~\eqref{eq:waveeq370} as
$\max(\tilde{a}^{f\pm}_{\omega_1}(\zeta,0))=0$ for the idler,
$\max(\tilde{a}^{f\pm}_{\omega_2}(\zeta,0))=0.5$ for the signal, and
$\max(\tilde{a}^{f\pm}_{\omega_3}(\zeta,0))=2.5$ for the pump. Notice that these are amplitudes, hence the intensity measures of these go as the respective {\it squares} of these amplitudes.
The corresponding initial pulse widths are all chosen equal as 10~ps, and the pulses are also of ${\rm sech}(\ldots)$-shape.

For the practical implementation of the simulation, step parameters of
$N_{\zeta}=4001$, $\Delta\zeta=0.0005$ for the discretized normalized spatial step in $\zeta$, and $N_s=2^{11}$, $\Delta s=0.01$ for the dicretized normalized ``time'' $s$.

We continue with the definition of the dispersion parameters of the dispersive
$\hat{D}$ operator.
Notice that the sign of the group-velocity dispersion parameters $k''_j=d^2 k_k/d\omega^2>0$, a long-wavelength, or equivalently low-frequency, pulse travels faster than a short-wavelength, or high-frequency, pulse.
Notice that the group velocity dispersion parameter $|k''_3|$ for the pump {\it always needs to be non-zero}, due to the way that the normalization of the system of PDE:s is constructed.

The electric-dipolar part of group velocity parameters $k'_j$ is for the pump taken as $k'_3 = 5.0\times10^{-9}\ {\rm s}/{\rm m}$, corresponding to a group velocity of $2/3$ of the speed or light ($v_{g,3}=1/k'_3 = 2.0\times10^8\ {\rm m}/{\rm s}$), or equivalently with a reasonable group velocity index of $n_{g}=1.5$.
For the idler and signal pulses, we assume that these travel at a speed roughly 5 percent faster than the pump, hence with $k'_1 = k'_2 = 0.95\times k'_3$, hence with group velocities $v_{g,1}=v_{g,2}\approx 1.05\times v_{g,3}$.

For the electric-dipolar part of group velocity dispersion parameters $k''_j$ for the idler, signal and pump, we all chose these as equal, with value $k''_1 =k''_2 =k''_3 =d(k'_3)/d\omega=10\times10^{-21}\ {\rm s}^2/{\rm m}$. Again, please notice that $k''_3$ always must be non-zero, due to the way that the normlization of the coupled OPA system is constructed.

For the chiral differential corrections/contributions $a_j$ to the group velocity parameters $a'_j$ for idler, signal and pump, we choose these such that $a'_1=0.15\times k'_1\ {\rm s}/{\rm m}$. For the moment, this is probably an unrealistically high value, but we will currently use this for illustration of the differential group velocity between LCP/RCP modes.

As for the chiral differential corrections/contributions to the group velocity dispersion parameters $a''_j$ for idler, signal and pump, we choose all these as zero for the moment, as these are not the primary parameters of interest. In other words, $a''_1=a''_1=a''_1=0\ {\rm s}^2/{\rm m}$.

Finally, for the phase mismatch, we pick values $\delta k=100.0\ {\rm m}^{-1}$
and $\delta\alpha=200.0\ {\rm m}^{-1}$, resulting in a normalized phase mismatch from Eq.~\eqref{eq:waveeq350} of
$$
  \Delta\phi_{\pm}\equiv(\Delta k\pm\Delta\alpha) \tau^2/|k''_3|
     ={{(10\times10^{-12})^2\ {\rm s}^2}
        \over{10.0\times10^{-21}\ {\rm s}^2/{\rm m}}}
      \times((100.0\pm 200.0)\ {\rm m}^{-1})
      =\cases{3.0&\cr -1.0&}
$$
The behaviour of this system is illustrated in the following next sections.
\vfill\eject

\subsection{Pulse propagation with $\Delta\alpha=0$}
\centerline{\epsfxsize=246pt\epsfbox{figs/figure-1-stoke-s0.eps}}
\noindent
{\captionwide{\bf Figure~1.} Normalized Stokes parameter
  $S_0=|a^{f+}_{\omega_j}|^2+|a^{f-}_{\omega_j}|^2$ (total intensity) for
  the idler, signal and pump pulses, resolved in normalized time
  $s=(t-k'_3 z)/\tau$ and along normalized spatial coordinate
  $\zeta=|k''_3|z/\tau^2$, in the case where $\Delta\alpha=0\ {\rm m}^{-1}$.}
\bigskip
\centerline{\epsfxsize=246pt\epsfbox{figs/figure-1-stoke-s3.eps}}
\noindent
{\captionwide{\bf Figure~2.} The corresponding normalized Stokes parameter
  $S_3=(|a^{f+}_{\omega_j}|^2+|a^{f-}_{\omega_j}|^2)/S_0$ (ellipticity) for the
  idler, signal and pump pulses, corresponding to the same parameters as
  in Fig.~1.}
\vfill\eject

\subsection{Pulse propagation with $\Delta\alpha>0$}
\centerline{\epsfxsize=246pt\epsfbox{figs/figure-2-stoke-s0.eps}}
\noindent
{\captionwide{\bf Figure~3.} Identical parameters as for Fig.~1, with exception
  for a non-zero and positive $\Delta\alpha=200\ {\rm m}^{-1}$.}
\bigskip
\centerline{\epsfxsize=246pt\epsfbox{figs/figure-2-stoke-s3.eps}}
\noindent
{\captionwide{\bf Figure~4.} The corresponding normalized Stokes parameter
  $S_3=(|a^{f+}_{\omega_j}|^2+|a^{f-}_{\omega_j}|^2)/S_0$ (ellipticity) for the
  idler, signal and pump pulses, corresponding to the same parameters as
  in Fig.~3.}
\vfill\eject

\subsection{Pulse propagation with $\Delta\alpha>0$}
\centerline{\epsfxsize=246pt\epsfbox{figs/figure-3-stoke-s0.eps}}
\noindent
{\captionwide{\bf Figure~5.} Identical parameters as for Figs.~1 and~3, with
  exception for a non-zero and negative $\Delta\alpha=-200\ {\rm m}^{-1}$.}
\bigskip
\centerline{\epsfxsize=246pt\epsfbox{figs/figure-3-stoke-s3.eps}}
\noindent
{\captionwide{\bf Figure~6.} The corresponding normalized Stokes parameter
  $S_3=(|a^{f+}_{\omega_j}|^2+|a^{f-}_{\omega_j}|^2)/S_0$ (ellipticity) for the
  idler, signal and pump pulses, corresponding to the same parameters as
  in Fig.~5.}
\vfill\eject

\bye
